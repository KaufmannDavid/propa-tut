\documentclass{beamer}
\usetheme{metropolis}

\usepackage[ngerman]{babel}
\usepackage[autostyle=true,german=quotes]{csquotes}
\usepackage[linewidth=1pt]{mdframed}
\usepackage{hyperref}
\usepackage{makecell}
\usepackage{pifont}
\usepackage{tikz}
\usetikzlibrary{positioning, calc, arrows, fit, decorations.pathreplacing, shapes, shapes.multipart, snakes}
\usepackage{verbatim}
\usepackage{tabularx}
\usepackage{textcomp}
%\usepackage{pdfpages}

\batchmode

\hypersetup{
	colorlinks,
	urlcolor=blue,
	linkcolor=black % for ToC
}
\newenvironment{qaa}[1]{
	#1

	\begin{mdframed}
		\small
}{
	\end{mdframed}
}

\newcommand{\true}{\ding{51}}
\newcommand{\false}{\ding{55}}
\newcommand{\code}[1]{
	\begin{mdframed}
		\verbatiminput{#1}
	\end{mdframed}
}

\title{Tutorium 05: $\lambda$-Kalkül}
% \subtitle{}
\author{Paul Brinkmeier}
\institute{Tutorium Programmierparadigmen am KIT}
\date{18. November 2019}

\begin{document}

\begin{frame}
	\titlepage
\end{frame}

\section{Heutiges Programm}
\begin{frame}{Programm}
	\begin{itemize}
		\item Übungsblatt 4
		\item $\lambda$-Kalkül: Basics + Church-Zahlen
		\pause
		\item $\lambda$-Kalkül in Haskell
	\end{itemize}
\end{frame}

\section{Übungsblatt 4}

\begin{frame}{2.1, 2.3 --- AST: Datenstruktur}
	\code{demos/AstType.hs}

	\begin{itemize}
		\item \texttt{t} ist Typvariable, um bspw. \texttt{Int}s als Namen zuzulassen
		\item Das kommt bspw. bei Compiler-Optimierungen zum Einsatz
	\end{itemize}
\end{frame}

\begin{frame}{2.2 --- AST: Auswertung}
	\code{demos/AstEval.hs}
\end{frame}

\begin{frame}{2.3 --- AST: Boolsche Ausdrücke}
	\code{demos/AstEval2.hs}

	\begin{itemize}
		\item Aufgabe sorgfältig lesen, nur 0 ist \enquote{falsey} in C
		\item $\leadsto$ kann einem in der Klausur in den Arsch beißen
	\end{itemize}
\end{frame}

\begin{frame}{2.4 --- AST: \texttt{Show}}
	\code{demos/AstShow.hs}

	\begin{itemize}
		\item \texttt{Show t => Show (Exp t)} $\Leftrightarrow$ \enquote{Wenn man \texttt{t}s anzeigen kann, kann man auch \texttt{Exp t}s anzeigen}
	\end{itemize}
\end{frame}

\section{Wiederholung}

\begin{frame}{Algebraische Datentypen}
	\code{demos/DataExamples.hs}

	\begin{itemize}
		\item Keyword \text{data} definiert \emph{neuen} Typ
		\item \enquote{\texttt{enum} auf Meth}
	\end{itemize}
\end{frame}

\begin{frame}{Typklassen}
	\code{demos/TypeClassExamples.hs}

	\begin{itemize}
		\item Typklassen stellen globale Operationen für Typen bereit
		\item Bspw. \texttt{Eq} und \texttt{Ord} für Vergleiche, \texttt{Enum} für Aufzählbarkeit
	\end{itemize}
\end{frame}

\section{$\lambda$-Kalkül}

\begin{frame}{$\lambda$-Terme}
	Ein Term im $\lambda$-Kalkül hat eine der drei folgenden Formen:

	\vspace{0.5cm}

	\begin{tabularx}{\textwidth}{ X | X | X }
		\textbf{Notation} & \textbf{Besteht aus}                      & \textbf{Bezeichnung} \\
		\hline
		$x$               & $x$ : Variablenname                       & Variable             \\
		\hline
		$\lambda{}p.b$    &
			\begin{tabular}[t]{@{}c@{}}$p$ : Variablenname\\$b$ : $\lambda$-Term\end{tabular}
									      & Abstraktion          \\
		\hline
		$f$ $a$           & $f$, $a$ : $\lambda$-Terme                & Funktionsanwendung   \\
	\end{tabularx}

	\vspace{0.5cm}

	\begin{itemize}
		\item \enquote{$\lambda$-Term}: rekursive Datenstruktur
		\item Semantik definieren wir später
		\pause
		\item Jetzt: Ergänzt das Modul \texttt{Lambda} um die fehlenden Typen
		\begin{itemize}
			\item +Fragen zur ÜB-Korrektur
		\end{itemize}
	\end{itemize}
\end{frame}

\begin{frame}{$\lambda$-Terme in Haskell}
	\code{demos/Lambda.hs}

	\begin{itemize}
		\item \url{//github.com/pbrinkmeier/pp-tut}
		\item Modul \texttt{x} liegt in \texttt{slides/demos/x.hs}
	\end{itemize}
\end{frame}

\end{document}
