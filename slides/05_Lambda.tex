\documentclass{beamer}

\usetheme{metropolis}

\usepackage[ngerman]{babel}
\usepackage[autostyle=true,german=quotes]{csquotes}
\usepackage[linewidth=1pt]{mdframed}
\usepackage{hyperref}
\usepackage{makecell}
\usepackage{pifont}
\usepackage{tikz}
\usetikzlibrary{positioning, calc, arrows, fit, decorations.pathreplacing, shapes, shapes.multipart, snakes}
\usepackage{verbatim}
\usepackage{textcomp}
\usepackage{centernot}
\usepackage{tabularx}
\usepackage{ulem}
%\usepackage{pdfpages}

\batchmode

\hypersetup{
	colorlinks,
	urlcolor=blue,
	linkcolor=black % for ToC
}
\newenvironment{qaa}[1]{
	#1

	\begin{mdframed}
		\small
}{
	\end{mdframed}
}

\newcommand{\true}{\ding{51}}
\newcommand{\false}{\ding{55}}
\newcommand{\code}[1]{
	\begin{mdframed}
		\verbatiminput{#1}
	\end{mdframed}
}


\title{Tutorium 05: $\lambda$-Kalkül}
% \subtitle{}
\author{Paul Brinkmeier}
\institute{Tutorium Programmierparadigmen am KIT}
\date{8. Dezember 2020}

\begin{document}

\begin{frame}
	\titlepage
\end{frame}

\section{Heutiges Programm}
\begin{frame}{Programm}
	\begin{itemize}
		\item Übungsblatt 3
		\item $\lambda$-Kalkül: Wiederholung der Basics
                \item $\lambda$-Kalkül: Auswertungsstrategien, Church-Zahlen, SKI
	\end{itemize}
\end{frame}

\section{Übungsblatt 3}

\begin{frame}{1 --- Typen und Typklassen in Haskell}
  \begin{figure}
    \texttt{fun7 x = \emph{if} show x /= [] \emph{then} x \emph{else} error}
  \end{figure}

  \vfill

  Ansatz?

  \pause

  \begin{itemize}
    \item \texttt{error :: String -> a}
    \item Wegen \texttt{\emph{if}}: Auch \texttt{x :: String -> a}
    \item Wegen \texttt{show x}: Constraint \texttt{Show (String -> a)}
    \item Also: \texttt{fun7 :: Show (String -> a) => (String -> a) -> String -> a}
  \end{itemize}
\end{frame}

\begin{frame}{2.1, 2.3 --- AST: Datenstruktur}
	\code{../demos/AstType.hs}

	\begin{itemize}
		\item \texttt{t} ist Typvariable, um bspw. \texttt{Int}s als Namen zuzulassen
		\item Das kommt bspw. bei Compiler-Optimierungen zum Einsatz
	\end{itemize}
\end{frame}

\begin{frame}{2.2 --- AST: Auswertung}
	\code{../demos/AstEval.hs}

        \begin{itemize}
          \item Zum Auswerten der Summe müssen erstmal die Summanden bekannt sein $\leadsto$ rekursiv \texttt{eval} aufrufen
        \end{itemize}
\end{frame}

\begin{frame}{2.3 --- AST: Boolsche Ausdrücke}
	\code{../demos/AstEval2.hs}

	\begin{itemize}
		\item Aufgabe sorgfältig lesen, nur 0 ist \enquote{falsey} in C
                \item Alternative Lösung: \texttt{bool2int}/\texttt{int2boot} implementieren
	\end{itemize}
\end{frame}

\begin{frame}{2.4 --- AST: \texttt{Show}}
	\code{../demos/AstShow.hs}

	\begin{itemize}
		\item \texttt{Show t => Show (Exp t)}: \enquote{Wenn man ein \texttt{t} anzeigen kann, kann man auch eine \texttt{Exp t} anzeigen}
	\end{itemize}
\end{frame}

\begin{frame}{3.1 --- \texttt{ropeLength}}
	\code{../demos/RopeLength.hs}
\end{frame}

\begin{frame}{3.2 --- \texttt{ropeConcat}}
	\code{../demos/RopeConcat.hs}
\end{frame}

\begin{frame}{3.3 --- \texttt{ropeSplitAt}}
	\code{../demos/RopeSplitAt.hs}
\end{frame}

\begin{frame}{3.4 --- \texttt{ropeInsert}}
	\code{../demos/RopeInsert.hs}
\end{frame}

\begin{frame}{3.5 --- \texttt{ropeDelete}}
	\code{../demos/RopeDelete.hs}
\end{frame}

\section{Wiederholung}

\begin{frame}{Cheatsheet: Lambda-Kalkül/Basics}
  \begin{itemize}
    \item Terme $t$: Variable ($x$), Funktion ($\lambda x . t$), Anwendung ($t \; t$)
    \item \emph{$\alpha$-Äquivalenz}: Gleiche Struktur
    \item \emph{$\eta$-Äquivalenz}: Unterversorgung
    \item \emph{Freie Variablen}, \emph{Substitution}, \emph{RedEx}
    \item \emph{$\beta$-Reduktion}: \\
          $(\lambda{}p.b)$ $t \Rightarrow b\left[p\rightarrow{}t\right]$
  \end{itemize}
\end{frame}

\section{Church-Zahlen}

\begin{frame}{Peano-Axiome}
	\begin{eqnarray*}
		c_0 &= ?\\
		c_1 &= s (c_0)\\
		c_2 &= s (s (c_0))\\
		c_3 &= s (s (s (c_0)))\\
		c_8 &= s (s (s (s (s (s (s (s (c_0))))))))
	\end{eqnarray*}

	\begin{enumerate}
		\item Die 0 ist Teil der natürlichen Zahlen
		\item Wenn $n$ Teil der natürlichen Zahlen ist,\\
	 	      ist auch $s(n) = n + 1$ Teil der natürlichen Zahlen
	\end{enumerate}
\end{frame}

\begin{frame}{Church-Zahlen}
	\begin{itemize}
		\item \enquote{Zahlen} im $\lambda$-Kalkül werden durch Funktionen in Normalform dargestellt
		\item $c_n$ $f$ $x =$ $f$ $n$-mal angewendet auf $x$
		\item Bspw. $(c_3$ $g$ $y) = g$ $(g $ $(g$ $y)) = g^3$ $y$\\
		      Mit $c_3 = \lambda{}f.\lambda{}x.f$ $(f $ $(f$ $x))$
		\item Schreibt eine $\lambda$-Funktion $succ$, die eine Church-Zahl nimmt und zu deren Nachfolger auswertet
              \end{itemize}
              \pause

              \begin{equation*}
                \csucc = \lam{n}{\lam{s}{\lam{z}{
                  \app{s}{(
                    \app{n}{\app{s}{z}}
                  )}
                }}}
              \end{equation*}
\end{frame}

\section{Auswertungsstrategien}

\begin{frame}{Auswertungsstrategien}
        \begin{equation*}
          \overbrace{\app{
              \uline{\csucc}
            }{
              \overbrace{(\app{\uline{\csucc}}{\cn{0}})}^{\text{Redex 2}}
            }}^{\text{Redex 1}}
        \end{equation*}

        mit

        \begin{eqnarray*}
          \cn{0} =& \lam{s}{\lam{z}{z}} \\
          \csucc =& \lam{n}{\lam{s}{\lam{z}{\app{s}{(\app{\app{n}{s}}{z})}}}}
        \end{eqnarray*}

	\begin{itemize}
		\item Welcher Redex soll zuerst ausgewertet werden?
		\item $\leadsto$ verschiedene Auswertungsstrategien
	\end{itemize}
\end{frame}

\begin{frame}{Normalreihenfolge}
        \begin{eqnarray*}
          &
          \app{
            \uline{\csucc}
          }{
            (\app{\uline{\csucc}}{\cn{0}})
          } \\
          \Rightarrow_\beta&
          \lam{s}{\lam{z}{\app{s}{(\app{\app{(\app{\uline{\csucc}}{\cn{0}})}{s}}{z})}}} \\
          \Rightarrow_\beta&
          \lam{s}{\lam{z}{
            \app{s}{
              (\app{\app{\uline{(\lam{s}{\lam{z}{
                \app{s}{(\app{\app{\uline{\cn{0}}}{s}}{z})}
              }})}}{s}}{z})
            }
          }} \\
          \Rightarrow_\beta^2&
          \lam{s}{\lam{z}{
            \app{s}{(\app{s}{(\app{\app{\uline{\cn{0}}}{s}}{z})})}
          }} \\
          \Rightarrow_\beta^2&
          \lam{s}{\lam{z}{
            \app{s}{(\app{s}{z})}
          }} \centernot\implies
        \end{eqnarray*}

        \vfill

        Normalreihenfolge: Linkester Redex zuerst.
\end{frame}

\begin{frame}{Call-by-Name}
        \begin{eqnarray*}
          &
          \app{
            \uline{\csucc}
          }{
            (\app{\uline{\csucc}}{\cn{0}})
          } \\
          \Rightarrow_\beta&
          \lam{s}{\lam{z}{\app{s}{(\app{\app{(\app{\uline{\csucc}}{\cn{0}})}{s}}{z})}}} \centernot\implies_{\text{CbN}}
        \end{eqnarray*}

        \vfill

        Call-by-Name: Linkester Redex zuerst, aber:

        \begin{itemize}
          \item Funktionsinhalte werden nicht weiter reduziert
          \item $\leadsto$ Betrachte nur Redexe, die nicht von einem $\lambda$ umgeben sind
          \item So funktioniert auch Laziness in Haskell (mit Auflagen)
        \end{itemize}
\end{frame}

\begin{frame}{Call-by-Value}
        \begin{eqnarray*}
          &
          \app{
            \uline{\csucc}
          }{
            (\app{\uline{\csucc}}{\cn{0}})
          } \\
          \Rightarrow_\beta&
          \app{
            \uline{\csucc}
          }{
            (\lam{s}{\lam{z}{
              \app{s}{
                (\app{\app{\uline{\cn{0}}}{s}}{z})
              }
            }})
          } \\
          \Rightarrow_\beta&
          \lam{s}{\lam{z}{
          \app{\app{
            \uline{(\lam{s}{\lam{z}{
              \app{s}{
                (\app{\app{\uline{\cn{0}}}{s}}{z})
              }
            }})}
          }{s}}{z}
          }} \centernot\implies_\text{CbV}
        \end{eqnarray*}
        \vfill

        Call-by-Name: Linkester Redex zuerst, aber:

        \begin{itemize}
          \item Funktionsinhalte werden nicht weiter reduziert
          \item $\leadsto$ Betrachte nur Redexe, die nicht von einem $\lambda$ umgeben sind
          \item Berechne Argumente vor dem Einsetzen
          \item $\leadsto$ Betrachte nur Redexe, deren Argument \emph{unter CbV} nicht weiter reduziert werden muss
        \end{itemize}
\end{frame}

\begin{frame}{Cheatsheet: Lambda-Kalkül/Fortgeschritten}
  \begin{itemize}
    \item Auswertungsstrategien (von lässig nach streng):
    \begin{itemize}
      \item \emph{Volle $\beta$-Reduktion}
      \item \emph{Normalreihenfolge}
      \item \emph{Call-by-Name}
      \item \emph{Call-by-Value}
    \end{itemize}
    \item Datenstrukturen:
    \begin{itemize}
      \item \emph{Church-Booleans}
      \item \emph{Church-Zahlen}
      \item \emph{Church-Listen}
    \end{itemize}
    \item Rekursion durch \emph{Y-Kombinator}
  \end{itemize}
\end{frame}

\section{Klausuraufgaben zum $\lambda$-Kalkül}

\begin{frame}{19SS A4 --- SKI-Kalkül (13P.)}
	\begin{eqnarray*}
          S =& \lam{x}{\lam{y}{\lam{z}{\appp{x}{z}{(\app{y}{z})}}}} \\
		K =& \lam{x}{\lam{y}{x}} \\
		I =& \lam{x}{x}
	\end{eqnarray*}

	\begin{itemize}
		\item SKI-Kalkül kann alles, was $\lambda$-Kalkül auch kann, allein mit den Kombinatoren $S$, $K$ und $I$
		\item Definiere $U = \lam{x}{\appp{x}{S}{K}}$
		\item Aufgabe: Beweise, dass man $S$, $K$ und $I$ durch $U$ darstellen kann:
		\pause
		\begin{itemize}
			\item $\appp{U}{U}{x} \stackrel{?}{\implies} x$
                        \item $\app{U}{(\app{U}{(\app{U}{U})})} = \app{U}{(\app{U}{I})} \stackrel{?}{\implies} K$
                        \item $\app{U}{(\app{U}{(\app{U}{(\app{U}{U})})})} = \app{U}{K} \stackrel{?}{\implies} S$
		\end{itemize}
	\end{itemize}
\end{frame}

\begin{frame}{18SS A4 --- Currying im $\lambda$-Kalkül (13P.)}
	\begin{eqnarray*}
		\cpair =& \lam{a}{\lam{b}{\lam{f}{\appp{f}{a}{b}}}} \\
                \cfst  =& \lam{p}{\app{p}{(\lam{x}{\lam{y}{x}})}} \\
                \csnd  =& \lam{p}{\app{p}{(\lam{x}{\lam{y}{y}})}} \\
                \app{\cfst}{(\appp{\cpair}{a}{b})} =& a \\
                \app{\csnd}{(\appp{\cpair}{a}{b})} =& b \\
	\end{eqnarray*}

	\begin{itemize}
		\item Schreibe $\ccurry$ und $\cuncurry$, sodass:
		\begin{itemize}
                  \item $\appp{(\app{\ccurry}{f})}{a}{b} = \appp{f}{(\appp{\cpair}{a}{b})}$
                  \item $\app{(\app{\cuncurry}{g})}{(\appp{\cpair}{a}{b})} = \appp{g}{a}{b}$
		\end{itemize}
	\end{itemize}
\end{frame}

\begin{frame}{18WS A5 --- Listen im $\lambda$-Kalkül (10P.)}
	\begin{eqnarray*}
		\cnil  =& \lam{n}{\lam{c}{n}} \\
                \ccons =& \lam{x}{\lam{xs}{\lam{n}{\lam{c}{(\appp{c}{x}{xs})}}}}
	\end{eqnarray*}

	\begin{itemize}
		\item Schreibe $head$ und $tail$, sodass:
		\begin{itemize}
                  \item $\app{\chead}{(\appp{\ccons}{A}{B})} \stackrel{*}{\implies} A$
                  \item $\app{\ctail}{(\appp{\ccons}{A}{B})} \stackrel{*}{\implies} B$
		\end{itemize}
		\pause
		\item Schreibe $\creplicate$, sodass:
		\begin{itemize}
                  \item $\appp{\creplicate}{\cn{n}}{A} = \underbrace{\appp{\ccons}{A}{(\appp{\ccons}{A}{... (\appp{\ccons}{A}{\cnil})})}}_\text{$n$ mal}$
		\end{itemize}
              \item Erinnerung: $\appp{\cn{n}}{f}{x} = \underbrace{\app{f}{(\app{f}{... (\app{f}{x})})}}_\text{$n$ mal}$
		\pause
              \item Werte aus: $\appp{\creplicate}{\cn{3}}{A} \stackrel{*}{\implies} ?$
	\end{itemize}
\end{frame}

\end{document}
