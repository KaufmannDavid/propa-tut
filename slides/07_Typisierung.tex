\documentclass{beamer}
\usetheme{metropolis}

\usepackage[ngerman]{babel}
\usepackage[autostyle=true,german=quotes]{csquotes}
\usepackage[linewidth=1pt]{mdframed}
\usepackage{hyperref}
\usepackage{makecell}
\usepackage{pifont}
\usepackage{tikz}
\usetikzlibrary{positioning, calc, arrows, fit, decorations.pathreplacing, shapes, shapes.multipart, snakes}
\usepackage{verbatim}
\usepackage{tabularx}
\usepackage{textcomp}
\usepackage{centernot}
\usepackage{amsmath}
\usepackage{xcolor}
\usepackage{tikz}
%\usepackage{pdfpages}

\batchmode

\hypersetup{
	colorlinks,
	urlcolor=blue,
	linkcolor=black % for ToC
}
\newenvironment{qaa}[1]{
	#1

	\begin{mdframed}
		\small
}{
	\end{mdframed}
}

\newcommand{\true}{\ding{51}}
\newcommand{\false}{\ding{55}}
\newcommand{\code}[1]{
	\begin{mdframed}
		\verbatiminput{#1}
	\end{mdframed}
}

\title{Tutorium 07: Typisierung}
% \subtitle{}
\author{Paul Brinkmeier}
\institute{Tutorium Programmierparadigmen am KIT}
\date{03. Dezember 2019}

\begin{document}

\begin{frame}
	\titlepage
\end{frame}

\section{Heutiges Programm}

\begin{frame}{Programm}
	\begin{itemize}
		\item ÜBs 5 und 6
		\item Typisierter $\lambda$-Kalkül
		\item Einführung in Prolog
	\end{itemize}
\end{frame}

\begin{frame}{Fragen von letzter Woche}
	\begin{itemize}
		\item Zwischenschritte beim SKI-Kalkül müssen \emph{nicht} angegeben werden
		\pause
		\item Nicht mehr relevante Aufgaben $\leadsto$ kommt auf Vorlesung an
		\begin{itemize}
			\item Alles, was in der Vorlesung angesprochen wird, ist relevant
			\pause
			\item Bis auf Z-Folien
		\end{itemize}
		\pause
		\item Freie Variablen sind relevant für $\alpha$-Äquivalenz
		\begin{itemize}
			\item Kommt aber so in Klausuraufgaben nicht vor
		\end{itemize}
		\pause
		\item \enquote{Standardbibliothek} für $\lambda$-Kalkül
		\begin{itemize}
			\item Alles von den ÜBs
			\item Sonstiges nur mit Quellenangabe
			\begin{itemize}
				\item Bspw. ProPa Folie X, ProPa Klausur SS18 Seite Y
			\end{itemize}
		\end{itemize}
	\end{itemize}
\end{frame}

\newcommand{\aeq}{\stackrel{\alpha}{=}}
\newcommand{\naeq}{\stackrel{\alpha}{\neq}}
\newcommand{\eeq}{\stackrel{\eta}{=}}

\newcommand{\E}{\;}

\newcommand{\liin}[2]{#1\E{}#2}
\newcommand{\liiin}[3]{#1\E{}#2\E{}#3}
\newcommand{\livn}[4]{#1\E{}#2\E{}#3\E{}#4}
\newcommand{\lvn}[5]{#1\E{}#2\E{}#3\E{}#4\E{}#5}

\newcommand{\lii}[2]{(#1\E{}#2)}
\newcommand{\liii}[3]{(#1\E{}#2\E{}#3)}

\newcommand{\liir}[2]{\textcolor{red}{\underline{(}}#1\E{}#2\textcolor{red}{\underline{)}}}
\newcommand{\liiir}[3]{\textcolor{red}{\underline{(}}#1\E{}#2\E{}#3\textcolor{red}{\underline{)}}}

\newcommand{\subst}[3]{(#1)\left[#2\,\to\,#3\right]}

\newcommand{\abs}[2]{\lambda{}#1.#2}

\section{Übungsblatt 5}

\subsection{1.5 --- $\beta$-Reduktion}

% Häufiger Fehler: 6 lässt sich wegen Klammerung nicht machen

\begin{frame}{1.5 --- $\beta$-Reduktion}
	Gegeben war:

	\begin{equation*}
		\lvn{(\abs{a}{a})}{(\abs{b}{b})}{\lii{(\abs{c}{c})}{\liii{(\abs{d}{d})}{(\abs{e}{e})}{(\abs{f}{f})}}}{\abs{g}{g}}{\lii{(\abs{h}{h})}{(\abs{i}{i})}}
	\end{equation*}

	\begin{itemize}
		\item Vorgehensweise: Redexe markieren, Termliste durchsuchen
		\item Häufigster Fehler: $\lii{(\abs{h}{h})}{(\abs{i}{i})}$ kann man in $\abs{g}{g}$ \emph{nicht} reinziehen (Funktionsaufrufe sind linksassoziativ)
	\end{itemize}
\end{frame}

\subsection{4 --- Church-Paare}

\newcommand{\reducesTo}[1]{\stackrel{#1}{\implies}}

\begin{frame}{4 --- Church-Paare}
	\begin{eqnarray*}
		pair &= \abs{a}{\abs{b}{\abs{f}{\liiin{f}{a}{b}}}} \\
		\liiin{pair}{c_{42}}{c_{100}} &\reducesTo{2} \abs{f}{\liiin{f}{c_{42}}{c_{100}}}
	\end{eqnarray*}

	\begin{itemize}
		\item $\leadsto$ \emph{Destructuring}/\emph{Pattern Matching}/Fallunterscheidung durch Aufruf einer Funktion mit den Elementen des Tupels
		\item Wie bei den Listen von letzter Woche
	\end{itemize}

	\begin{eqnarray*}
		fst &= \abs{p}{\liin{p}{(\abs{a}{\abs{b}{a}})}} \\
		snd &= \abs{p}{\liin{p}{(\abs{a}{\abs{b}{b}})}}
	\end{eqnarray*}

	\begin{itemize}
		\item $fst$/$snd$ ruft Tupel mit Funktion auf, die nur ihr erstes/zweites Argument zurückgibt
	\end{itemize}
\end{frame}

\begin{frame}{4 --- Church-Paare}
	Geben Sie $next$ an, sodass $\liin{next}{(n, m)} = (m, m + 1)$.

	\pause

	\begin{equation*}
		next = \abs{p}{\lii{p}{(\abs{n}{\abs{m}{\liiin{pair}{m}{\lii{succ}{m}}})}}}
	\end{equation*}

	In Haskell: \texttt{next (n, m) = (m, succ m)}

	\pause

	\begin{equation*}
		next = \abs{p}{\liiin{pair}{\lii{snd}{p}}{\lii{succ}{\lii{snd}{p}}}}
	\end{equation*}

	In Haskell: \texttt{next p = (snd p, succ (snd p))}
\end{frame}

\begin{frame}{4 --- Church-Paare}
	\begin{eqnarray*}
		pred =& \abs{n}{\liin{fst}{\liii{n}{next}{\liii{pair}{c_0}{c_0}}}} \\
		sub =& \abs{m}{\abs{n}{\liiin{n}{pred}{m}}}
	\end{eqnarray*}

	\begin{itemize}
		\item $next$ wird $n$-mal auf $(0, 0)$ angewendet, erhöht immer das zweite Argument und speichert eine Kopie davon $\leadsto$ im letzten Schritt ist das Tupel $(n - 1, n)$, $fst$ liefert also $n - 1$
		\item $sub$ nimmt einfach $n$-mal den Vorgänger von $m$
	\end{itemize}
\end{frame}

\section{Übungsblatt 6}

\subsection{Typsysteme}

\begin{frame}{Typsystem für $\lambda$-Terme}
	\begin{align*}
		\textrm{Für Variablen $t$:} && \frac{\Gamma{}(t) = \sigma \;\;\;\;\; \sigma \succeq \tau}{\Gamma \vdash t : \tau} \textrm{\textsc{Var}} \\[1em]
		\textrm{Für Aufrufe $\liin{f}{a}$:} && \frac{\Gamma \vdash f : \phi \to \alpha \;\;\;\;\; \Gamma \vdash a : \phi}{\Gamma \vdash \liin{f}{a} : \alpha} \textrm{\textsc{App}} \\[1em]
		\textrm{Für Funktionsterme $\abs{p}{b}$:} && \frac{\Gamma, p : \pi \vdash b : \rho}{\Gamma \vdash \abs{p}{b} : \pi \to \rho} \textrm{\textsc{Abs}}
	\end{align*}

	\begin{itemize}
		\item $\Gamma$ ist ein Typkontext
		\item D.h. \emph{Liste} von Variable/Typ-Tupeln
		\item Bspw. $x : \textrm{\texttt{char}}, y : \textrm{\texttt{int}}$
		\item \textbf{Vorsicht}: $\Gamma$ ist keine Menge, denn $\Gamma$ hat eine Reihenfolge!
	\end{itemize}
\end{frame}

\begin{frame}{\textsc{Var}}
	\begin{itemize}
		\item \tikz[baseline, remember picture]{\node [fill=green!20,draw] (varRetL) {\enquote{Der Typkontext $\Gamma$ enthält einen Typ $\sigma$ für $t$}};}
		\item \tikz[baseline, remember picture]{\node [fill=red!20,draw] (varEatL) {\enquote{$\sigma$ kann mit $\tau$ instanziiert werden}};}
	\end{itemize}

	\begin{equation*}
		\frac{
			\tikz[baseline, remember picture]{\node[fill=green!20,draw] (varRet) {$\Gamma{}(t) = \sigma$};}
			\;\;\;\;
			\tikz[baseline, remember picture]{\node[fill=red!20,draw] (varEat) {$\sigma \succeq \tau$};}
		}{
			\tikz[baseline, remember picture]{\node[fill=blue!20,draw] (varShow) {
				$\Gamma \vdash t : \tau$
			};}
		} \textrm{\textsc{Var}}
	\end{equation*}

	\begin{itemize}
		\item dann gilt:
		\item \tikz[baseline, remember picture]{\node [fill=blue!20,draw] (varShowL) {\enquote{Variable $t$ hat im Kontext $\Gamma$ den Typ $\tau$}};}
		\vspace{0.5em}
		\item $\sigma \succeq \tau$ $\leadsto$ \enquote{$\sigma$ hat $\tau$s Struktur und ist (mind.) allgemeiner}
		\begin{itemize}
			\item $\textrm{int} \to \textrm{int} \succeq \textrm{int} \to \textrm{int}$
			\item $\forall \alpha . \alpha \to \alpha \succeq \textrm{int} \to \textrm{int}$
			\item $\alpha \to \alpha \not\succeq \textrm{int} \to \textrm{int}$
			\item $\textrm{int} \to \textrm{int} \not\succeq \forall \alpha . \alpha \to \alpha$
		\end{itemize}
	\end{itemize}

	\tikz[overlay, remember picture]{
		\draw[->] (varShowL) edge [bend left] (varShow);
		\draw[->] (varRetL) edge [bend right] (varRet);
		\draw[->] (varEatL) edge [bend left] (varEat);
	}
\end{frame}

\newcommand{\tikzmark}[3]{\tikz[baseline, remember picture]{
	\node[fill=#1,draw] (#2) {#3};
}}

\newcommand{\typeRule}[3]{\frac{#2}{#3} \textrm{\textsc{#1}}}

\begin{frame}{\textsc{App}}
	\begin{itemize}
		\item \tikzmark{green!20}{fTypeL}{\enquote{$f$ ist im Kontext $\Gamma$ eine Funktion, die $\phi$s auf $\alpha$s abbildet}}
		\item \tikzmark{red!20}{aTypeL}{\enquote{$a$ ist im Kontext $\Gamma$ ein Term des Typs $\phi$}}
	\end{itemize}
	\begin{equation*}
		\typeRule{App}{
			\tikzmark{green!20}{fType}{$\Gamma \vdash f : \phi \to \alpha$}
			\;\;\;\;\;
			\tikzmark{red!20}{aType}{$\Gamma \vdash a : \phi$}
		}{
			\tikzmark{blue!20}{eType}{$\Gamma \vdash \liin{f}{a} : \alpha$}
		}
	\end{equation*}

	\begin{itemize}
		\item dann gilt:
		\item \tikzmark{blue!20}{eTypeL}{\enquote{$a$ eingesetzt in $f$ ergibt einen Term des Typs $\alpha$}}
	\end{itemize}

	\tikz[overlay, remember picture]{
		\draw[->] (fTypeL) edge [bend right] (fType);
		\draw[->] (aTypeL) edge [bend left] (aType);
		\draw[->] (eTypeL) edge [bend left] (eType);
	}
\end{frame}

\begin{frame}{\textsc{Abs}}
	\begin{itemize}
		\item \tikzmark{red!20}{bodyTypeL}{\enquote{Damit $b$ als Funktion von $p$ typisierbar ist...}}
		\item \tikzmark{green!20}{contextL}{\enquote{... müssen wir den Typ von $p$ in den Kontext einfügen}}
	\end{itemize}

	\begin{equation*}
		\typeRule{Abs}{
			\tikzmark{green!20}{context}{$\Gamma{}, p : \pi$}
			\;
			\tikzmark{red!20}{bodyType}{$\vdash b : \rho$}
		}{
			\tikzmark{blue!20}{absType}{$\Gamma \vdash \abs{p}{b} : \pi \to \rho$}
		}
	\end{equation*}

	\begin{itemize}
		\item dann gilt:
		\item \tikzmark{blue!20}{absTypeL}{\enquote{$\abs{p}{b}$ ist eine Funktion, die $\pi$s auf $\rho$s abbildet}}
	\end{itemize}

	\tikz[overlay, remember picture]{
		\draw[->] (bodyTypeL) edge [bend left] (bodyType);
		\draw[->] (contextL) edge [bend right] (context);
		\draw[->] (absTypeL) edge [bend left] (absType);
	}
\end{frame}

\subsection{3 --- $\lambda$-Terme und ihre Typen}

\begin{frame}{3 --- $\lambda$-Terme und ihre Typen}
	Geben sie für jeden Typ der unten stehenden Tabelle \emph{all} die Terme aus $A$ - $F$ an, die diesen Typ haben können.

	\begin{itemize}
		\item Zeigt, dass $D$ den vierten Typ haben kann.
		\item $D = \abs{x}{\abs{y}{\liin{y}{\lii{x}{y}}}}$
		\item Typ: $((\alpha \to \beta) \to \alpha) \to (\alpha \to \beta) \to \beta$
		\pause
		\item Ansatz (unten links: leerer Kontext):
	\end{itemize}

	\begin{equation*}
		\typeRule{Abs}{
			... \vdash ...
		}{
			\vdash \abs{x}{\abs{y}{\liin{y}{\lii{x}{y}}}} : ((\alpha \to \beta) \to \alpha) \to (\alpha \to \beta) \to \beta
		}
	\end{equation*}
\end{frame}

\subsection{4 --- Typ-Prüfung}

\begin{frame}{4 --- Typ-Prüfung}
	\begin{itemize}
		\item Sehr gute Übungsaufgabe zum Thema
		\item Abgegeben von ca. 3 Leuten :(
		\item Wenn ihr mir die Aufgabe aufs nächste Blatt (oder per Mail schreibt), korrigiere ich sie euch noch zur Übung
	\end{itemize}
\end{frame}

\section{Einführung in Prolog}

\begin{frame}{Prolog --- Umgebung}
	\begin{itemize}
		\item Prolog ist eine Programmiersprache, wenn auch eine \enquote{komische}
		\item $\leadsto$ gut wird man durch Übung
		\item Zum Üben:
		\begin{itemize}
			\item SWI-Prolog --- gängige Prolog-Umgebung
			\item \href{https://swish.swi-prolog.org/}{SWISH} --- SWI-Prolog Web-IDE zum Testen
			\item VIPR, VIPER --- PSE-Tools des IPD, auf der \href{https://pp.ipd.kit.edu/lehre/WS201920/paradigmen/uebung/}{Seite der Übung} verlinkt
		\end{itemize}
	\end{itemize}
\end{frame}

\begin{frame}{How did we get here?}
	\begin{itemize}
		\item Erste Computerprogramme: Pragmatischerweise imperativ
		\begin{itemize}
			\item lat. imperare: befehlen
			\item \enquote{Computer, tu dies, dann das, überspringe 5 Anweisungen}
			\item $\approx$ Turing-Maschine
			\item $\leadsto$ einfacher zu implementieren als \enquote{$\lambda$-Maschine}
		\end{itemize}
		\item Nächster Schritt: Strukturierte/Prozedurale Programmierung
		\begin{itemize}
			\item \enquote{Go To Statement Considered Harmful} (Edsger Dijkstra)
			\item $\approx$ \texttt{if}, \texttt{while}, \texttt{for}, Prozeduren
		\end{itemize}
	\item \enquote{Große} Programme ($\geq 10$ kLoC) $\leadsto$ Objektorientierte Prog.
		\begin{itemize}
			\item Objekte + Design Patterns, um Komponenten abzugrenzen
			\item Letztlich \enquote{syntactical sugar} für imperative Programme
		\end{itemize}
	\end{itemize}
\end{frame}

\begin{frame}{How did we get here?}
	\begin{itemize}
		\item Imperative Sprachen:
		\begin{itemize}
			\item Rein imperativ $\approx$ Programm ist Liste von Befehlen (MIMA)
			\item Strukturiert $\approx$ Programm ist Liste von Prozeduren (C)
			\item Objektorientiert $\approx$ Programm ist gegeben durch das Verhalten von Objekten (Java)
		\end{itemize}
		\item Gegenstück? \pause \enquote{Deklarative} Sprachen
		\begin{itemize}
			\item lat. declarare: kenntlich machen, erklären
			\item \enquote{Computer, so kannst du das Problem lösen; mach mal}
			\item Bspw. \texttt{fibs =\\0 : 1 : zipWith (+) (take 1 fibs) (take 2 fibs)}
			\pause
			\item Irgendwie schwierig in Hardware zu machen
		\end{itemize}
	\end{itemize}
\end{frame}

\begin{frame}{How did we get here?}
	\begin{itemize}
		\item Funktionale Programmierung:
		\begin{itemize}
			\item Programm ist Baum von Funktionsaufrufen
			\item \enquote{Teile-und-Herrsche}: Problemlösung durch aufteilen in gelöste Teilprobleme
			\item Unterste Ebene (\texttt{Prelude}) ist imperativ implementiert
		\end{itemize}
		\pause
		\item Nächsthöhere Abstraktion:
		\begin{itemize}
			\item Implementiere \enquote{Lösungs-Maschine} imperativ
			\item Problem in Prädikatenlogik darstellen, Maschine löst das
			\item Bspw. \texttt{X ist Voraussetzung für PSE, löse nach X.}
			\item $\leadsto$ logische Programmierung
		\end{itemize}
	\end{itemize}
\end{frame}

\begin{frame}{Modulhandbuch}
	\code{demos/modulhandbuch.pl}

	\begin{itemize}
		\item Legt \texttt{semester/2} und \texttt{requires/2} für \texttt{opruefung} und Abhängigkeiten an
		\item Schreibt \texttt{reqT/2}, was auch transitive Abhängigkeiten findet
	\end{itemize}
\end{frame}

\end{document}
