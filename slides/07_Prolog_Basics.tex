\documentclass{beamer}
\usetheme{metropolis}

\usepackage[ngerman]{babel}
\usepackage[autostyle=true,german=quotes]{csquotes}
\usepackage[linewidth=1pt]{mdframed}
\usepackage{hyperref}
\usepackage{makecell}
\usepackage{pifont}
\usepackage{tikz}
\usetikzlibrary{positioning, calc, arrows, fit, decorations.pathreplacing, shapes, shapes.multipart, snakes}
\usepackage{verbatim}
\usepackage{tabularx}
\usepackage{textcomp}
\usepackage{centernot}
\usepackage{amsmath}
\usepackage{xcolor}
%\usepackage{pdfpages}

\batchmode

\hypersetup{
	colorlinks,
	urlcolor=blue,
	linkcolor=black % for ToC
}
\newenvironment{qaa}[1]{
	#1

	\begin{mdframed}
		\small
}{
	\end{mdframed}
}

\newcommand{\true}{\ding{51}}
\newcommand{\false}{\ding{55}}
\newcommand{\code}[1]{
	\begin{mdframed}
		\verbatiminput{#1}
	\end{mdframed}
}

\title{Tutorium 07: Prolog-Basics}
% \subtitle{}
\author{Paul Brinkmeier}
\institute{Tutorium Programmierparadigmen am KIT}
\date{02. Dezember 2019}

\begin{document}

\begin{frame}
	\titlepage
\end{frame}

\section{Heutiges Programm}

\begin{frame}{Programm}
	\begin{itemize}
		\item ÜBs 5 und 6
		\item Typisierter $\lambda$-Kalkül
		\item Einführung in Prolog
		\item Aufgaben zu Prolog
	\end{itemize}
\end{frame}

\begin{frame}{Information}
	% Fragen zu Aufgabenrelevanz, \-Kalkül-stdlib, etc.
\end{frame}

\section{Wiederholung}

\newcommand{\aeq}{\stackrel{\alpha}{=}}
\newcommand{\naeq}{\stackrel{\alpha}{\neq}}
\newcommand{\eeq}{\stackrel{\eta}{=}}

\newcommand{\E}{\;}

\newcommand{\liin}[2]{#1\E{}#2}
\newcommand{\liiin}[3]{#1\E{}#2\E{}#3}
\newcommand{\livn}[4]{#1\E{}#2\E{}#3\E{}#4}
\newcommand{\lvn}[5]{#1\E{}#2\E{}#3\E{}#4\E{}#5}

\newcommand{\lii}[2]{(#1\E{}#2)}
\newcommand{\liii}[3]{(#1\E{}#2\E{}#3)}

\newcommand{\liir}[2]{\textcolor{red}{\underline{(}}#1\E{}#2\textcolor{red}{\underline{)}}}
\newcommand{\liiir}[3]{\textcolor{red}{\underline{(}}#1\E{}#2\E{}#3\textcolor{red}{\underline{)}}}

\newcommand{\subst}[3]{(#1)\left[#2\,\to\,#3\right]}

\section{Übungsblatt 5}

\subsection{1.5 --- $\beta$-Reduktion}

% Häufiger Fehler: 6 lässt sich wegen Klammerung nicht machen

\subsection{4 --- Church-Paare}

% Backtracken zu Fallunterscheidung im \-Kalkül

\section{Übungsblatt 6}

\subsection{3 --- $\lambda$-Terme und ihre Typen}

% Typsierten \-Kalkül wiederholen
% Baum aufmalen für E (bspw.)

\subsection{4 --- Typ-Prüfung}

% Wenn Zeit ist, Baum screenshotten und zeigen

\section{Einführung in Prolog}

% Aufbauen anhand Modulhandbuch, Brettspiel?

\section{Aufgaben zu Prolog}

% Klausuraufgaben?
% Modulhandbuchaufgaben?
% Wolf-Ziege-Kohl nur anders?
% Iwas mit Automaten?

\end{document}
