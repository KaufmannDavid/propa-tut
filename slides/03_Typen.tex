\documentclass{beamer}
\usetheme{metropolis}

\usepackage[ngerman]{babel}
\usepackage[autostyle=true,german=quotes]{csquotes}
\usepackage[linewidth=1pt]{mdframed}
\usepackage{hyperref}
\usepackage{makecell}
\usepackage{pifont}
\usepackage{tikz}
\usetikzlibrary{positioning, calc, arrows, fit, decorations.pathreplacing, shapes, shapes.multipart, snakes}
\usepackage{verbatim}
\usepackage{textcomp}
%\usepackage{pdfpages}

\batchmode

\hypersetup{
	colorlinks,
	urlcolor=blue,
	linkcolor=black % for ToC
}
\newenvironment{qaa}[1]{
	#1

	\begin{mdframed}
		\small
}{
	\end{mdframed}
}

\newcommand{\true}{\ding{51}}
\newcommand{\false}{\ding{55}}
\newcommand{\code}[1]{
	\begin{mdframed}
		\verbatiminput{#1}
	\end{mdframed}
}

\title{Tutorium 03: Typen in Haskell}
% \subtitle{}
\author{Paul Brinkmeier}
\institute{Tutorium Programmierparadigmen am KIT}
\date{05. November 2019}

\begin{document}

\begin{frame}
	\titlepage
\end{frame}

\begin{frame}{Übungsblätter}
	\begin{itemize}
		\item ÜB1 korrigiert
		\item ÜB2 korrigiert
		\item ÜB3: Bis Donnerstag Mittag
	\end{itemize}
\end{frame}

\section{Heutiges Programm}
\begin{frame}{Programm}
	\begin{itemize}
		\item Übungsblatt 2
		\item Algebraische Datentypen
		\item Implementierung eines einfach verketteten Liste
		\item Typklassen
		\item Implementierung einer Queue
	\end{itemize}
\end{frame}

\section{Übungsblatt 2}

\begin{frame}{1 --- Gültigkeitsbereiche}
	\code{demos/Binding.hs}

	\begin{itemize}
		\item Zeile 3: Welches \texttt{x} wird in \texttt{f x 2} verwendet?
		\item Zeile 3: Welches \texttt{x} wird in \texttt{y} verwendet?
		\item Zeile 4: Bindet \texttt{let} oder \texttt{where} stärker?
	\end{itemize}
\end{frame}

\begin{frame}{1 --- Gültigkeitsbereiche, Z.3}
	\code{demos/LetXInFx.hs}
	\pause
	\begin{itemize}
		\item \texttt{y} hängt in GHCi
		\item $\leadsto$ unendliche Schleife
		\item $\leadsto$ das im \texttt{let} definierte \texttt{x} wird verwendet
	\end{itemize}
\end{frame}

\begin{frame}{1 --- Gültigkeitsbereiche, Z.3}
	\code{demos/LetX.hs}
	\pause
	\begin{itemize}
		\item \texttt{y = 200} in GHCi
		\item $\leadsto$ das im \texttt{let} definierte \texttt{x} wird verwendet
	\end{itemize}
\end{frame}

\begin{frame}{1 --- Gültigkeitsbereiche, Z.4}
	\code{demos/LetVsWhere.hs}
	\pause
	\begin{itemize}
		\item \texttt{stronger = "let"} in GHCi
		\item $\leadsto$ \texttt{let} ist stärker
		\item \texttt{let ... in ...}: normaler Ausdruck
		\item \texttt{where}: nur in Definition
	\end{itemize}
\end{frame}

\begin{frame}{2.1 --- Tiefgründige Typen}
	\code{demos/DeepTypes.hs}
\end{frame}

\begin{frame}{2.2 --- Tiefgründige Typen}
	\begin{qaa}{Warum gibt keinen Wert, der \texttt{[a]} darstellt?}
		\texttt{a} ist eine Typvariable, kein spezifischer Typ.
		Bspw. sind \texttt{[1]} und \texttt{["test"]} beide gültige Werte mit dem Typ \texttt{[a]}.
		Es gibt aber keine Wert, der für alle möglichen Typen ein \texttt{a} dartstellt (anders als bspw. \texttt{null} in Java).
	\end{qaa}
\end{frame}

\begin{frame}{2.3 --- Tiefgründige Typen}
	\code{demos/NestedList.hs}

  \begin{itemize}
    \item Eine Liste von Listen von Listen... ist immer noch eine Liste
    \item $\leadsto$ Kürzester Wert: leere Liste
  \end{itemize}
\end{frame}

\begin{frame}{3 --- Listenkombinatoren}
  \code{demos/Polynom.hs}
\end{frame}

\section{Algebraische Datentypen}

\begin{frame}{Algebraische Datentypen}
	\code{demos/DataExamples.hs}

	\begin{itemize}
		\item Keyword \text{data} definiert \emph{neuen} Typ
		\item \enquote{\texttt{enum} auf Meth}
		\item Ersetzt oft Vererbung im Entwurfsprozess
	\end{itemize}
\end{frame}

\begin{frame}{Algebraische Datentypen}
	\code{demos/DataExamples2.hs}
	
	\begin{itemize}
		\item Typen werden definiert als Menge von Werten
		\item Auch: Summentypen
		\begin{itemize}
			\item Produkttypen: \texttt{struct}, Tupel, etc.
			\item Menge der möglichen Werte: Summe der möglichen Werte der Konstruktoren
		\end{itemize}
	\end{itemize}
\end{frame}

\subsection{List}

\begin{frame}{Übungsaufgabe: Liste implementieren}
	\code{demos/List.hs}

	\begin{itemize}
		\item \texttt{filter' :: (a -> Bool) -> List a -> List a}
		\item \texttt{all' :: (a -> Bool) -> List a -> Bool}
		\item \texttt{any' :: (a -> Bool) -> List a -> Bool}
		\pause
		\item \texttt{fromList :: [a] -> List a}
		\item \texttt{toList :: List a -> [a]}
		\item \texttt{head, tail, last, init}
		\pause
		\item \texttt{foldl' :: (b -> a -> b) -> b -> List a -> b}
		\item \texttt{foldr' :: (a -> b -> b) -> b -> List a -> b}
	\end{itemize}
\end{frame}

\section{Typklassen}

\begin{frame}{Typklassen}
	\code{demos/Classes.hs}

	\begin{itemize}
		\item Typklasse $\leadsto$ Einschränkung für einen Typ
		\begin{itemize}
			\item \texttt{Ord a} $\leadsto$ \texttt{a}s müssen sortierbar sein
			\item \texttt{Eq a} $\leadsto$ \texttt{a}s müssen gleich oder ungleich sein
		\end{itemize}
	\end{itemize}
\end{frame}

\begin{frame}{Definition von Typklassen}
	\code{demos/MyEq.hs}
\end{frame}

\subsection{Queue}

\begin{frame}{Übungsaufgabe: Queue implementieren}
	\code{demos/Queue.hs}

	\begin{itemize}
		\item \texttt{head :: Queue a -> a}
		\item \texttt{popHead :: Queue a -> Queue a}
		\item \texttt{last :: Queue a -> a}
		\item \texttt{popLast :: Queue a -> Queue a}
	\end{itemize}
\end{frame}

\begin{frame}{Übungsaufgabe: Typklasse Mappable}
  \code{demos/Mappable.hs}

  \begin{itemize}
    \item Implementiert diese Typklasse für \texttt{List} und \texttt{Queue}
    \item \texttt{Mappable m} $\leadsto$ Ein \texttt{m} enthält Elemente, auf die eine Funktion angewendet werden kann.
    \item Bspw.:
    \begin{itemize}
      \item \texttt{mmap (+1) Null = Null}
      \item \texttt{mmap (*2) (Cons 1 (Cons 2 (Cons 3 Null))) =\\
        (Cons 2 (Cons 4 (Cons 6 Null)))}
    \end{itemize}
  \end{itemize}
\end{frame}

\end{document}
