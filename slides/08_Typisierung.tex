\documentclass{beamer}

\usetheme{metropolis}

\usepackage[ngerman]{babel}
\usepackage[autostyle=true,german=quotes]{csquotes}
\usepackage[linewidth=1pt]{mdframed}
\usepackage{hyperref}
\usepackage{makecell}
\usepackage{pifont}
\usepackage{tikz}
\usetikzlibrary{positioning, calc, arrows, fit, decorations.pathreplacing, shapes, shapes.multipart, snakes}
\usepackage{verbatim}
\usepackage{textcomp}
\usepackage{centernot}
\usepackage{tabularx}
\usepackage{ulem}
%\usepackage{pdfpages}

\batchmode

\hypersetup{
	colorlinks,
	urlcolor=blue,
	linkcolor=black % for ToC
}
\newenvironment{qaa}[1]{
	#1

	\begin{mdframed}
		\small
}{
	\end{mdframed}
}

\newcommand{\true}{\ding{51}}
\newcommand{\false}{\ding{55}}
\newcommand{\code}[1]{
	\begin{mdframed}
		\verbatiminput{#1}
	\end{mdframed}
}


\title{Tutorium 08: Typisierung}
% \subtitle{}
\author{Paul Brinkmeier}
\institute{Tutorium Programmierparadigmen am KIT}
\date{14. Dezember 2021}

\begin{document}

\begin{frame}
    \titlepage
\end{frame}

\section{ÜB 5, 6}

\subsection{ÜB 5}

\newcommand{\E}{\;}

\newcommand{\liin}[2]{#1\E{}#2}
\newcommand{\liiin}[3]{#1\E{}#2\E{}#3}
\newcommand{\livn}[4]{#1\E{}#2\E{}#3\E{}#4}
\newcommand{\lvn}[5]{#1\E{}#2\E{}#3\E{}#4\E{}#5}

\newcommand{\lii}[2]{(#1\E{}#2)}
\newcommand{\liii}[3]{(#1\E{}#2\E{}#3)}

\newcommand{\liir}[2]{\textcolor{red}{\underline{(}}#1\E{}#2\textcolor{red}{\underline{)}}}
\newcommand{\liiir}[3]{\textcolor{red}{\underline{(}}#1\E{}#2\E{}#3\textcolor{red}{\underline{)}}}

\newcommand{\abs}[2]{\lambda{}#1.#2}

\begin{frame}{5.1 --- Klammerung im $\lambda$-Kalkül}
	\footnotesize
	\begin{eqnarray}
		\liiin{c_0}{c_1}{\liii{c_2}{c_3}{c_4}}{c_5}                   =& \pause \liin{\liir{\liir{c_0}{c_1}}{\lii{\liir{c_2}{c_3}}{c_4}}}{c_5}             \\
		\liin{\liii{c_0}{c_1}{c_2}}{\liii{c_3}{c_4}{c_5}}             =& \pause \liin{\lii{\liir{c_0}{c_1}}{c_2}}{\lii{\liir{c_3}{c_4}}{c_5}}              \\
		\livn{c_0}{c_1}{\liii{c_2}{c_3}{c_4}}{\lii{c_5}{c_6}}         =& \pause \liin{\liir{\liir{c_0}{c_1}}{\lii{\liir{c_2}{c_3}}{c_4}}}{\lii{c_5}{c_6}}  \\
		\lvn{c_0}{c_1}{\liii{c_2}{c_3}{c_4}}{c_5}{c_6}                =& \pause \liin{\liir{\liir{\liir{c_0}{c_1}}{\lii{\liir{c_2}{c_3}}{c_4}}}{c_5}}{c_6} \\
		\livn{c_0}{\lii{c_1}{\liii{c_2}{c_3}{c_4}}}{c_5}{c_6}         =& \pause \liin{\liir{\liir{c_0}{\lii{c_1}{\lii{\liir{c_2}{c_3}}{c_4}}}}{c_5}}{c_6}  \\
		\liin{(\abs{y}{\liiin{c_0}{c_1}{c_2}})}{\liii{c_3}{c_4}{c_5}} =& \pause \liin{(\abs{y}{\liin{\lii{c_0}{c_1}}{c_2}})}{\lii{\liir{c_3}{c_4}}{c_5}}  \\
                \liin{(\abs{y}{\lii{c_0}{(\abs{z}{\liin{c_1}{c_2}})}})}{\liii{c_3}{c_4}{c_5}} =& \pause \liin{(\abs{y}{\liir{c_0}{(\abs{z}{\liir{c_1}{c_2}})}})}{\lii{\liir{c_3}{c_4}}{c_5}}
	\end{eqnarray}

	\normalsize
	\begin{itemize}
		\item Funktionsanwendungen sind linksassoziativ, wie in Haskell
		\item Eigentlich: In Haskell sind Funktionsanwendungen linksassoziativ, wie im $\lambda$-Kalkül
	\end{itemize}
\end{frame}

\setcounter{equation}{0}

\begin{frame}{5.1 --- Klammerung im $\lambda$-Kalkül}
	\begin{eqnarray}
		\liin{(\abs{y}{y})}{c_0} \stackrel{?}{=} \abs{y}{\liin{y}{c_0}} \\
		\abs{y}{\lii{y}{c_0}} \stackrel{?}{=} \abs{y}{\liin{y}{c_0}}
	\end{eqnarray}

      \only<1>{
        \begin{itemize}
                \item Term 1 $\approx$ \texttt{App (Abs "{}y"{} (Var "{}y"{})) (Var "{}c0"{})}
                \item Term 2 $\approx$ \texttt{Abs "{}y"{} (App (Var "{}y"{}) (Var "{}c0"{}))}
                \item Funktionsanwendung bindet am stärkesten $\leadsto$ (2)
        \end{itemize}
      }
      \only<2>{
        \begin{itemize}
          \item Term 1 $\approx$ \texttt{app(abs(y, y), c0)}
          \item Term 2 $\approx$ \texttt{abs(y, app(y, c0))}
          \item Funktionsanwendung bindet am stärkesten $\leadsto$ (2)
        \end{itemize}
      }

\end{frame}

\setcounter{equation}{0}

\begin{frame}{1.3 --- Klammerung im $\lambda$-Kalkül}
	\begin{eqnarray}
		\subst{\liin{(x)}{c_0}}{x}{\abs{y}{y}} =& \liin{(\abs{y}{y})}{c_0} \\
		\subst{\liin{x}{c_0}}{x}{(\abs{y}{y})} =& \liin{(\abs{y}{y})}{c_0} \\
		\subst{\liin{x}{c_0}}{x}{\abs{y}{y}} =& \liin{(\abs{y}{y})}{c_0}
	\end{eqnarray}

	\begin{itemize}
		\item Alle drei Substitutionen führen zum selben Ergebnis
		\item Klammern auf der rechten Seite haben nichts mit der Substution zu tun
		\item \enquote{Für beliebiges $t$ repräsentieren $t$ und $(t)$ den gleichen $\lambda$-Term} stimmt
	\end{itemize}
\end{frame}

\setcounter{equation}{0}

\begin{frame}{1.4 --- Klammerung im $\lambda$-Kalkül}
	Angenommen, $x = \liin{c_0}{c_1}$.\\
	Welche der folgenden Aussagen gelten?

	\begin{eqnarray}
		\liiin{c_0}{c_1}{c_2}                 =& \liin{x}{c_2} \\
		\liiin{c_2}{c_0}{c_1}                 =& \liin{c_2}{x} \\
		\livn{c_2}{\lii{c_3}{c_4}}{c_0}{c_1}  =& \liiin{c_2}{\lii{c_3}{c_4}}{x} \\
		\liin{c_2}{\liii{c_0}{c_1}{c_3}}{c_4} =& \liiin{c_2}{\lii{x}{c_3}}{c_4}
	\end{eqnarray}

	\pause

	\begin{itemize}
		\item 1 und 4 gelten
		\item $\liiin{c_2}{c_0}{c_1} = \liin{\liir{c_2}{c_0}}{c_1} \neq \liin{c_2}{\liir{c_0}{c_1}} = \liin{c_2}{x}$
		\item $\livn{c_2}{\lii{c_3}{c_4}}{c_0}{c_1} \stackrel{*}{\neq} \liiin{c_2}{\lii{c_3}{c_4}}{x}$
	\end{itemize}
\end{frame}

\begin{frame}{1.4 --- Klammerung im $\lambda$-Kalkül}
    \begin{equation*}
        \app{\app{\app{\app{\uline{(\lam{a}{a})}}{(\lam{b}{b})}}{(\app{\uline{(\lam{c}{c})}}{(\appp{\uline{(\lam{d}{d})}}{(\lam{e}{e})}{(\lam{f}{f})})})}}{(\lam{g}{g})}}{(\app{\uline{(\lam{h}{h})}}{(\lam{i}{i})})}
    \end{equation*}

    \begin{itemize}
        \item Gute Übung um Redex zu finden
        \item Warum ist $\lam{g}{g}$ nicht die linke Seite eines Redex?
    \end{itemize}
\end{frame}

\begin{frame}{Wiederholung: Auswertungsstrategien}
    \begin{equation*}
      \overbrace{\app{
          \uline{\csucc}
        }{
          \overbrace{(\app{\uline{\csucc}}{\cn{0}})}^{\text{Redex 2}}
        }}^{\text{Redex 1}}
    \end{equation*}

    mit

    \begin{eqnarray*}
      \cn{0} =& \lam{s}{\lam{z}{z}} \\
      \csucc =& \lam{n}{\lam{s}{\lam{z}{\app{s}{(\app{\app{n}{s}}{z})}}}}
    \end{eqnarray*}

\begin{itemize}
    \item Welcher Redex soll zuerst ausgewertet werden?
    \item $\leadsto$ verschiedene Auswertungsstrategien
\end{itemize}
\end{frame}

\begin{frame}{Wiederholung: Normalreihenfolge}
    \begin{eqnarray*}
      &
      \app{
        \uline{\csucc}
      }{
        (\app{\uline{\csucc}}{\cn{0}})
      } \\
      \Rightarrow_\beta&
      \lam{s}{\lam{z}{\app{s}{(\app{\app{(\app{\uline{\csucc}}{\cn{0}})}{s}}{z})}}} \\
      \Rightarrow_\beta&
      \lam{s}{\lam{z}{
        \app{s}{
          (\app{\app{\uline{(\lam{s}{\lam{z}{
            \app{s}{(\app{\app{\uline{\cn{0}}}{s}}{z})}
          }})}}{s}}{z})
        }
      }} \\
      \Rightarrow_\beta^2&
      \lam{s}{\lam{z}{
        \app{s}{(\app{s}{(\app{\app{\uline{\cn{0}}}{s}}{z})})}
      }} \\
      \Rightarrow_\beta^2&
      \lam{s}{\lam{z}{
        \app{s}{(\app{s}{z})}
      }} \centernot\implies
    \end{eqnarray*}

    \vfill

    Normalreihenfolge: Linkester Redex zuerst.
\end{frame}

\begin{frame}{Wiederholung: Call-by-Name}
    \begin{eqnarray*}
      &
      \app{
        \uline{\csucc}
      }{
        (\app{\uline{\csucc}}{\cn{0}})
      } \\
      \Rightarrow_\beta&
      \lam{s}{\lam{z}{\app{s}{(\app{\app{(\app{\uline{\csucc}}{\cn{0}})}{s}}{z})}}} \centernot\implies_{\text{CbN}}
    \end{eqnarray*}

    \vfill

    Call-by-Name: Linkester Redex zuerst, aber:

    \begin{itemize}
      \item Funktions\emph{inhalte} werden nicht weiter reduziert
      \item $\leadsto$ Betrachte nur Redexe, die nicht von einem $\lambda$ umgeben sind
      \item So (ähnlich) funktioniert auch Laziness in Haskell
    \end{itemize}
\end{frame}

\begin{frame}{Wiederholung: Call-by-Value}
    \begin{eqnarray*}
      &
      \app{
        \uline{\csucc}
      }{
        (\app{\uline{\csucc}}{\cn{0}})
      } \\
      \Rightarrow_\beta&
      \app{
        \uline{\csucc}
      }{
        (\lam{s}{\lam{z}{
          \app{s}{
            (\app{\app{\uline{\cn{0}}}{s}}{z})
          }
        }})
      } \\
      \Rightarrow_\beta&
      \lam{s}{\lam{z}{
      \app{\app{
        \uline{(\lam{s}{\lam{z}{
          \app{s}{
            (\app{\app{\uline{\cn{0}}}{s}}{z})
          }
        }})}
      }{s}}{z}
      }} \centernot\implies_\text{CbV}
    \end{eqnarray*}
    \vfill

    Call-by-Value: Linkester Redex zuerst, aber:

    \begin{itemize}
      \item Funktionsinhalte werden nicht weiter reduziert
      \item $\leadsto$ Betrachte nur Redexe, die nicht von einem $\lambda$ umgeben sind
      \item Berechne \emph{Argumente vor dem Einsetzen}
      \item $\leadsto$ Betrachte nur Redexe, deren Argument \emph{unter CbV} nicht weiter reduziert werden muss
    \end{itemize}
\end{frame}

\begin{frame}{5.2 --- Redexe, Auswertungsstrategie}
    \only<1>{
        \begin{equation*}
            \app{\overbrace{\appr{
                (\lam{t}{\lam{f}{f}})
            }{
                (\appr{(
                    \lam{y}{\appr{(\lam{x}{\app{x}{x}})}{(\lam{x}{\app{x}{x}})}}
                )}{
                    (\overbrace{\appr{(\lam{x}{x})}{(\lam{x}{x})}}^{\text{CbV}})
                })
            }}^{\text{NRF, CbN}}}{
                (\lam{t}{\lam{f}{f}})
            }
        \end{equation*}
    }
    \only<2>{
        \begin{equation*}
            \lam{y}{\overbrace{\appr{(
                \lam{z}{\app{\appr{(\lam{x}{x})}{(\lam{x}{x})}}{z}}
            )}{
                y
            }}^{\text{NRF}}}
        \end{equation*}
    }

    \begin{itemize}
        \item NRF: Linkester Redex
        \item CbN: Linkester Redex, nicht in Lambda
        \item CbV: Linkester Redex, nicht in Lambda, Argumente zuerst
    \end{itemize}
\end{frame}

\begin{frame}{5.3 --- Church-Zahlen in Haskell}
    \code{../demos/ChurchNumbers.hs}

    \begin{itemize}
        \item Alte Klausuraufgabe
        \item Typisch: Lambda-Wissen hilft beim Lösen
    \end{itemize}
\end{frame}

\begin{frame}{5.4 --- Church-Paare}
    \begin{align*}
        pair &= \lam{a}{\lam{b}{\lam{p}{\appp{p}{a}{b}}}} \\
        (3, 5) &\approx \appp{pair}{c_3}{c_5} \Rightarrow^2_\beta \lam{p}{\appp{p}{c_3}{c_5}} \\
        fst &= \lam{p}{\app{p}{(\lam{a}{\lam{b}{a}})}}
    \end{align*}

    \only<1>{
        \begin{itemize}
            \item \enquote{Fertiges} Paar nimmt Funktion $p$ als Argument
            \item $p$ wählt ersten oder zweiten Eintrag
        \end{itemize}
    }
    \only<2>{
        \begin{itemize}
            \item $snd = \lam{p}{\app{p}{(\lam{a}{\lam{b}{b}})}}$
            \item $next$: Berechne zu $(n, m)$: $(m, m+1)$
            \item $next = \lam{p}{\appp{pair}{(\app{snd}{p})}{(\app{succ}{(\app{snd}{p})})}}$
            \item $pred = \lam{n}{\app{fst}{(\appp{n}{next}{(\appp{pair}{c_0}{c_0})})}}$
            \item $sub = \lam{m}{\lam{n}{\appp{n}{pred}{m}}}$
        \end{itemize}
    }
\end{frame}

\subsection{ÜB 6}

\section{Typisierung}

\subsection{Klausuraufgabe WS16/17 A3 a)}

\begin{frame}{Klausuraufgabe WS16/18 A3 a) (6P.)}
  \begin{align*}
    C_1 &= \{ \alpha_9 = \alpha_{10} \to \alpha_8, \alpha_9 = \alpha_4, \alpha_{10} = \texttt{bool} \} \\
    C_2 &= \{ \alpha_{12} = \alpha_{13} \to \alpha_{11}, \alpha_{12} = \alpha_4, \alpha_{13} = \texttt{int} \}
  \end{align*}

  \only<1>{
  \footnotesize
  \begin{enumerate}[i.]
    \item Geben Sie allgemeinste Unifikatoren $\sigma_1$ für $C_1$ und $\sigma_2$ für $C_2$ an.
    \item Ist auch $C_1 \cup C_2$ unifizierbar?
    \item Ist der Ausdruck\\
      \begin{equation*}\lam{a}{\lam{f}{\app{\app{f}{(\app{a}{\pa{true}})}}{(\app{a}{17})}}}\end{equation*}\\
        typisierbar? Begründen Sie ihre Antwort \emph{kurz}.
  \end{enumerate}
  }

  \only<2>{
    Geben Sie allgemeinste Unifikatoren $\sigma_1$ für $C_1$ und $\sigma_2$ für $C_2$ an.

    \begin{align*}
      \sigma_1 &= \texttt{unify}(\{ \alpha_9 = \alpha_{10} \to \alpha_8, \alpha_9 = \alpha_4, \alpha_{10} = \texttt{bool} \}) \\
               &= ... = \unifier{\su{\alpha_9}{\pa{bool} \to \alpha_8}, \su{\alpha_4}{\pa{bool} \to \alpha_8}, \su{\alpha_{10}}{\pa{bool}}} \\
      \sigma_2 &= \texttt{unify}(\{ \alpha_{12} = \alpha_{13} \to \alpha_{11}, \alpha_{12} = \alpha_4, \alpha_{13} = \texttt{int} \}) \\
               &= ... = \unifier{\su{\alpha_{12}}{\pa{int} \to \alpha_{11}}, \su{\alpha_4}{\pa{int} \to \alpha_{11}}, \su{\alpha_{13}}{\pa{int}}}
    \end{align*}
  }

  \only<3>{
    Ist auch $C_1 \cup C_2$ unifizierbar?
    \begin{align*}
      \sigma_1 &= ... = \unifier{\su{\alpha_9}{\pa{bool} \to \alpha_8}, \underline{\su{\alpha_4}{\pa{bool} \to \alpha_8}}, \su{\alpha_{10}}{\pa{bool}}} \\
      \sigma_2 &= ... = \unifier{\su{\alpha_{12}}{\pa{int} \to \alpha_{11}}, \underline{\su{\alpha_4}{\pa{int} \to \alpha_{11}}}, \su{\alpha_{13}}{\pa{int}}}
    \end{align*}

    A: Nein, da die \emph{allgemeinsten Unifikatoren} $\sigma_1$ und $\sigma_2$ einen Konflikt für $\alpha_4$ enthalten:
    $\texttt{unify}(\{ \pa{bool} = \pa{int} \}) = \texttt{fail}$
  }

  \only<4>{
    Ist der Ausdruck\\
      \begin{equation*}\lam{a}{\lam{f}{\app{\app{f}{(\app{a}{\pa{true}})}}{(\app{a}{\pa{17}})}}}\end{equation*}\\
    typisierbar? Begründen Sie ihre Antwort \emph{kurz}.

    \vfill

    A: Nein, da $a$ mit zwei verschiedenen Typen verwendet wird.
  }
\end{frame}

\newcommand{\tikzmark}[3]{\tikz[baseline, remember picture]{
	\node[fill=#1,draw] (#2) {#3};
}}

\begin{frame}{Cheatsheet: Typisierter Lambda-Kalkül}
  \begin{mathpar}
    \inferrule{
      \Gamma{}, p : \pi \vdash b : \rho
    }{
      \Gamma \vdash \lam{p}{b} : \pi \to \rho
    } \textrm{\textsc{Abs}}
    \and
    \inferrule{
      \Gamma \vdash f : \phi \to \alpha \\
      \Gamma \vdash x : \phi
    }{
      \Gamma \vdash \app{f}{x} : \alpha
    } \textrm{\textsc{App}}
    \and
    \inferrule{
      \Gamma{}(t) = \tau
    }{
      \Gamma \vdash t : \tau
    } \textrm{\textsc{Var}}
    \and
    \inferrule{
      c \in \textsc{Const}
    }{
      \Gamma \vdash c : \tau_c
    } \textrm{\textsc{Const}}
  \end{mathpar}

  \begin{itemize}
    \item Typvariablen: $\tau$, $\alpha$, $\pi$, $\rho$
    \item Funktionstypen: $\tau_1 \to \tau_2$, rechtsassoziativ
    \item \emph{Typisierungsregeln sind eindeutig}: Eine Regel pro Termform
  \end{itemize}
\end{frame}

\begin{frame}{Was bedeuten eigentlich $\vdash$, $\Gamma$ und $:$?}
  \begin{equation*}
    \lam{a}{\lam{f}{\app{f}{(\app{a}{\pa{true}})}}}
  \end{equation*}

  Um zu einem solchen Term ein Typisierungsproblem zu beschreiben, notieren wir:

  \begin{equation*}
    \Gamma \vdash \lam{a}{\lam{f}{\app{f}{(\app{a}{\pa{true}})}}} : \tau
  \end{equation*}

  \enquote{Im \emph{Typkontext} $\Gamma$ hat der Term den Typen $\tau$.}

  \begin{itemize}
    \item $\Gamma$: Enthält Typen für freie Variablen.
    \item $... \vdash ... : ...$ --- Notation für Typisierungsproblem.
  \end{itemize}
\end{frame}

\begin{frame}{$\Gamma$ in Aktion}
  \begin{align*}
    &\Gamma \vdash a + 42 : \pa{int} \\
    &\textsc{Const} = \{ 42 \}, \tau_{42} = \texttt{int}
  \end{align*}

  Damit die Aussage \enquote{$a + 42$ hat in $\Gamma$ den Typen \texttt{int}} stimmt, müssen wir für $\Gamma$ wählen:
  
  \begin{itemize}
    \pause
    \item $\Gamma = a : \texttt{int}, + : \texttt{int} \to \texttt{int} \to \texttt{int}$
    \pause
    \item Allgemeiner: $\Gamma = a : \alpha, + : \alpha \to \texttt{int} \to \texttt{int}$
  \end{itemize}
\end{frame}

\subsection{Typisierungsregeln im Detail}

\begin{frame}{Typisierungsregel für Lambdas}
	\begin{itemize}
		\item \tikzmark{green!20}{contextL}{\enquote{Unter Einfügung des Typs $\pi$ von $p$ in den Kontext...}}
		\item \tikzmark{red!20}{bodyTypeL}{\enquote{... ist $b$ als Funktion von $p$ typisierbar.}}
	\end{itemize}

	\begin{mathpar}
		\inferrule{
			\tikzmark{green!20}{context}{$\Gamma{}, p : \pi$} \\
			\tikzmark{red!20}{bodyType}{$\vdash b : \rho$}
		}{
			\tikzmark{blue!20}{absType}{$\Gamma \vdash \lam{p}{b} : \pi \to \rho$}
                } \textrm{\textsc{Abs}}
	\end{mathpar}

	\begin{itemize}
		\item Daraus folgt:
		\item \tikzmark{blue!20}{absTypeL}{\enquote{$\lam{p}{b}$ ist eine Funktion, die $\pi$s auf $\rho$s abbildet}}
	\end{itemize}

	\tikz[overlay, remember picture]{
		\draw[->] (bodyTypeL) edge [bend left] (bodyType);
		\draw[->] (contextL) edge [bend right] (context);
		\draw[->] (absTypeL) edge [bend left] (absType);
	}
\end{frame}

\begin{frame}{Typisierungsregel für Funktionsanwendungen}
	\begin{itemize}
		\item \tikzmark{green!20}{fTypeL}{\enquote{$f$ ist im Kontext $\Gamma$ eine Funktion, die $\phi$s auf $\alpha$s abbildet.}}
		\item \tikzmark{red!20}{aTypeL}{\enquote{$x$ ist im Kontext $\Gamma$ ein Term des Typs $\phi$.}}
	\end{itemize}
	\begin{mathpar}
		\inferrule{
			\tikzmark{green!20}{fType}{$\Gamma \vdash f : \phi \to \alpha$} \\
			\tikzmark{red!20}{aType}{$\Gamma \vdash x : \phi$}
		}{
			\tikzmark{blue!20}{eType}{$\Gamma \vdash \app{f}{x} : \alpha$}
                } \textrm{\textsc{App}}
	\end{mathpar}

	\begin{itemize}
		\item Daraus folgt:
		\item \tikzmark{blue!20}{eTypeL}{\enquote{$x$ eingesetzt in $f$ ergibt einen Term des Typs $\alpha$.}}
	\end{itemize}

	\tikz[overlay, remember picture]{
		\draw[->] (fTypeL) edge [bend right] (fType);
		\draw[->] (aTypeL) edge [bend left] (aType);
		\draw[->] (eTypeL) edge [bend left] (eType);
	}
\end{frame}

\begin{frame}{Einfache Typisierungsregel für Variablen}
	\begin{itemize}
		\item \tikz[baseline, remember picture]{\node [fill=green!20,draw] (varRetL) {\enquote{Der Typkontext $\Gamma$ enthält einen Typ $\tau$ für $t$.}};}
	\end{itemize}

	\begin{mathpar}
		\inferrule{
			\tikz[baseline, remember picture]{\node[fill=green!20,draw] (varRet) {$\Gamma{}(t) = \tau$};}
		}{
			\tikz[baseline, remember picture]{\node[fill=blue!20,draw] (varShow) {
				$\Gamma \vdash t : \tau$
			};}
		} \textrm{\textsc{Var}}
	\end{mathpar}

	\begin{itemize}
		\item Daraus folgt:
		\item \tikz[baseline, remember picture]{\node [fill=blue!20,draw] (varShowL) {\enquote{Variable $t$ hat im Kontext $\Gamma$ den Typ $\tau$.}};}
	\end{itemize}

	\tikz[overlay, remember picture]{
		\draw[->] (varShowL) edge [bend left] (varShow);
		\draw[->] (varRetL) edge [bend right] (varRet);
	}
\end{frame}

% https://tex.stackexchange.com/questions/9466/color-underline-a-formula
\def\mathunderline#1#2{\color{#1}\underline{{\color{black}#2}}\color{black}}

\begin{frame}{Typisierung: Beispiel}
  \only<1>{
    \begin{mathpar}
      x : \texttt{bool} \vdash \lam{f}{\app{f}{x}} : (\texttt{bool} \to \alpha) \to \alpha
    \end{mathpar}

    \enquote{Unter der Annahme, dass $x$ den Typ \texttt{bool} hat, hat $\lam{f}{\app{f}{x}}$ den Typ $(\texttt{bool} \to \alpha) \to \alpha$.}
  }

  \only<2>{
    \begin{mathpar}
      \inferrule{
        \mathunderline{red}{x : \texttt{bool}}, \mathunderline{blue}{f} : \mathunderline{yellow}{\texttt{bool} \to \alpha} \vdash
        \mathunderline{green}{\app{f}{x}} :
        \mathunderline{orange}{\alpha}
      }{
        \mathunderline{red}{x : \texttt{bool}} \vdash \lam{\mathunderline{blue}{f}}{\mathunderline{green}{\app{f}{x}}} : (\mathunderline{yellow}{\texttt{bool} \to \alpha}) \to \mathunderline{orange}{\alpha}
      } \textsc{Abs}
    \end{mathpar}

    Pattern-Matching: Der äußerste Term ist ein Lambda, also wenden wir die $\textsc{Abs}$-Regel an.

    \begin{columns}
        \begin{column}{0.5\textwidth}
    \begin{align*}
      \mathunderline{red}{\Gamma} &= \mathunderline{red}{x : \texttt{bool}} \\
      \mathunderline{blue}{p} &= \mathunderline{blue}{f}, \mathunderline{green}{b} = \mathunderline{green}{\app{f}{x}} \\
      \mathunderline{yellow}{\pi} &= \mathunderline{yellow}{\texttt{bool} \to \alpha} \\
      \mathunderline{orange}{\rho} &= \mathunderline{orange}{\alpha}
    \end{align*}
\end{column}
\begin{column}{0.5\textwidth}

    \begin{mathpar}
      \inferrule{
        \mathunderline{red}{\Gamma{}}, \mathunderline{blue}{p} : \mathunderline{yellow}{\pi} \vdash \mathunderline{green}{b} : \mathunderline{orange}{\rho}
      }{
        \mathunderline{red}{\Gamma} \vdash \lam{\mathunderline{blue}{p}}{\mathunderline{green}{b}} : \mathunderline{yellow}{\pi} \to \mathunderline{orange}{\rho}
      } \textrm{\textsc{Abs}}
    \end{mathpar}
\end{column}
\end{columns}
  }

  \only<3>{
    \begin{mathpar}
      \inferrule{
        \inferrule{
          \inferrule{\Gamma(f) = \texttt{bool} \to \alpha}{\Gamma \vdash f : \texttt{bool} \to \alpha} \textsc{Var} \\
          \inferrule{\Gamma(x) = \texttt{bool}           }{\Gamma \vdash x : \texttt{bool}           } \textsc{Var}
        }{
          x : \pa{bool}, f : \texttt{bool} \to \alpha \vdash \app{f}{x} : \alpha
        } \textsc{App}
      }{
        x : \texttt{bool} \vdash \lam{f}{\app{f}{x}} : (\texttt{bool} \to \alpha) \to \alpha
      } \textsc{Abs}
    \end{mathpar}

    \begin{equation*}
      \Gamma = x : \pa{bool}, f : \texttt{bool} \to \alpha
    \end{equation*}

    \begin{columns}
      \begin{column}{0.5\textwidth}
        \begin{figure}
          \begin{tikzpicture}
            \node {$\lambda f$}
              [level distance=10mm]
              child {node {@}
                child {node {f}}
                child {node {x}}
              }
            ;
          \end{tikzpicture}
        \end{figure}
      \end{column}
      \begin{column}{0.35\textwidth}
        \begin{equation*}
          \lam{f}{\app{f}{x}}
        \end{equation*}
      \end{column}
    \end{columns}
  }
\end{frame}

\begin{frame}{Typinferenz}
    Problemstellung bei Typinferenz: Zu einem gegebenen Term den passenden Typ finden.
  
    \begin{itemize}
      \item Struktur des Terms erkennen. Wo sind:
      \begin{itemize}
        \item Lambdas?
        \item Funktionsanwendungen?
        \item Variablen/Konstanten?
      \end{itemize}
      \item Entsprechenden Baum aufstellen.
      \item Typgleichungen finden.
      \item Gleichungssystem unifizieren.
    \end{itemize}
\end{frame}

\subsection{Von Typisierungsregeln zu Typinferenz}

\begin{frame}{Von Typisierungsregeln zu Typinferenz}
  Beim inferieren wird das Pattern-matching der Typen durch die \emph{Unifikation} übernommen.
  Deswegen schreiben wir anstelle von konkreten Typen immer $\alpha_i$ und merken uns die Gleichungen für später:

  \only<1>{
    \begin{equation*}
      \inferrule{
        \Gamma{}, p : \pi \vdash b : \rho
      }{
        \Gamma \vdash \lam{p}{b} : \pi \to \rho
      } \textrm{\textsc{Abs}}
      \;\;
      \leadsto
      \;\;
      {\inferrule{
        \Gamma{}, p : \alpha_j \vdash b : \alpha_k
      }{
        \Gamma \vdash \lam{p}{b} : \alpha_i
      } \textrm{\textsc{Abs}} \atop
      \{ \alpha_i = \alpha_j \to \alpha_k \}}
    \end{equation*}
  }

  \only<2>{
    \begin{equation*}
      \inferrule{
        \Gamma \vdash f : \phi \to \alpha \\
        \Gamma \vdash x : \phi
      }{
        \Gamma \vdash \app{f}{x} : \alpha
      } \textrm{\textsc{App}}
      \;\;
      \leadsto
      \;\;
      {\inferrule{
        \Gamma \vdash f : \alpha_j \\
        \Gamma \vdash x : \alpha_k
      }{
        \Gamma \vdash \app{f}{x} : \alpha_i
      } \textrm{\textsc{App}} \atop
      \{ \alpha_j = \alpha_k \to \alpha_i \}}
    \end{equation*}
  }

  \only<3>{
    \begin{equation*}
      \inferrule{
        \Gamma{}(t) = \tau
      }{
        \Gamma \vdash t : \tau
      } \textrm{\textsc{Var}}
      \;\;
      \leadsto
      \;\;
      {\inferrule{
        \Gamma{}(t) = \alpha_j
      }{
        \Gamma \vdash t : \alpha_i
      } \textrm{\textsc{Var}} \atop
      \{ \alpha_i = \alpha_j \}}
    \end{equation*}
  }
\end{frame}

\subsection{Hindley-Milner-Algorithmus}

\begin{frame}{Algorithmus zur Typinferenz}
	\begin{itemize}
		\item Stelle Typherleitungsbaum auf
		\begin{itemize}
			\item In jedem Schritt werden neue Typvariablen $\alpha_i$ angelegt
			\item Statt die Typen direkt im Baum einzutragen, werden Gleichungen in einem Constraint-System eingetragen
		\end{itemize}
		\item Unifiziere Constraint-System zu einem Unifikator
		\begin{itemize}
			\item Robinson-Algorithmus, im Grunde wie bei Prolog
                        \item I.d.R.: Allgemeinster Unifikator (mgu)
		\end{itemize}
	\end{itemize}

	\begin{columns}
		\scriptsize
		\begin{column}{0.3\textwidth}
                  \begin{mathpar}
    \inferrule{
      \Gamma{}(t) = \alpha_j
    }{
      \Gamma \vdash t : \alpha_i
    } \textrm{\textsc{Var}}
                  \end{mathpar}

                  \center
                        Constraint:\\$\{ \alpha_i = \alpha_j \}$
		\end{column}
		\begin{column}{0.3\textwidth}
                  \begin{mathpar}
    \inferrule{
      \Gamma \vdash f : \alpha_j \\
      \Gamma \vdash x : \alpha_k
    }{
      \Gamma \vdash \app{f}{x} : \alpha_i
    } \textrm{\textsc{App}}
                  \end{mathpar}
\center
			Constraint:\\$\{ \alpha_j = \alpha_k \to \alpha_i \}$
		\end{column}
		\begin{column}{0.3\textwidth}
                  \begin{mathpar}
    \inferrule{
      \Gamma{}, p : \alpha_j \vdash b : \alpha_k
    }{
      \Gamma \vdash \lam{p}{b} : \alpha_i
    } \textrm{\textsc{Abs}}
                  \end{mathpar}
                        \center
			Constraint:\\$\{ \alpha_i = \alpha_j \to \alpha_k \}$
		\end{column}
	\end{columns}
\end{frame}

\begin{frame}{Herleitungsbaum: Beispiel}
  \only<1>{
    \begin{mathpar}
      \vdash \lam{x}{\lam{y}{x}} : \alpha_1
    \end{mathpar}

    Beispielhafte Aufgabenstellung: Finde den Typen $\alpha_1$.
  }

  \only<2>{
    \begin{mathpar}
      \inferrule{
        \mathunderline{red}{x} : \alpha_2 \vdash \mathunderline{blue}{\lam{y}{x}} : \alpha_3
      }{
        \vdash \lam{\mathunderline{red}{x}}{\mathunderline{blue}{\lam{y}{x}}} : \alpha_1
      } \textsc{Abs}
    \end{mathpar}

    Typgleichungen:

    \begin{equation*}
      C = \{ \underline{\alpha_1 = \alpha_2 \to \alpha_3} \}
    \end{equation*}
  }

  \only<3>{
    \begin{mathpar}
      \inferrule{
        \inferrule{
          \mathunderline{red}{x : \alpha_2}, \mathunderline{blue}{y} : \alpha_4 \vdash \mathunderline{orange}{x} : \alpha_5
        }{
          \mathunderline{red}{x : \alpha_2} \vdash \lam{\mathunderline{blue}{y}}{\mathunderline{orange}{x}} : \alpha_3
        } \textsc{Abs}
      }{
        \vdash \lam{x}{\lam{y}{x}} : \alpha_1
      } \textsc{Abs}
    \end{mathpar}

    Typgleichungen:

    \begin{align*}
      C = \{ & \alpha_1 = \alpha_2 \to \alpha_3 \\
           , & \underline{\alpha_3 = \alpha_4 \to \alpha_5} \}
    \end{align*}
  }

  \only<4>{
    \begin{mathpar}
      \inferrule{
        \inferrule{
          \inferrule{
            (\mathunderline{red}{x : \alpha_2, y : \alpha_4})(\mathunderline{blue}{x}) = \alpha_2
          }{
            \mathunderline{red}{x : \alpha_2, y : \alpha_4} \vdash \mathunderline{blue}{x} : \alpha_5
          } \textsc{Var}
        }{
          x : \alpha_2 \vdash \lam{y}{x} : \alpha_3
        } \textsc{Abs}
      }{
        \vdash \lam{x}{\lam{y}{x}} : \alpha_1
      } \textsc{Abs}
    \end{mathpar}

    Typgleichungen:
    
    \begin{align*}
      C = \{ & \alpha_1 = \alpha_2 \to \alpha_3 \\
           , & \alpha_3 = \alpha_4 \to \alpha_5 \\
           , & \underline{\alpha_5 = \alpha_2} \}
    \end{align*}
  }
\end{frame}

\begin{frame}{Herleitungsbaum: Aufgabe}
  \begin{mathpar}
    \inferrule{
      ...
    }{
      \vdash \lam{f}{\app{f}{(\lam{x}{x})}} : \alpha_1
    } \textsc{Abs}
  \end{mathpar}

  Findet den Typen $\alpha_1$. Teilpunkte gibt es für:

  \begin{itemize}
    \item Herleitungsbaum,
    \item Typgleichungsmenge $C$,
    \item Unifikation per Robinsonalgorithmus.
  \end{itemize}
\end{frame}

\begin{frame}{Herleitungsbaum: Aufgabe}
\begin{mathpar}
\inferrule{
\inferrule{
\inferrule{
(f : \alpha_2)(f) = \alpha_2
}{
f : \alpha_2 \vdash f : \alpha_4
} \textsc{Var}
\\
\inferrule{
\inferrule{
\Gamma(x) = \alpha_6
}{
\Gamma \vdash x : \alpha_7
} \textsc{Var}
}{
f : \alpha_2 \vdash \lam{x}{x} : \alpha_5
} \textsc{Abs}
}{
f : \alpha_2 \vdash \app{f}{(\lam{x}{x})} : \alpha_3
} \textsc{App}
}{
\vdash \lam{f}{\app{f}{(\lam{x}{x})}} : \alpha_1
} \textsc{Abs}\\
\Gamma = f : \alpha_2, x : \alpha_6
\end{mathpar}

\begin{align*}
  C = \{ & \alpha_1 = \alpha_2 \to \alpha_3, \alpha_4 = \alpha_5 \to \alpha_3, \\
           & \alpha_2 = \alpha_4, \\
           & \alpha_5 = \alpha_6 \to \alpha_7, \alpha_6 = \alpha_7 \}
\end{align*}
\end{frame}

\section{\textsc{Let}-Polymorphismus}

\begin{frame}{\textsc{Let}-Polymorphismus: Motivation}
  \begin{equation*}
    \lam{f}{\app{f}{f}}
  \end{equation*}

  \begin{itemize}
    \item Diese Funktion verwendet $f$ auf zwei Arten:
    \begin{itemize}
      \item $\alpha \to \alpha$: Rechte Seite.
      \item $(\alpha \to \alpha) \to (\alpha \to \alpha)$: Linke Seite, nimmt $f$ als Argument und gibt es zurück.
    \end{itemize}
    \pause
    \item Problem: $\alpha \to \alpha$ und $(\alpha \to \alpha) \to (\alpha \to \alpha)$ sind nicht unifizierbar!
    \begin{itemize}
      \item \enquote{occurs check}: $\alpha$ darf sich nicht selbst einsetzen.
    \end{itemize}
  \item Idee: Bei jeder Verwendung eines polymorphen Typen erzeugen wir \emph{neue Typvariablen}, um diese Beschränkung zu umgehen.
  \end{itemize}
\end{frame}

\begin{frame}{Typschemata und Instanziierung}
  \only<1>{
    \begin{itemize}
      \item Idee: Bei jeder Verwendung eines polymorphen Typen erzeugen wir \emph{neue Typvariablen}, um diese Beschränkung zu umgehen.
      \item Ein \emph{Typschema} ist ein Typ, in dem manche Typvariablen allquantifiziert sind:
    \end{itemize}

    \begin{align*}
      \phi     & = \forall \alpha_1 . \; ... \; \forall \alpha_n . \tau \\
      \alpha_i & \in FV(\tau)
    \end{align*}

    \begin{itemize}
      \item \emph{Typschemata kommen bei uns immer nur in Kontexten vor!}
      \item Beispiele:
      \begin{itemize}
        \item $\forall \alpha . \alpha \to \alpha$
        \item $\forall \alpha . \alpha \to \beta \to \alpha$
      \end{itemize}
    \end{itemize}

  }

  \only<2>{
    \begin{itemize}
      \item Ein Typschema spannt eine Menge von Typen auf, mit denen es \emph{instanziiert} werden kann:
    \end{itemize}

    \begin{align*}
      \forall \alpha . \alpha \to \alpha & \succeq \text{int} \to \text{int} \\
      \forall \alpha . \alpha \to \alpha & \succeq \tau \to \tau \\
      \forall \alpha . \alpha \to \alpha & \not\succeq \tau \to \sigma \\
      \forall \alpha . \alpha \to \alpha & \not\succeq \tau \to \tau \to \tau \\
      \forall \alpha . \alpha \to \alpha & \succeq (\tau \to \tau) \to (\tau \to \tau)
    \end{align*}
  }
\end{frame}

\begin{frame}{\textsc{Let}-Polymorphismus}
  \only<1>{
    Um Typschemata bei der Inferenz zu verwenden, müssen wir zunächst die Regel für Variablen anpassen:

    \begin{mathpar}
      \inferrule{
        \Gamma(x) = \phi \\
        \phi \succeq_{\text{frische $\alpha_i$}} \tau
      }{
        \Gamma \vdash x : \alpha_j
      } \textsc{Var} \\
      \text{Constraint:} \; \{ \alpha_j = \tau \}
    \end{mathpar}

    \begin{itemize}
      \item $\succeq_\text{frische $\alpha_i$}$ instanziiert ein Typschema mit $\alpha_i$, die noch nicht im Baum vorkommen.
      \item Jetzt brauchen wir noch eine Möglichkeit, Typschemata zu erzeugen.
    \end{itemize}
  }

  \only<2>{
    Mit einen \textsc{Let}-Term wird ein Typschema eingeführt:

    \begin{mathpar}
      \inferrule{
        \Gamma \vdash t_1 : \alpha_i \\
        \Gamma' \vdash t_2 : \alpha_j
      }{ 
        \Gamma \vdash \texttt{let} \;\; x = t_1 \;\; \texttt{in} \;\; t_2 : \alpha_k
      } \textsc{Let}
    \end{mathpar}

    \pause

    \begin{align*}
      \sigma_{let} &= mgu(C_{let}) \\
           \Gamma' &= \sigma_{let}(\Gamma), x : ta(\sigma_{let}(\alpha_i), \sigma_{let}(\Gamma)) \\
          C'_{let} &= \{ \alpha_n = \sigma_{let}(\alpha_n) \mid \sigma_{let}(\alpha_n) \;\; \text{ist definiert} \}
    \end{align*}

    Constraints: $C'_{let} \cup C_{body} \cup \{ a_j = a_k \}$
  }
\end{frame}

\begin{frame}{Beispiel: \textsc{Let}-Polymorphismus}
    \scriptsize
    \begin{mathpar}
      \inferrule{
        \inferrule{
          ...
        }{
          \vdash \lam{x}{x} : \alpha_2
        } \textsc{Abs} \\
        \inferrule{
          \inferrule{
            \Gamma'(f) = \forall \alpha_5 . \alpha_5 \to \alpha_5 \\\\
            \succeq \alpha_8 \to \alpha_8
          }{
            \Gamma' \vdash f : \alpha_6
          } \textsc{Var} \\
          \inferrule{
            \Gamma'(f) = \forall \alpha_5 . \alpha_5 \to \alpha_5 \\\\
            \succeq \alpha_9 \to \alpha_9
          }{
            \Gamma' \vdash f : \alpha_7
          } \textsc{Var}
        }{
          \Gamma' \vdash \app{f}{f} : \alpha_3
        } \textsc{App}
      }{ 
        \vdash \texttt{let} \;\; f = \lam{x}{x} \;\; \texttt{in} \;\; \app{f}{f} : \alpha_1
      } \textsc{Let}
    \end{mathpar}

    \begin{align*}
           C_{let} &= \{ \alpha_2 = \alpha_4 \to \alpha_5, \alpha_4 \to \alpha_5 \} \\
      \sigma_{let} &= \unifier{\su{\alpha_2}{\alpha_5 \to \alpha_5}, \su{\alpha_4}{\alpha_5}} \\
      \Gamma'      &= x : \forall \alpha_5 . \alpha_5 \to \alpha_5 \\
          C'_{let} &= \{ \alpha_2 = \alpha_5 \to \alpha_5, \alpha_4 = \alpha_5 \} \\
          C_{body} &= \{ \alpha_6 = \alpha_7 \to \alpha_3, \alpha_6 = \alpha_8 \to \alpha_8, \alpha_7 = \alpha_9 \to \alpha_9 \} \\
                 C &= C'_{let} \cup C_{body} \cup \{ \alpha_3 = \alpha_1 \}
    \end{align*}
\end{frame}

\end{document}