\documentclass{beamer}

\usetheme{metropolis}

\usepackage[ngerman]{babel}
\usepackage[autostyle=true,german=quotes]{csquotes}
\usepackage[linewidth=1pt]{mdframed}
\usepackage{hyperref}
\usepackage{makecell}
\usepackage{pifont}
\usepackage{tikz}
\usetikzlibrary{positioning, calc, arrows, fit, decorations.pathreplacing, shapes, shapes.multipart, snakes}
\usepackage{verbatim}
\usepackage{textcomp}
\usepackage{centernot}
\usepackage{tabularx}
\usepackage{ulem}
%\usepackage{pdfpages}

\batchmode

\hypersetup{
	colorlinks,
	urlcolor=blue,
	linkcolor=black % for ToC
}
\newenvironment{qaa}[1]{
	#1

	\begin{mdframed}
		\small
}{
	\end{mdframed}
}

\newcommand{\true}{\ding{51}}
\newcommand{\false}{\ding{55}}
\newcommand{\code}[1]{
	\begin{mdframed}
		\verbatiminput{#1}
	\end{mdframed}
}


\title{Tutorium 08: Typisierung, Prolog}
% \subtitle{}
\author{Paul Brinkmeier}
\institute{Tutorium Programmierparadigmen am KIT}
\date{21. Dezember 2021}


\begin{document}

\begin{frame}
    \titlepage
\end{frame}

\section{Rückblick}

\begin{frame}{Rückblick}
    Bisherige Themen:

    \begin{itemize}
        \item Haskell: Funktionale Programmierung
        \item Lambda-Kalkül: Beta-Reduktion, Church-Encodings
        \item Typisierung: Lambda-Typen, Let, Typinferenz
        \item Prolog: Logische Programmierung
    \end{itemize}

    Nach den Ferien:

    \begin{itemize}
        \item Parallelprogrammierung: MPI, Java
        \item Design by Contract: OpenJML
        \item Compiler: Parser + Java ByteCode
    \end{itemize}
\end{frame}

\section{ÜB 7}

\section{Prolog}

\subsection{Kleine Aufgaben}

\begin{frame}{Maximum-Funktion}
    Schreibt ein Prädikat \texttt{max/3}, das das Maximum bestimmt:

    \code{code/max.out}

    \begin{itemize}
        \item Schreibt die Funktion zuerst in Haskell
        \begin{itemize}
            \item Verwendet die Guard-Notation
            \item Wie viele Vergleiche benötigt man?
        \end{itemize}
        \item Lässt sich die Haskell-Version 1:1 übersetzen?
    \end{itemize}
\end{frame}

\begin{frame}{Einteilen in Gruppen}
    Schreibt ein Prädikat \texttt{classify/2}, das Zahlen in folgende Gruppen einteilt:

    \begin{itemize}
        \item \texttt{negativ}: Zahlen kleiner als 0
        \item \texttt{interessant}: Zahlen von 0 bis 42 (inklusiv)
        \item \texttt{zugross}: Zahlen über 42
    \end{itemize}

    \code{code/classify.out}
\end{frame}

\subsection{Cut}

\begin{frame}{Cut!}
    \only<1>{
        \code{../demos/classify1.pro}

        \begin{itemize}
            \item Wenn die Zahl \texttt{negativ} ist, können \texttt{X >= 0} und \texttt{X > 42} nicht stimmen
            \item $\leadsto$ Prolog sollte diese eigentlich überspringen
            \item Prolog macht aber keine Arithmetik für uns
            \item Aushilfe: Der Cut\texttt{!}
        \end{itemize}
    }

    \only<2>{
        \code{../demos/classify2.pro}

        \begin{itemize}
            \item Vermeidung redundanter Berechnungen: Cut, geschrieben \texttt{!}
            \item Cut kann immer einmal erfüllt werden
            \item Reerfüllung lässt aber die ganze Regel fehlschlagen
            \item Verschiedene Erklärungen:
            \begin{itemize}
                \item Cut \enquote{schneidet Reerfüllungsbaum ab}
                \item Cut entfernt Choice Points
                \item Cut \enquote{verpflichtet} Prolog zur Erfüllung einer bestimmten Regel
            \end{itemize}
        \end{itemize}
    }

    \only<3>{
        \code{../demos/classify2.pro}

        \begin{itemize}
            \item Schließen sich die Tests einzelner Fälle gegenseitig aus, kann man nach ihnen gefahrlos Cuts einfügen
            \item \enquote{Grüner} Cut: Ändert nicht Verhalten, nur Laufzeit des Programms
            \item Hier:
            \begin{itemize}
                \item \texttt{X < 0} $\leadsto$ \texttt{X >= 0} und \texttt{X > 42} unmöglich
                \item \texttt{X =< 42} $\leadsto$ \texttt{X > 42} unmöglich
                \item \texttt{!} $\Leftrightarrow$ \enquote{Ab hier kann kein anderer Fall mehr eintreten, fertig}
            \end{itemize}
        \end{itemize}
    }

    \only<4>{
        \code{../demos/classify3.pro}

        \begin{itemize}
            \item Einige Tests lassen sich \emph{durch Cuts ersetzen}
            \item \enquote{Roter} Cut: Ändert Verhalten \emph{und} Laufzeit des Programms
            \item Fehleranfällig! Wenn möglich vermeiden.
            \item Programme ohne rote Cuts sind
            \begin{itemize}
                \item einfacher zu verstehen.
                \item etwa genauso performant.
            \end{itemize}
            \item Faustregel: Reihenfolge der Regeln sollte austauschbar sein
        \end{itemize}
    }
\end{frame}

\begin{frame}{Summe Vorwärts und Rückwärts}
    Schreibt ein Prädikat \texttt{sum/3}, sodass \texttt{sum(A, B, C)} die Gleichung

    \begin{equation*}
        \texttt{A} + \texttt{B} = \texttt{C}
    \end{equation*}

    löst, wenn \emph{zwei oder mehr Variablen belegt sind}.

    \begin{itemize}
        \item 3 Fälle.
        \item Verwendet \texttt{number}, um zu Prüfen ob eine Variable mit einer Zahl belegt ist.
    \end{itemize}
\end{frame}

\subsection{Klausuraufgabe}

\begin{frame}{Klausur SS15: Haus vom Nikolaus}
\end{frame}

\section{Ein Haskell-Projekt}

\end{document}