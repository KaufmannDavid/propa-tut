\documentclass{beamer}
\usetheme{metropolis}

\usepackage[ngerman]{babel}
\usepackage[autostyle=true,german=quotes]{csquotes}
\usepackage[linewidth=1pt]{mdframed}
\usepackage{hyperref}
\usepackage{makecell}
\usepackage{pifont}
\usepackage{tikz}
\usetikzlibrary{positioning, calc, arrows, fit, decorations.pathreplacing, shapes, shapes.multipart, snakes}
\usepackage{verbatim}
\usepackage{tabularx}
\usepackage{textcomp}
\usepackage{centernot}
\usepackage{amsmath}
\usepackage{xcolor}
\usepackage{tikz}
\usepackage{underscore}

\batchmode

\hypersetup{
	colorlinks,
	urlcolor=blue,
	linkcolor=black % for ToC
}
\newenvironment{qaa}[1]{
	#1

	\begin{mdframed}
		\small
}{
	\end{mdframed}
}

\newcommand{\true}{\ding{51}}
\newcommand{\false}{\ding{55}}
\newcommand{\code}[1]{
	\begin{mdframed}
		\verbatiminput{#1}
	\end{mdframed}
}

\title{Tutorium 13: Design by Contract}
% \subtitle{}
\author{Paul Brinkmeier}
\institute{Tutorium Programmierparadigmen am KIT}
\date{28. Januar 2020}

\begin{document}

\begin{frame}
	\titlepage
\end{frame}

\begin{frame}{Design by Contract}
	ProPa-Stoff zu Design by Contract:

	\begin{itemize}
		\item Grundlagen: Pre-/Postconditions, Caller, Callee
		\begin{itemize}
			\item A.K.A.: Vor-/Nachbedingungen, Aufrufer, Aufgerufener
		\end{itemize}
		\item JML (Java Modeling Language):
		\begin{itemize}
			\item \texttt{@ requires}
			\item \texttt{@ ensures} (mit \texttt{\string\old} und \texttt{\string\result})
			\item \texttt{@ invariant}
			\item \texttt{/*@ pure @*/}, \texttt{/*@ nullable @*/}, \texttt{/*@ spec_public @*/}
			\item Quantoren: \texttt{\string\forall}, \texttt{\string\exists}
			\item Liskovsches Substitutionsprinzip
			\pause
			\item Java-\texttt{assert}-Keyword
		\end{itemize}
	\end{itemize}
\end{frame}

\section{JML-Klausuraufgabe}

\begin{frame}{JML-Klausuraufgabe}
    Klausur 19SS, Aufgabe 6d (3P.)

    {
    \footnotesize
    \code{code/19ss-a6d.java}

    (d) Der Vertrag der Methode \texttt{combine} wird \emph{vom Aufgerufenen} verletzt.
    Begründen Sie dies und geben Sie an, wie die verletzte Nachbedingung angepasst werden könnte.
    }
\end{frame}

\begin{frame}{JML-Klausuraufgabe}
    Klausur 19SS, Aufgabe 6e (2P.)

    {
    \footnotesize
    \code{code/19ss-a6e.java}

    (d) Wird der Vertrag hier \emph{vom Aufrufer} erfüllt?
    Begründen Sie kurz.
    }
\end{frame}

\section{JML}

\begin{frame}{\texttt{@ requires}}
	\code{code/jml/requires.java}

	\begin{itemize}
		\item \texttt{@ requires} definiert eine Vorbedingung für eine Methode.
		\item Vorbedingungen müssen vom Aufrufer erfüllt werden.
	\end{itemize}
\end{frame}

\begin{frame}{\texttt{@ ensures}}
	\code{code/jml/ensures.java}

	\begin{itemize}
		\item \texttt{@ ensures} definiert eine Nachbedingung für eine Methode.
		\item Nachbedingungen müssen vom Aufgerufenen erfüllt werden.
        \item Mit \texttt{\string\old} und \texttt{\string\result} werden Beziehungen zwischen Ursprungszustand, Rückgabewert und neuem Zustand eingeführt.
	\end{itemize}
\end{frame}

\begin{frame}{\texttt{@ invariant}}
	\code{code/jml/invariant.java}

	\begin{itemize}
		\item \texttt{@ invariant} definiert Invarianten für eine Klasse.
		\item Diese können bspw. wiederverwendet werden, um Vorbedingungen für Methoden zu erfüllen.
	\end{itemize}
\end{frame}

\begin{frame}{\texttt{/*@ pure @*/}}
	\code{code/jml/pure.java}

	\begin{itemize}
		\item Verträge sind implizit \texttt{public}.\\
		$\leadsto$ \texttt{private}-Attribute nicht verwendbar
		\item Um Getter-Funktionen in Verträgen nutzen zu können, müssen diese frei von Seiteneffekten und mit \texttt{/*@ pure *@/} markiert sein.
	\end{itemize}
\end{frame}

\begin{frame}{\texttt{/*@ spec_public @*/}}
	\code{code/jml/specpublic.java}

	\begin{itemize}
		\item Alternative: \texttt{private}-Attribute als \texttt{/*@ spec_public @*/} markieren.
		\item Immer noch \texttt{private}, können vom Checker aber trotzdem gesehen werden.
	\end{itemize}
\end{frame}

\begin{frame}{Quantoren, logische Operatoren}
	\code{code/jml/quantors.java}

	\begin{itemize}
		\item Für das Arbeiten mit Aussagen in Verträgen gibt es ein paar Helferchen:
		\begin{itemize}
			\item \texttt{\string\forall <decl>; <cond>; <expr>}
			\item \texttt{\string\exists <decl>; <cond>; <expr>}
			\item \texttt{<cond> ==> <expr>}
		\end{itemize}
	\end{itemize}
\end{frame}

\section{Übungsaufgabe: RPN-Taschenrechner}

\begin{frame}{Reverse Polish Notation}
	\begin{equation*}
		(2 + 4) * (10 - 3)
	\end{equation*}

	\begin{itemize}
		\item \enquote{Natürliche} Darstellung: Infix-Notation
		\begin{itemize}
			\item \enquote{Problem}: Man braucht Klammern!
		\end{itemize}
		\pause
		\item Alternative: Polnische (Präfix-)Notation
		\begin{itemize}
			\item \texttt{* + 2 4 - 10 3}
			\item \texttt{multipy (add 2 4) (subtract 10 3)} ($\leadsto$ LISP)
		\end{itemize}
		\pause
		\item Einfach zu implementieren: Umgekehrte polnische Notation
		\begin{itemize}
			\item \texttt{10 3 - 2 4 + *}
			\item Links nach rechts durchgehen
			\item Zahlen werden auf einen Stack gelegt
			\item Operatoren nehmen Operanden vom Stack, legen Ergebnis auf den Stack
		\end{itemize}
		\pause
		\item Bspw. alte Taschenrechner, Forth, \href{https://de.wikipedia.org/wiki/Shunting-yard-Algorithmus}{Shunting-yard-Algorithmus}
	\end{itemize}
\end{frame}

\begin{frame}{RPN-Taschenrechner}
	\begin{itemize}
		\item \texttt{demos/java/rpncalculator/RpnCalculator.java}
		\item Überprüft manuell den Vertrag von \texttt{pop()}.
		\pause
		\item Überlegt euch einen entsprechenden Vertrag für \texttt{push(BigInteger)}.
		\pause
		\item Identifiziert seiteneffektfreie Methoden und markiert sie mit \texttt{/*@ pure @*/}.
		\pause
		\item Überlegt euch Invarianten für \texttt{elementCount} und \texttt{stack.length}.
		\pause
		\item Überprüft eure Verträge mit OpenJML:\\
			{\footnotesize \texttt{java -jar oj/openjml.jar -exec <solver> -esc <.java> -method push,pop}}
		\item Überpüft dann zusäztlich die Methode \texttt{execute(List<Token>)}.
	\end{itemize}
\end{frame}

\section{Ende}

\begin{frame}{Ende}
	\begin{itemize}
		\item Im Campus-System kann man sich bis zum 17.03. für die ProPa-Klausur anmelden
		\item \href{https://campus.studium.kit.edu/renewal/payment.php}{Rückmelden} bis zum 15.02.
	\end{itemize}
\end{frame}

\end{document}
