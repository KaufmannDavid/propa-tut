\documentclass{beamer}
\usetheme{metropolis}

\usepackage[ngerman]{babel}
\usepackage[autostyle=true,german=quotes]{csquotes}
\usepackage[linewidth=1pt]{mdframed}
\usepackage{hyperref}
\usepackage{makecell}
\usepackage{pifont}
\usepackage{tikz}
\usetikzlibrary{positioning, calc, arrows, fit, decorations.pathreplacing, shapes, shapes.multipart, snakes}
\usepackage{verbatim}
\usepackage{textcomp}
\usepackage{centernot}
\usepackage{tabularx}
\usepackage{ulem}
%\usepackage{pdfpages}

\batchmode

\hypersetup{
	colorlinks,
	urlcolor=blue,
	linkcolor=black % for ToC
}
\newenvironment{qaa}[1]{
	#1

	\begin{mdframed}
		\small
}{
	\end{mdframed}
}

\newcommand{\true}{\ding{51}}
\newcommand{\false}{\ding{55}}
\newcommand{\code}[1]{
	\begin{mdframed}
		\verbatiminput{#1}
	\end{mdframed}
}


\title{Tutorium 13: Actoren \& Compiler}
% \subtitle{}
\author{David Kaufmann}
\institute{Tutorium Programmierparadigmen am KIT}
\date{7. Februar 2023}

\begin{document}

\begin{frame}
	\titlepage
\end{frame}

\section{Actor Model}
\begin{frame}{Actor Model}
    \begin{itemize}
        \item Actors sind computation units mit State, Behaviour, Mailbox
        \item kommunizieren über Nachrichten
        \item verarbeiten immer nur eine Nachricht
        \item verarbeiten Nachrichten in der Reihenfolge in der sie empfangen wurden
    \end{itemize}
\end{frame}

\begin{frame}{HelloWorldActor}
    \footnotesize
    \code{./code/actors/HelloWorldActor.java}
\end{frame}

\begin{frame}{Further Methods}
    \begin{itemize}
        \item \texttt{preStart(), postStop(), preRestart(), postRestart()}
        \item \texttt{getSelf()}: Referenz auf sich selbst
        \item \texttt{getContext()}: Context um weiter Actoren zu erzeugen
        \item \texttt{getSender()}: Sender der aktuell verarbeiteten Nachricht
    \end{itemize}
\end{frame}

\begin{frame}{Actor Creation}
    Muss auf einen \textbf{Context} aufgerufen werden, entweder \texttt{ActorSystem} oder \texttt{getContext()} von innerhalb eines Actors.
    
    Actor beaufsichtigt alle Aktoren die er erstellt hat
    \footnotesize
    \code{./code/actors/CreateActor.java} 
\end{frame}

\begin{frame}{Messages}
    Muss auf den Empfänger aufgerufen werden
    \begin{itemize}
        \item \texttt{tell(Object message, ActorRef sender)}: asyncron, nicht blockierend
        \item \texttt{Future<?> Patterns.ask(ActorRef target, Object msg, Timeout timeout)}: kann awaited werden, sollte man aber vermeiden
    \end{itemize}
\end{frame}

\begin{frame}{Running Actors}
    \begin{itemize}
        \item \texttt{ActorSystem.create(String name)}: Erzeugt ein ActorSystem
        \item \texttt{void stop(ActorRef actorToStop)}: Muss auf eine ActorRefFactory aufgerufen werden (ActorSystem, Context)
        \item \texttt{PoisonPill.getInstance()}: Kann als Nachricht an einen Actor gesendet werden
        \item \texttt{ActorSystem.terminater()}: Terminiert ActorSystem
    \end{itemize}
\end{frame}

\section{Klausuraufgabe SS21}

\section{Einführung in Compilerbau}

\begin{frame}{Compiler in ProPa}
	\begin{itemize}
		\item Ein bisschen...
		\begin{itemize}
                        \item Lexikalische Analyse
			\item Syntaktische Analyse (Parsen)
			\item \textcolor{gray}{Semantische Analyse, Optimierung}
			\item Codegenerierung
		\end{itemize}
		\pause
		\item Klausur:
		\begin{itemize}
			\item SLL(k)-Form beweisen
			\item Rekursiven Abstiegsparser schreiben/vervollständigen
			\item First/Follow-Mengen berechnen
			\item Java-Bytecode
				% Zeigen: Java-BC-Aufgabe (21SS), Code-Generierung nächste Mittwoch und Freitag (findet noch statt)
		\end{itemize}
	\end{itemize}
\end{frame}

\section{Compiler}

\subsection{Motivation}

\begin{frame}{Compiler: Motivation}
	\begin{itemize}
		\item Maschine(-nmodell) versteht i.d.R. eingeschränkten Instruktionssatz
		\item $\leadsto$ Programme in Maschinensprache sind schwer les-/schreibbar
		\pause
		\item Also: Erfinde einfacher zu Schreibende ($\approx$ mächtigere) Sprache, die dann in die Sprache der Maschine übersetzt wird.
		\item Diesen Übersetzungsschritt sollte optimalerweise ein Programm erledigen, da wir sonst auch einfach direkt Maschinensprache-Programme schreiben können.
	\end{itemize}
\end{frame}

\begin{frame}{Compiler}
	\begin{itemize}
		\item Übersetzer für formale Sprachen nennt man \emph{Compiler}
		\item Beispiele:
		\begin{itemize}
			\item C, Haskell, Rust, Go $\to$ X86
			\item Java, Scala, Kotlin $\to$ Java-Bytecode
			\item TypeScript $\to$ JavaScript/WebAssembly
		\end{itemize}
		\pause
		\item Single-pass vs. Multi-pass
		\begin{itemize}
			\item Single-pass: Eingabe wird einmal gelesen, Ausgabe währenddessen erzeugt (ältere Compiler)
			\item Multi-pass: Eingabe wird in Zwischenschritten in verschiedene Repräsentationen umgewandelt
			\begin{itemize}
				\item Quellsprache, Tokens, AST, Zwischensprache, Zielsprache
			\end{itemize}
		\end{itemize}
	\end{itemize}
\end{frame}

\subsection{Lexikalische Analyse}

\begin{frame}{Lexikalische Analyse}
	\begin{columns}
		\begin{column}{0.5\textwidth}
			\code{code/lexinput.java}
		\end{column}
		\begin{column}{0.5\textwidth}
			\code{code/lexoutput.java}
		\end{column}
	\end{columns}

	\begin{itemize}
		\item Lexikalische Analyse (Tokenisierung) verarbeitet eine Zeichensequenz in eine Liste von \emph{Tokens}.
		\item Tokens sind Zeichengruppen, denen eine Semantik innewohnt:
		\begin{itemize}
			\item \texttt{int} --- Typ einer Ganzzahl
			\item \texttt{=} --- Zuweisungsoperator
			\item \texttt{x1} --- Variablen- oder Methodenname
			\item \texttt{123} --- Literal einer Ganzzahl
			\item \texttt{"123"} --- String-Literal
			\item etc.
		\end{itemize}
		\item Lösbar mit regulären Ausdrücken, Automaten
	\end{itemize}
\end{frame}

\subsection{Syntaktische Analyse}

\begin{frame}{Syntaktische Analyse}
	\begin{itemize}
                \item Syntaktische Analyse stellt die unterliegende (Baum-)Struktur der bisher linear gelesenen Eingabe fest:
		\begin{itemize}
			\item Blockstruktur von Programmen
			\item Baumstruktur von HTML-Dateien
			\item Header + Inhalt-Struktur von Mails
			\item Verschachtelte arithmetische Ausdrücke
		\end{itemize}
		\item Syntaktische Analyse ist das größte Compiler-Thema in PP.
		\pause
		\item Übliche Vorgehensweise (in PP):
		\begin{itemize}
			\item Grammatik $G$ erfinden
			\item Ggf. $G$ in andere Form $G'$ bringen
			\item Rekursiven Abstiegsparser für $G'$ implementieren
		\end{itemize}
		\item Alternativ: Parser-Kombinatoren, Yacc, etc.
	\end{itemize}
\end{frame}

\begin{frame}{Beispiel: Arithmetische Ausdrücke}
  \begin{center}
    \textcolor<1>{red}{\texttt{a+a+b}} \hfill \textcolor<2>{red}{\texttt{a*c+b}} \hfill \textcolor<3>{red}{\texttt{a*c+b*d}}
  \end{center}

    \begin{itemize}
      \item Zu beachten: Punkt-vor-Strich (Präzedenz), Klammerung, etc.
      \item Nicht mehr mit regulären Ausdrücken lösbar
      \item Beispielgrammatik:
    \end{itemize}

    \begin{columns}
      \begin{column}{0.5\textwidth}
        \begin{align*}
          E \to & \; E \;\; \texttt{+} \;\; E \\
           \mid & \; E \;\; \texttt{*} \;\; E \\
           \mid & \; \texttt{(} \;\; E \;\; \texttt{)} \\
           \mid & \; \texttt{id}
        \end{align*}
      \end{column}
      \begin{column}{0.5\textwidth}
        \only<1>{
          \begin{tikzpicture}[level distance=10mm, sibling distance=10mm]
            \node {E}
              child {
                node {E}
                  child {
                    node {E}
                      child {node {\texttt{id[a]}}}
                  }
                  child {node {\texttt{+}}}
                  child {
                    node {E}
                      child {node {\texttt{id[a]}}}
                  }
              }
              child {
                node {\texttt{+}}
              }
              child {
                node {E}
                  child {
                    node {\texttt{id[b]}}
                  }
              }
            ;
          \end{tikzpicture}
        }
        \only<2>{
          \begin{tikzpicture}[level distance=10mm, sibling distance=10mm]
            \node {E}
              child {
                node {E}
                  child {
                    node {E}
                      child {node {\texttt{id[a]}}}
                  }
                  child {node {\texttt{*}}}
                  child {
                    node {E}
                      child {node {\texttt{id[c]}}}
                  }
              }
              child {
                node {\texttt{+}}
              }
              child {
                node {E}
                  child {
                    node {\texttt{id[b]}}
                  }
              }
            ;
          \end{tikzpicture}
        }
        \only<3>{
          \begin{tikzpicture}[level distance=7.5mm, sibling distance=10mm]
            \node {E}
              child {
                node {E}
                  child {
                    node {E}
                      child {node {\texttt{id[a]}}}
                  }
                  child {node {\texttt{*}}}
                  child {
                    node {E}
                      child {node {E}
                        child {node {\texttt{id[c]}}}
                      }
                      child {node {\texttt{+}}}
                      child {node {E}
                        child {node {\texttt{id[b]}}}
                      }
                  }
              }
              child {
                node {\texttt{*}}
              }
              child {
                node {E}
                  child {
                    node {\texttt{id[d]}}
                  }
              }
            ;
          \end{tikzpicture}
        }
      \end{column}
    \end{columns}
\end{frame}

\begin{frame}{Beispiel: Mathematische Ausdrücke}
	\begin{equation*}
          E \to \; E \;\; \texttt{+} \;\; E \;
           \mid \; E \;\; \texttt{*} \;\; E \;
           \mid \; \texttt{(} \;\; E \;\; \texttt{)} \;
           \mid \; \texttt{id}
	\end{equation*}

        \begin{center}
          \texttt{a*c+b}
        \end{center}

	\pause

        \begin{columns}
          \begin{column}{0.5\textwidth}
            \begin{tikzpicture}[level distance=7.5mm, sibling distance=10mm]
              \node {E}
                child {
                  node {E}
                    child {
                      node {E}
                        child {node {\texttt{id[a]}}}
                    }
                    child {node {\texttt{*}}}
                    child {
                      node {E}
                        child {node {\texttt{id[c]}}}
                    }
                }
                child {
                  node {\texttt{+}}
                }
                child {
                  node {E}
                    child {
                      node {\texttt{id[b]}}
                    }
                }
              ;
            \end{tikzpicture}
          \end{column}
          \begin{column}{0.5\textwidth}
            \begin{tikzpicture}[level distance=7.5mm, sibling distance=10mm]
              \node {E}
                child {
                  node {E}
                    child {
                      node {\texttt{id[a]}}
                    }
                }
                child {
                  node {\texttt{*}}
                }
                child {
                  node {E}
                    child {
                      node {E}
                        child {node {\texttt{id[c]}}}
                    }
                    child {node {\texttt{+}}}
                    child {
                      node {E}
                        child {node {\texttt{id[b]}}}
                    }
                }
              ;
            \end{tikzpicture}
          \end{column}
        \end{columns}

	\begin{itemize}
          \item Grammatik nicht eindeutig $\leadsto$ schlecht
          \item Grammatik garantiert nicht Punkt-vor-Strich $\leadsto$ schlecht
          \item Grammatik ist linksrekursiv $\leadsto$ nicht einfach zu parsen $\leadsto$ schlecht
	\end{itemize}
\end{frame}

\begin{frame}{Grammar Engineering 1 --- Präzedenz}
  \begin{equation*}
    E \to \; E \;\; \texttt{+} \;\; E \;
     \mid \; E \;\; \texttt{*} \;\; E \;
     \mid \; \texttt{(} \;\; E \;\; \texttt{)} \;
     \mid \; \texttt{id}
  \end{equation*}

  \begin{itemize}
    \item Punkt-vor-Strich (\enquote{Operatorpräzedenz}) wird von dieser naiven Grammatik nicht beachtet.
    \item Lösung: Ein Nichtterminal pro Präzedenzstufe:
    \begin{itemize}
      \item \emph{Summen} von \emph{Produkten} von \emph{Atomen}.
      \item Herkömmliche Begriffe: Ausdruck, Term und Faktor.
    \end{itemize}
  \end{itemize}

  \begin{alignat*}{2}
    & Expr   & \; \to \; & Expr \; \texttt{+} \; Term \\
          && \mid \;\; & Term \\
    & Term   & \; \to \; & Term \; \texttt{+} \; Factor \\
            && \mid \;\; & Factor \\
    & Factor & \; \to \; & \texttt{(} \; Expr \; \texttt{)} \; \mid \; \texttt{id}
  \end{alignat*}
\end{frame}

\begin{frame}{Welche Art von Grammatik wollen wir denn genau?}
  \begin{columns}
    \begin{column}{0.5\textwidth}
      \begin{tikzpicture}[
        box/.style={
          draw=black,
          rounded corners
        },
        every fit/.style={
          inner sep=2mm
        },
        node distance=3mm
      ]
        \node[box] (rg) {RG};
        \node[below=of rg] (sll) {SLL(1)};
        \node[box,fit=(rg) (sll),draw=blue] (b1) {};

        \node[right=of b1] (lr) {LR(1)};
        \node[box,fit=(b1) (lr)] (b3) {};

        \node[below=of b3] (dcfg) {DCFG};
        \node[box,fit=(b3) (dcfg)] (b4) {};

        \node[below=of b4] (cfg) {CFG};
        \node[box,fit=(b4) (cfg)] {};
      \end{tikzpicture}
    \end{column}
    \begin{column}{0.5\textwidth}
      \footnotesize

      \begin{itemize}
        \item CFG-Parsen ist i.A. in $O(n^3)$, bspw. Earley-Algorithmus.
        \item Reguläre Grammatiken ($\approx$ reg. Sprachen) sind uns nicht mächtig genug.
        \item \textcolor{red}{L}\textcolor{blue}{R}: \textcolor{red}{Left-to-right}, \textcolor{blue}{Rightmost}
        \item \textcolor{red}{L}\textcolor{blue}{L}: \textcolor{red}{Left-to-right}, \textcolor{blue}{Leftmost}
        \item SLL-Parsing $\in O(n)$
      \end{itemize}
    \end{column}
  \end{columns}

  \vspace{1cm}
  \center
  CFG: Context-Free Grammar/Kontextfreie Grammatik
\end{frame}

\begin{frame}{Grammar Engineering 2 --- Linksrekursion eliminieren}
  \begin{columns}
    \begin{column}{0.5\textwidth}
      \begin{alignat*}{2}
        & Expr   & \; \to \; & Expr \; \texttt{+} \; Term \\
              && \mid \;\; & Term \\
        & Term   & \; \to \; & Term \; \texttt{+} \; Factor \\
                && \mid \;\; & Factor \\
        & Factor & \; \to \; & \texttt{(} \; Expr \; \texttt{)} \; \mid \; \texttt{id}
      \end{alignat*}

    \end{column}
    \begin{column}{0.5\textwidth}
      \begin{figure}
        \begin{tikzpicture}[
          level distance=7.5mm,
          sibling distance=15mm
        ]
          \node {Expr}
            child {
              node {Expr}
                child {
                  node {Expr}
                    child {
                      node {...}
                    }
                    child {node {Term}}
                }
                child {node {Term}}
            }
            child {node {Term}}
          ;
        \end{tikzpicture}
      \end{figure}
    \end{column}
  \end{columns}

  \vfill

  Problem: Die Linksableitung des Symbols $Expr$ in dieser Grammatik ist eine endlose Schleife.
\end{frame}

\begin{frame}{Grammar Engineering 2 --- Linksrekursion eliminieren}
  \begin{columns}
    \begin{column}{0.5\textwidth}
      \begin{alignat*}{2}
        & Expr   & \; \to \; & Expr \; \texttt{+} \; Term \\
              && \mid \;\; & Term \\
        & Term   & \; \to \; & Term \; \texttt{+} \; Factor \\
                && \mid \;\; & Factor \\
        & Factor & \; \to \; & \texttt{(} \; Expr \; \texttt{)} \; \mid \; \texttt{id}
      \end{alignat*}

    \end{column}
    \begin{column}{0.5\textwidth}
      \begin{alignat*}{2}
        & Expr   & \; \to \; & \; Term \; Expr' \\
        & Expr'  & \; \to \; & \; \texttt{+} \; Term \; Expr' \\
                && \mid \;\; & \; \epsilon \\
        & Term   & \; \to \; & \; Term \; Expr' \\
        & Term'  & \; \to \; & \; \texttt{*} \; Factor \; Term' \\
                && \mid \;\; & \; \epsilon \\
        & Factor & \; \to \; & \texttt{(} \; Expr \; \texttt{)} \; \mid \; \texttt{id}
      \end{alignat*}
    \end{column}
  \end{columns}

  \vfill

  \only<1>{
    Lösung: Linksrekursion eliminieren, durch folgendes Umschreiben der Grammatik:
  }

  \only<2>{
  \begin{columns}
    \begin{column}{0.5\textwidth}
      \begin{alignat*}{2}
        & Sym   & \; \to \; & Sym \; \alpha \\
               && \mid \;\; & \beta
      \end{alignat*}

    \end{column}
    \begin{column}{0.5\textwidth}
      \begin{alignat*}{2}
        & Sym   & \; \to \; & \beta \; Sym' \\
        & Sym'  & \; \to \; & \alpha \; Sym' \\
               && \mid \;\; & \epsilon
      \end{alignat*}
    \end{column}
  \end{columns}
  }
\end{frame}

\begin{frame}{Verbesserte Grammatik}
  \begin{alignat*}{2}
    & E      & \; \to \; & \; T \; EList \\
    & EList  & \; \to \; & \; \texttt{+} \; T \; EList \\
            && \mid \;\; & \; \epsilon \\
    & T      & \; \to \; & \; T \; EList \\
    & TList  & \; \to \; & \; \texttt{*} \; F \; TList \\
            && \mid \;\; & \; \epsilon \\
    & F      & \; \to \; & \texttt{(} \; E \; \texttt{)} \; \mid \; \texttt{id}
  \end{alignat*}

  \begin{itemize}
    \item Grammatik ist eindeutig \true
    \item Grammatik erzeugt nur korrekte Terme \true
    \item Grammatik enthält keine Linksrekursion \true
  \end{itemize}
\end{frame}

\begin{frame}{First-/Followmenge, Indizmenge}
	\footnotesize

	\begin{align*}
		EList \to & \;\; \epsilon \mid \textrm{\texttt{+}} \;\; T \;\; EList \mid \textrm{\texttt{-}} \;\; T \;\; EList
	\end{align*}
	
	Wie können wir bspw. bei $EList$ entscheiden, welche Produktion anzuwenden ist?
	\pause
	\begin{itemize}
		\item $\leadsto$ definiere \emph{Indizmenge} $IM_k(A \to \alpha) = \textrm{First}_k(\alpha \textrm{Follow}_k(A))$
		\item Wenn nächste $k$ Token in $IM_k(EList \to \phi)$ $\leadsto$ weiter mit $\phi$
		\pause
		\item $IM_1(EList \to \; \epsilon) = \textrm{First}_1(\epsilon \textrm{Follow}_1(EList)) = \{ \textrm{\texttt{)}}, \textrm{\texttt{\#}} \}$
		\item $IM_1(EList \to \; \textrm{\texttt{+}} \;\; T \;\; EList) = \textrm{First}_1(\textrm{\texttt{+}} \; T \; EList \; \textrm{Follow}_1(EList)) = \{ \textrm{\texttt{+}} \}$
		\item $IM_1(EList \to \; \textrm{\texttt{-}} \;\; T \;\; EList) = \textrm{First}_1(\textrm{\texttt{-}} \; T \; EList \; \textrm{Follow}_1(EList)) = \{ \textrm{\texttt{-}} \}$
		\pause
		\item $\textrm{First}_k(A)$: Menge an möglichen ersten $k$ Token in $A$
		\item $\textrm{Follow}_k(A)$: Menge an möglichen ersten $k$ Token nach $A$
	\end{itemize}
\end{frame}

\begin{frame}{SLL-Kriterium}
	Grammatik ist in $\textrm{SLL}(k)$-Form\\
	$:\Leftrightarrow \forall A \to \alpha, A \to \beta \in P: IM_k(A \to \alpha) \cap IM_k(A \to \beta) = \emptyset$

	\begin{itemize}
		\item $\textrm{SLL}(k)$: Bei jedem Nichtterminal muss die zu wählende Produktion an den nächsten $k$ Token wählbar sein.
		\item Nichtterminale mit nur einer Produktion sind hier irrelevant
		\item Schwierig daran: $\textrm{Follow}$-Mengen berechnen
	\end{itemize}
	\pause
	\begin{align*}
		E \to & \;\; E \;\; \textrm{\texttt{+}} \;\; T \mid E \;\; \textrm{\texttt{-}} \;\; T \mid T\\
		T \to & \;\; T \;\; \textrm{\texttt{*}} \;\; F \mid T \;\; \textrm{\texttt{/}} \;\; F \mid F\\
		F \to & \;\; \textrm{\texttt{num}} \; | \; \textrm{\texttt{(}} \;\; E \;\; \textrm{\texttt{)}}
	\end{align*}
	
	\begin{itemize}
		\item Begründet formal, dass obige Grammatik nicht $\textrm{SLL}(1)$.
		\item Berechnet $\textrm{Follow}_1(N)$ für $N \in \{ E, T, F \}$.
	\end{itemize}
\end{frame}

\begin{frame}{Rekursive Abstiegsparser}
	\footnotesize
	\begin{align*}
		E     \to & \;\; T \;\; EList\\
		EList \to & \;\; \epsilon \mid \textrm{\texttt{+}} \;\; T \;\; EList \mid \textrm{\texttt{-}} \;\; T \;\; EList\\
		T     \to & \;\; F \;\; TList\\
		TList \to & \;\; \epsilon \mid \textrm{\texttt{*}} \;\; F \;\; TList \mid \textrm{\texttt{/}} \;\; F \;\; TList\\
		F \to & \;\; \textrm{\texttt{num}} \; | \; \textrm{\texttt{(}} \;\; E \;\; \textrm{\texttt{)}}
	\end{align*}
	\begin{itemize}
            \pause
            \item Was bringt uns das diese Grammatik in SLL(1)-Form ist?
            \pause
		\item $G$ ist jetzt einfach ausprogrammierbar:
		\begin{itemize}
			\item 1 Methode per Nichtterminal: \texttt{parseE()}, \texttt{parseEList()}, ...
			\item \texttt{lexer.lex()} konsumiert das aktuelle Token
			\item \texttt{lexer.current} gibt nicht konsumierenden Zugriff auf das aktuelle Token
		\end{itemize}
	\end{itemize}
\end{frame}

\end{document}
