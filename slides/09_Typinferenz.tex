\documentclass{beamer}
\usetheme{metropolis}

\usepackage[ngerman]{babel}
\usepackage[autostyle=true,german=quotes]{csquotes}
\usepackage[linewidth=1pt]{mdframed}
\usepackage{hyperref}
\usepackage{makecell}
\usepackage{pifont}
\usepackage{tikz}
\usetikzlibrary{positioning, calc, arrows, fit, decorations.pathreplacing, shapes, shapes.multipart, snakes}
\usepackage{verbatim}
\usepackage{tabularx}
\usepackage{textcomp}
\usepackage{centernot}
\usepackage{amsmath}
\usepackage{xcolor}
\usepackage{tikz}
%\usepackage{pdfpages}

\batchmode

\hypersetup{
	colorlinks,
	urlcolor=blue,
	linkcolor=black % for ToC
}
\newenvironment{qaa}[1]{
	#1

	\begin{mdframed}
		\small
}{
	\end{mdframed}
}

\newcommand{\true}{\ding{51}}
\newcommand{\false}{\ding{55}}
\newcommand{\code}[1]{
	\begin{mdframed}
		\verbatiminput{#1}
	\end{mdframed}
}

\title{Tutorium 09: Typinferenz}
% \subtitle{}
\author{Paul Brinkmeier}
\institute{Tutorium Programmierparadigmen am KIT}
\date{16. Dezember 2019}

\begin{document}

\begin{frame}
	\titlepage
\end{frame}

\section{Heutiges Programm}

\begin{frame}{Programm}
	\begin{itemize}
		\item Keine ÜBs :(
		\item Unifikation in Prolog
		\item Typinferenz mit Typisierungsbäumen
		\item Allgemeinster Unifikator
	\end{itemize}
\end{frame}

\section{Datentypen in Prolog}

\begin{frame}{Funktoren}
	\code{demos/funktoren.pl}

	\begin{itemize}
		\item Funktor $\approx$ Name + Liste von Prolog-Ausdrücken
		\item Liste leer $\leadsto$ \enquote{Atom}
		\item Name wird immer klein geschrieben
		\begin{itemize}
			\item Großbuchstaben: bspw. \texttt{'List'}
		\end{itemize}
	\end{itemize}
\end{frame}

\begin{frame}{Variablen}
	\code{demos/variablen.pl}

	\begin{itemize}
		\item Variablen werden immer groß geschrieben
		\item \texttt{=} ist nicht Zuweisung, sondern Unifikation
		\item Unifikation $\approx$ (formales) Pattern-Matching
	\end{itemize}
\end{frame}

\begin{frame}{Unikation zweier Prolog-Terme nach Robinson}
	\code{code/robinson.pseudo}
\end{frame}

\begin{frame}{Unifikation zweier Funktoren nach Robinson}
	\code{code/robinson2.pseudo}

	\begin{itemize}
		\item Argumente des linken und rechten Funktors werden nacheinander unifiziert
		\item Dabei müssen die vorherigen Substitutionen beachtet werden
		\pause
		\item Umformung des Robinson-Algorithmus der Vorlesung, nur zur Veranschaulichung!
	\end{itemize}
\end{frame}

\begin{frame}{Prolog-Unifikation}
	Unifiziert:

	\begin{itemize}
		\item \texttt{A = x}
		\item \texttt{B = f(x)}
		\item \texttt{C = g(C)}
		\item \texttt{f(x, A, z) = f(x, y, B)}
		\item \texttt{g(x, A, z) = f(x, A, A)}
		\item \texttt{f(g(z)) = f(D)}
	\end{itemize}

	Ergebnis: Entweder \textbf{fail} oder ein Unifikator.
\end{frame}

\section{Typinferenz}

\newcommand{\aeq}{\stackrel{\alpha}{=}}
\newcommand{\naeq}{\stackrel{\alpha}{\neq}}
\newcommand{\eeq}{\stackrel{\eta}{=}}

\newcommand{\E}{\;}

\newcommand{\liin}[2]{#1\E{}#2}
\newcommand{\liiin}[3]{#1\E{}#2\E{}#3}
\newcommand{\livn}[4]{#1\E{}#2\E{}#3\E{}#4}
\newcommand{\lvn}[5]{#1\E{}#2\E{}#3\E{}#4\E{}#5}

\newcommand{\lii}[2]{(#1\E{}#2)}
\newcommand{\liii}[3]{(#1\E{}#2\E{}#3)}

\newcommand{\liir}[2]{\textcolor{red}{\underline{(}}#1\E{}#2\textcolor{red}{\underline{)}}}
\newcommand{\liiir}[3]{\textcolor{red}{\underline{(}}#1\E{}#2\E{}#3\textcolor{red}{\underline{)}}}

\newcommand{\subst}[3]{(#1)\left[#2\,\to\,#3\right]}
\newcommand{\abs}[2]{\lambda{}#1.#2}
\newcommand{\reducesTo}[1]{\stackrel{#1}{\implies}}
\newcommand{\tikzmark}[3]{\tikz[baseline, remember picture]{
	\node[fill=#1,draw] (#2) {#3};
}}

\newcommand{\typeRule}[3]{\frac{#2}{#3} \textrm{\textsc{#1}}}

\begin{frame}{$\lambda$-Terme}
	Ein Term im $\lambda$-Kalkül hat eine der drei folgenden Formen:

	\vspace{0.5cm}

	\begin{tabularx}{\textwidth}{ X | X | X }
		\textbf{Notation} & \textbf{Besteht aus}                      & \textbf{Bezeichnung} \\
		\hline
		$x$               & $x$ : Variablenname                       & Variable             \\
		\hline
		$\lambda{}p.b$    &
			\begin{tabular}[t]{@{}c@{}}$p$ : Variablenname\\$b$ : $\lambda$-Term\end{tabular}
									      & Abstraktion          \\
		\hline
		$f$ $a$           & $f$, $a$ : $\lambda$-Terme                & Funktionsanwendung   \\
	\end{tabularx}
\end{frame}

\begin{frame}{Wiederholung}
	\begin{itemize}
		\item Bisher: Typisierung \emph{prüfen}
		\item Gegeben Term $t$, Typ $\tau$ und Kontext $\Gamma$, zeige, dass $\Gamma \vdash t : \tau$
		\begin{itemize}
			\item \enquote{$t$ hat Typ $\tau$ im Kontext $\Gamma$}
		\end{itemize}
	\end{itemize}

	\begin{equation*}
		\typeRule{Abs}{
			...
		}{
			\texttt{f} : \textrm{int} \to \beta \vdash \abs{\texttt{x}}{\liin{\texttt{f}}{\texttt{x}}} : \textrm{int} \to \beta
		}
	\end{equation*}
	
	\begin{itemize}
		\item \enquote{Zeige, dass $\abs{\texttt{x}}{\liin{\texttt{f}}{\texttt{x}}}$ den Typ $\textrm{int} \to \beta$ hat}
		\pause
		\item Jetzt drehen wir den Spieß um:
	\end{itemize}

	\begin{equation*}
		\typeRule{Abs}{
			...
		}{
			\texttt{f} : \textrm{int} \to \beta \vdash \abs{\texttt{x}}{\liin{\texttt{f}}{\texttt{x}}} : \alpha_1
		}
	\end{equation*}

	\begin{itemize}
		\item \enquote{Finde den allgemeinsten Typen $\alpha_1$ von $\abs{\texttt{x}}{\liin{\texttt{f}}{\texttt{x}}}$}
	\end{itemize}
\end{frame}

\begin{frame}{\textsc{Var}}
	\begin{itemize}
		\item \tikz[baseline, remember picture]{\node [fill=green!20,draw] (varRetL) {\enquote{Der Typkontext $\Gamma$ enthält einen Typ $\sigma$ für $t$}};}
		\item \tikz[baseline, remember picture]{\node [fill=red!20,draw] (varEatL) {\enquote{$\sigma$ kann mit $\tau$ instanziiert werden}};}
	\end{itemize}

	\begin{equation*}
		\frac{
			\tikz[baseline, remember picture]{\node[fill=green!20,draw] (varRet) {$\Gamma{}(t) = \sigma$};}
			\;\;\;\;
			\tikz[baseline, remember picture]{\node[fill=red!20,draw] (varEat) {$\sigma \succeq \tau$};}
		}{
			\tikz[baseline, remember picture]{\node[fill=blue!20,draw] (varShow) {
				$\Gamma \vdash t : \tau$
			};}
		} \textrm{\textsc{Var}}
	\end{equation*}

	\begin{itemize}
		\item dann gilt:
		\item \tikz[baseline, remember picture]{\node [fill=blue!20,draw] (varShowL) {\enquote{Variable $t$ hat im Kontext $\Gamma$ den Typ $\tau$}};}
		\vspace{0.5em}
		\item $\sigma \succeq \tau$ $\leadsto$ \enquote{$\sigma$ hat $\tau$s Struktur und ist (mind.) allgemeiner}
		\begin{itemize}
			\item $\textrm{int} \to \textrm{int} \succeq \textrm{int} \to \textrm{int}$
			\item $\forall \alpha . \alpha \to \alpha \succeq \textrm{int} \to \textrm{int}$
			\item $\alpha \to \alpha \not\succeq \textrm{int} \to \textrm{int}$
			\item $\textrm{int} \to \textrm{int} \not\succeq \forall \alpha . \alpha \to \alpha$
		\end{itemize}
	\end{itemize}

	\tikz[overlay, remember picture]{
		\draw[->] (varShowL) edge [bend left] (varShow);
		\draw[->] (varRetL) edge [bend right] (varRet);
		\draw[->] (varEatL) edge [bend left] (varEat);
	}
\end{frame}

\begin{frame}{\textsc{App}}
	\begin{itemize}
		\item \tikzmark{green!20}{fTypeL}{\enquote{$f$ ist im Kontext $\Gamma$ eine Funktion, die $\phi$s auf $\alpha$s abbildet}}
		\item \tikzmark{red!20}{aTypeL}{\enquote{$a$ ist im Kontext $\Gamma$ ein Term des Typs $\phi$}}
	\end{itemize}
	\begin{equation*}
		\typeRule{App}{
			\tikzmark{green!20}{fType}{$\Gamma \vdash f : \phi \to \alpha$}
			\;\;\;\;\;
			\tikzmark{red!20}{aType}{$\Gamma \vdash a : \phi$}
		}{
			\tikzmark{blue!20}{eType}{$\Gamma \vdash \liin{f}{a} : \alpha$}
		}
	\end{equation*}

	\begin{itemize}
		\item dann gilt:
		\item \tikzmark{blue!20}{eTypeL}{\enquote{$a$ eingesetzt in $f$ ergibt einen Term des Typs $\alpha$}}
	\end{itemize}

	\tikz[overlay, remember picture]{
		\draw[->] (fTypeL) edge [bend right] (fType);
		\draw[->] (aTypeL) edge [bend left] (aType);
		\draw[->] (eTypeL) edge [bend left] (eType);
	}
\end{frame}

\begin{frame}{\textsc{Abs}}
	\begin{itemize}
		\item \tikzmark{red!20}{bodyTypeL}{\enquote{Damit $b$ als Funktion von $p$ typisierbar ist...}}
		\item \tikzmark{green!20}{contextL}{\enquote{... müssen wir den Typ von $p$ in den Kontext einfügen}}
	\end{itemize}

	\begin{equation*}
		\typeRule{Abs}{
			\tikzmark{green!20}{context}{$\Gamma{}, p : \pi$}
			\;
			\tikzmark{red!20}{bodyType}{$\vdash b : \rho$}
		}{
			\tikzmark{blue!20}{absType}{$\Gamma \vdash \abs{p}{b} : \pi \to \rho$}
		}
	\end{equation*}

	\begin{itemize}
		\item dann gilt:
		\item \tikzmark{blue!20}{absTypeL}{\enquote{$\abs{p}{b}$ ist eine Funktion, die $\pi$s auf $\rho$s abbildet}}
	\end{itemize}

	\tikz[overlay, remember picture]{
		\draw[->] (bodyTypeL) edge [bend left] (bodyType);
		\draw[->] (contextL) edge [bend right] (context);
		\draw[->] (absTypeL) edge [bend left] (absType);
	}
\end{frame}

\begin{frame}{Typinferenz}
	Vorgehensweise zur Typinferenz:
	\begin{itemize}
		\item Stelle Typherleitungsbaum auf
		\begin{itemize}
			\item In jedem Schritt werden neue Typvariablen $\alpha_i$ angelegt
			\item Statt die Typen direkt im Baum einzutragen, werden Gleichungen in einem Constraint-System eingetragen
		\end{itemize}
		\item Unifiziere Constraint-System zu einem Unifikator
		\begin{itemize}
			\item Robinson-Algorithmus, im Grunde wie bei Prolog
		\end{itemize}
	\end{itemize}
\end{frame}

\begin{frame}{Robinson-Algorithmus}
	\code{code/robinson-l.pseudo}

	Erzeugt Unifikator zu einem Constraint-System.
\end{frame}

\begin{frame}{Typinferenz: Übungsaufgabe}
	\begin{equation*}
		\typeRule{Abs}{
			...
		}{
			\texttt{f} : \textrm{int} \to \beta \vdash \abs{\texttt{x}}{\liin{\texttt{f}}{\texttt{x}}} : \alpha_1
		}
	\end{equation*}

	\begin{itemize}
		\item \enquote{Finde den allgemeinsten Typen $\alpha_1$ von $\abs{\texttt{x}}{\liin{\texttt{f}}{\texttt{x}}}$}
	\end{itemize}

	Erinnerung:

	\begin{itemize}
		\item Baum mit durchnummerierten $\alpha_i$ aufstellen
		\item Constraints sammeln:
	\end{itemize}

	\begin{columns}
		\scriptsize
		\begin{column}{0.3\textwidth}
			\begin{equation*}
			\typeRule{Var}{
				\Gamma(x) = \sigma
				\;\;\;\;
				\sigma \succeq \tau
			}{
				\Gamma \vdash x : \tau
			}
			\end{equation*}

			Constraint: $\{ \sigma = \tau \}$
		\end{column}
		\begin{column}{0.3\textwidth}
			\begin{equation*}
			\typeRule{App}{
				\Gamma \vdash f : \xi
				\;\;\;\;
				\Gamma \vdash x : \phi
			}{
				\Gamma \vdash \liin{f}{x} : \alpha
			}
			\end{equation*}

			Constraint: $\{ \xi = \phi \to \alpha \}$
		\end{column}
		\begin{column}{0.3\textwidth}
			\begin{equation*}
			\typeRule{Abs}{
				\Gamma, p : \pi \vdash b : \beta
			}{
				\Gamma \vdash \abs{p}{b} : \alpha
			}
			\end{equation*}

			Constraint: $\{ \alpha = \pi \to \beta \}$
		\end{column}
	\end{columns}

	\begin{itemize}
		\item Constraint-System auflösen
	\end{itemize}
\end{frame}

\begin{frame}{Typinferenz: Übungsaufgabe}
	\begin{equation*}
		\typeRule{Abs}{
			...
		}{
			\vdash \abs{\texttt{x}}{\abs{f}{\liiin{\texttt{f}}{\texttt{x}}{\texttt{x}}}} : \alpha_1
		}
	\end{equation*}

	\begin{itemize}
		\item \enquote{Finde den allgemeinsten Typen $\alpha_1$ von $\abs{\texttt{x}}{\abs{f}{\liiin{\texttt{f}}{\texttt{x}}{\texttt{x}}}}$}
	\end{itemize}

	Erinnerung:

	\begin{itemize}
		\item Baum mit durchnummerierten $\alpha_i$ aufstellen
		\item Constraints sammeln:
	\end{itemize}

	\begin{columns}
		\scriptsize
		\begin{column}{0.3\textwidth}
			\begin{equation*}
			\typeRule{Var}{
				\Gamma(x) = \sigma
				\;\;\;\;
				\sigma \succeq \tau
			}{
				\Gamma \vdash x : \tau
			}
			\end{equation*}

			Constraint: $\{ \sigma = \tau \}$
		\end{column}
		\begin{column}{0.3\textwidth}
			\begin{equation*}
			\typeRule{App}{
				\Gamma \vdash f : \xi
				\;\;\;\;
				\Gamma \vdash x : \phi
			}{
				\Gamma \vdash \liin{f}{x} : \alpha
			}
			\end{equation*}

			Constraint: $\{ \xi = \phi \to \alpha \}$
		\end{column}
		\begin{column}{0.3\textwidth}
			\begin{equation*}
			\typeRule{Abs}{
				\Gamma, p : \pi \vdash b : \beta
			}{
				\Gamma \vdash \abs{p}{b} : \alpha
			}
			\end{equation*}

			Constraint: $\{ \alpha = \pi \to \beta \}$
		\end{column}
	\end{columns}

	\begin{itemize}
		\item Constraint-System auflösen
	\end{itemize}
\end{frame}

\begin{frame}{Ende}
	\begin{itemize}
		\item Korrektur der handschriftlichen ÜB-Abgaben: Bis morgen
		\item Raum 131, 11:30
		\item Ansonsten bin ich den Rest der Woche noch häufiger am Infobau
	\end{itemize}

	Gute Ferienzeit!
\end{frame}

\end{document}
