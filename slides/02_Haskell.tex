\documentclass{beamer}
\usetheme{metropolis}

\input{common.tex}

\title{Tutorium 02: Kombinatoren, Lazyness}
% \subtitle{}
\author{David Kaufmann}
\institute{Tutorium Programmierparadigmen am KIT}
\date{09. November 2022}

\begin{document}

\begin{frame}
	\titlepage
\end{frame}

\begin{frame}{Verwirrung vom letzten Mal}
    Wir waren uns nicht sicher was wie stark bindet
    \code{../demos/Length.hs}
    
    \texttt{lengthAcc [1,2] 0}
    
    \texttt{=> (lengthAcc [2] 0) + 1}

    \texttt{=> ((lengthAcc [] 0) + 1) + 1}
\end{frame}

\begin{frame}{Notenskala}
	\begin{itemize}
		\item -
		\item Richtig, kleine Fehler
		\item Aufgabe nicht verstanden
		\item Grundansatz falsch
		\item Richtig!
		\item Richtiger Ansatz, aber unvollständig
	\end{itemize}
\end{frame}

\section{Übungsblatt 1}

\begin{frame}{Stabile Sortieralgorithmen}
    Wir sortieren diese Liste anhand dem ersten Element der Tupel
    \begin{itemize}
        \item \texttt{[(1,1),(4,2),(2,1),(5,1),(4,1)]}
    \end{itemize}
    Das ist stabil
    \begin{itemize}
        \item \texttt{[(1,1),(2,1),(4,2),(4,1),(5,1)]}
    \end{itemize}
    Das nicht 
    \begin{itemize}
        \item \texttt{[(1,1),(2,1),(4,1),(4,2),(5,1)]}
    \end{itemize}
    $\leadsto$ Elemente mit dem gleichen Key müssen ihre Reihenfolge beibehalten
\end{frame}

\section{Vorlesungswiederholung}

\begin{frame}{Listenkombinatoren}
  \begin{itemize}
    \item \texttt{foldr :: (a -> b -> b) -> b -> [a] -> b}
        \begin{itemize}
            \item Bsp.: \texttt{foldr f s [1,2,3]} berechnet \texttt{(f 1 (f 2 (f 3 s)))}
        \end{itemize}
    \item \texttt{foldl :: (b -> a -> b) -> b -> [a] -> b}
        \begin{itemize} 
            \item Bsp.: \texttt{foldl f s [1,2,3]} berechnet \texttt{(f (f (f s 1) 2) 3)}
        \end{itemize}
    \item Für beide gilt: \texttt{foldr operation initial list}
  \end{itemize}
\end{frame}

\begin{frame}{Listenkombinatoren}
    Weitere Kombinatoren:
    \begin{itemize}
        \item \texttt{map :: (a -> b) -> [a] -> [b]}
        \item \texttt{filter :: (a -> Bool) -> [a] -> [a]}
        \item \texttt{zipWith :: (a -> b -> c) -> [a] -> [b] -> [c]}
        \item \texttt{zip :: [a] -> [b] -> [(a, b)]}
        \item \texttt{and, or :: [Bool] -> Bool}
    \end{itemize}
    Idee: Statt Rekursion selbst zu formulieren verwenden wir fertige \enquote{Bausteine}, sogenannte \enquote{Kombinatoren}.
\end{frame}

\begin{frame}{Aufgaben}
	Schreibt ein Modul \texttt{Tut02} mit:

	\begin{itemize}
		\item \texttt{import Prelude hiding (foldl, foldr, map, filter, scanl, zip, zipWith)}
		\item \texttt{map} --- Einmal von Hand, einmal per Fold
		\item \texttt{filter} --- Einmal von Hand, einmal per Fold
		\item \texttt{squares l} --- Liste der Quadrate der Elemente von \texttt{l}
		\item \texttt{zip as bs} --- Erstellt Tupel der Elemente von \texttt{as} und \texttt{bs}
		\item \texttt{zipWith as bs} --- Wendet komponentenweise \texttt{f} auf die Elemente von \texttt{as} und \texttt{bs} an
		\begin{itemize}
			\item Bspw. \texttt{zipWith (+) [1, 1, 2, 3] [1, 2, 3, 5] == [2, 3, 5, 8]}
		\end{itemize}
   		\pause
		\item \texttt{foldl}
		\item \texttt{scanl f i l} --- Wie \texttt{foldl}, gibt aber eine Liste aller Akkumulatorwerte zurück
		\begin{itemize}
			\item Bspw. \texttt{scanl (*) 1 [1, 3, 5] == [1, 3, 15]}
		\end{itemize}
	\end{itemize}
\end{frame}

\begin{frame}{Lazy Evaluation}
	\code{code/ghci-lazy.output}

	\begin{itemize}
		\item Was heißt Lazy Evaluation?
		\item Wieso tritt erst bei der zweiten Eingabe ein Fehler auf?
		\pause
		\item $\leadsto$ Berechnungen finden erst statt, wenn es \emph{absolut} nötig ist
	\end{itemize}
\end{frame}

\begin{frame}{Lazy Evaluation}
      \href{https://wiki.haskell.org/Lazy\_evaluation}{wiki.haskell.org/Lazy\_Evaluation}:

	\begin{displayquote}
		Lazy evaluation means that expressions are not evaluated when they are bound to variables, but their evaluation is \textbf{deferred} until their results are needed by other computations.
	\end{displayquote}

	\begin{itemize}
		\item Auch: \emph{call-by-name} im Gegensatz zu \emph{call-by-value} in bspw. C
		\item Was bringt das?
		\pause
		\item Ermöglicht bspw. arbeiten mit unendlichen Listen
		\item Berechnungen, die nicht gebraucht werden, werden nicht ausgeführt
	\end{itemize}
\end{frame}

\begin{frame}{Lazy Evaluation}
    \begin{itemize}
        \item Arithmetische Operatoren \texttt{((+), (-), (*))} werten Argumente \textbf{immer} aus
        \item Vergleichsoperatoren \texttt{(<), (<=), (==)} werten Argumente \textbf{immer} aus
        \item Boolsche Operatoren \texttt{((&&), (||))} nutzen Short-Circuit-Auswertung
        \item Strukturelle Operatoren \texttt{((:), (,))} werten Argumente \textbf{nie} aus
    \end{itemize}
\end{frame}

\begin{frame}{List Comprehension}
    Sytnax: \texttt{[e|$q_1,...,q_m$]}
    \begin{itemize}
        \item \texttt{e} ist ein Ausdruck
        \item \texttt{$q_1,...,q_m$} sind Generatoren (\texttt{p <- xs}) oder Prädikate
    \end{itemize}
    \pause
    Die Reihenfolge ist wichtig!
    \begin{itemize}
        \item \texttt{[(x,y)| x <- [1..10], y <- [1..x]]} Funktioniert
        \item \texttt{[(x,y)| y <- [1..x], x <- [1..10]]} Kompilierfehler
        \item \texttt{[(x,y)| x <- [1..], y <- [1..]]} Immer \texttt{(1,n)}
    \end{itemize}
\end{frame}

\begin{frame}{Aufgabe}
    Erstelle eine Liste, die alle Tupel \texttt{(x,y)} mit $x,y \in \mathbb{N}_+$ genau einmal enthält.
    \pause
    
    \texttt{[(x,y)|n <- [1..], x <- [1..n-1], y <- [n-x]]}
\end{frame}

\begin{frame}{Cheatsheet: Tupel und Konzepte}
  \begin{itemize}
    \item \emph{List comprehension}, \emph{Laziness}
    \item \texttt{[f x | x <- xs, p x]} $\equiv$ \texttt{map f (filter p xs)}\\
      Bspw.: \texttt{[x * x | x <- [1..]]} $\Rightarrow$ \texttt{[1,4,9,16,25,...]}
    \item \emph{Tupel}
    \item \texttt{(,) :: a -> b -> (a, b)} (\enquote{Tupel-Konstruktor})
    \item \texttt{fst :: (a, b) -> a}
    \item \texttt{snd :: (a, b) -> b}
  \end{itemize}
\end{frame}

\end{document}
