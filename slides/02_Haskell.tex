\documentclass{beamer}
\usetheme{metropolis}

\usepackage[ngerman]{babel}
\usepackage[autostyle=true,german=quotes]{csquotes}
\usepackage[linewidth=1pt]{mdframed}
\usepackage{hyperref}
\usepackage{makecell}
\usepackage{pifont}
\usepackage{tikz}
\usetikzlibrary{positioning, calc, arrows, fit, decorations.pathreplacing, shapes, shapes.multipart, snakes}
\usepackage{verbatim}
\usepackage{textcomp}
%\usepackage{pdfpages}

\batchmode

\hypersetup{
	colorlinks,
	urlcolor=blue,
	linkcolor=black % for ToC
}
\newenvironment{qaa}[1]{
	#1

	\begin{mdframed}
		\small
}{
	\end{mdframed}
}

\newcommand{\true}{\ding{51}}
\newcommand{\false}{\ding{55}}
\newcommand{\code}[1]{
	\begin{mdframed}
		\verbatiminput{#1}
	\end{mdframed}
}

\title{Tutorium 02: Mehr Haskell}
% \subtitle{}
\author{Paul Brinkmeier}
\institute{Tutorium Programmierparadigmen am KIT}
\date{17. November 2020}

\begin{document}

\begin{frame}
	\titlepage
\end{frame}

\begin{frame}{Notenskala}
	\begin{itemize}
		\item -
		\item Richtig, kleine Fehler
		\item Aufgabe nicht verstanden
		\item Grundansatz falsch
		\item Richtig!
		\item Richtiger Ansatz, aber unvollständig
	\end{itemize}
\end{frame}

\section{Heutiges Programm}
\begin{frame}{Programm}
	\begin{itemize}
		\item Übungsblatt 1/Aufgabe 1
		\item Wiederholung der Vorlesung
		\item Aufgaben zu Haskell
	\end{itemize}
\end{frame}

\section{Übungsblatt 1}

\begin{frame}{1.1, 1.2 --- \texttt{pow1} und \texttt{pow2}}
	\code{../demos/Arithmetik1.hs}
\end{frame}

\begin{frame}{1.3 --- \texttt{pow3}}
	\code{../demos/Arithmetik2.hs}
\end{frame}

\begin{frame}{1.4 --- \texttt{root}}
	\code{../demos/Arithmetik3.hs}
\end{frame}

\begin{frame}{1.5 --- \texttt{isPrime}}
	\code{../demos/Arithmetik4.hs}
\end{frame}

\begin{frame}{Aufgaben 2 und 3}
  Teilaufgaben 2 und 3 sind noch nicht korrigiert.

  $\leadsto$ nächste Woche
\end{frame}

\section{Wiederholung: Funktionen}

\begin{frame}{Eingebaute Funktionen: Funktionen höherer Ordnung}
	\begin{itemize}
		\item Für \texttt{[ a ]}:
		\begin{itemize}
			\item \texttt{map :: (a -> b) -> [ a ] -> [ b ]}
			\item \texttt{filter :: (a -> Bool) -> [ a ] -> [ a ]}
			\item \texttt{all :: (a -> Bool) -> [ a ] -> Bool}
			\item \texttt{any :: (a -> Bool) -> [ a ] -> Bool}
			\item \texttt{foldl :: (b -> a -> b) -> b -> [ a ] -> b}
		\end{itemize}
		\item Für Funktionen:
		\begin{itemize}
			\item \texttt{(.) :: (a -> b) -> (b -> c) -> (a -> c)}
			\item \texttt{(\$) :: (a -> b) -> a -> b}
			\item \texttt{flip :: (a -> b -> c) -> (b -> a -> c)}
		\end{itemize}
	\end{itemize}
\end{frame}

\begin{frame}{Aufgaben}
	Schreibt ein Modul \texttt{Tut02} mit:

	\begin{itemize}
		\item \texttt{import Prelude ()} --- Verhindert Laden der Standardbibliothek
		\item \texttt{map}
		\item \texttt{filter}
		\item \texttt{squares l} --- Liste der Quadrate der Elemente von \texttt{l}
		\item \texttt{odd, even} --- Prüft ob eine Zahl (un-)gerade ist
		\item \texttt{odds, evens} --- Liste aller (un-)geraden Zahlen \texttt{>= 0}
		\item \texttt{foldl}
		\item \texttt{scanl f l} --- Wie \texttt{foldl}, gibt aber eine Liste aller Akkumulatorwerte zurück
		\begin{itemize}
			\item Bspw. \texttt{scanl (*) 1 [1, 3, 5] == [1, 3, 15]}
		\end{itemize}
	\end{itemize}
\end{frame}

\section{Lazy Evaluation}

\begin{frame}{Lazy Evaluation}
	\code{code/ghci-lazy.output}

	\begin{itemize}
		\item Was heißt Lazy Evaluation?
		\item Wieso tritt erst bei der zweiten Eingabe ein Fehler auf?
		\pause
		\item $\leadsto$ Berechnungen finden erst statt, wenn es \emph{absolut} nötig ist
	\end{itemize}
\end{frame}

\begin{frame}{Lazy Evaluation}
	\url{//wiki.haskell.org/Lazy\_evaluation}:

	\begin{displayquote}
		Lazy evaluation means that expressions are not evaluated when they are bound to variables, but their evaluation is \textbf{deferred} until their results are needed by other computations.
	\end{displayquote}

	\begin{itemize}
		\item Auch: \emph{call-by-name} im Gegensatz zu \emph{call-by-value} in bspw. C
		\item Was bringt das?
		\pause
		\item Ermöglicht arbeiten mit unendlichen Listen
		\item Berechnungen, die nicht gebraucht werden, werden nicht ausgeführt
	\end{itemize}
\end{frame}

\section{Hangman}

\begin{frame}{Hangman}
	\begin{itemize}
		\item \url{//pbrinkmeier.de/Hangman.hs}
		\item \texttt{showHangman} --- Zeigt aktuellen Spielstand als \texttt{String}
		\item \texttt{updateHangman} --- Bildet Usereingabe (als \texttt{String}) und alten Zustand auf neuen Zustand ab
		\item \texttt{initHangman} --- Anfangszustand, leere Liste
	\end{itemize}
\end{frame}

\begin{frame}{Hangman --- CLI-Framework}
	\code{demos/CLI.hs}

	\begin{itemize}
		\item \texttt{s} ist der Typ des Spielzustands
		\item Anfänglicher Zustand: \texttt{[]} --- leere Liste an Rateversuchen
		\item Parameter 1: \texttt{showHangman}
		\item Parameter 2: \texttt{updateHangman}
		\item Parameter 3: \texttt{initHangman}
	\end{itemize}
\end{frame}

\begin{frame}{Hangman --- Beispiele}
	\begin{itemize}
		\item \texttt{showHangman "Test" ['e'] == ". e . . | e"}
		\item \texttt{showHangman "Test" ['s', 'f'] == ". . s . | s f"}
		\item \texttt{updateHangman "f" ['a'] == ['f', 'a']}
	\end{itemize}
\end{frame}

\end{document}
