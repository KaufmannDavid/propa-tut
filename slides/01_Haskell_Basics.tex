\documentclass{beamer}
\usetheme{metropolis}

\usepackage[ngerman]{babel}
\usepackage[autostyle=true,german=quotes]{csquotes}
\usepackage[linewidth=1pt]{mdframed}
\usepackage{hyperref}
\usepackage{makecell}
\usepackage{pifont}
\usepackage{tikz}
\usetikzlibrary{positioning, calc, arrows, fit, decorations.pathreplacing, shapes, shapes.multipart, snakes}
\usepackage{verbatim}
\usepackage{textcomp}
%\usepackage{pdfpages}

\batchmode

\hypersetup{
	colorlinks,
	urlcolor=blue,
	linkcolor=black % for ToC
}
\newenvironment{qaa}[1]{
	#1

	\begin{mdframed}
		\small
}{
	\end{mdframed}
}

\newcommand{\true}{\ding{51}}
\newcommand{\false}{\ding{55}}
\newcommand{\code}[1]{
	\begin{mdframed}
		\verbatiminput{#1}
	\end{mdframed}
}

\title{Tutorium 01: Haskell Basics}
% \subtitle{}
\author{Paul Brinkmeier}
\institute{Tutorium Programmierparadigmen am KIT}
\date{21. Oktober 2019}

\begin{document}

\begin{frame}
	\titlepage
\end{frame}

\section{Heutiges Programm}
\begin{frame}{Programm}
	\begin{itemize}
		\item Haskell installieren
		\item Wiederholung der Vorlesung
		\item Aufgaben zu Haskell
	\end{itemize}
\end{frame}

\section{Organisatorisches}

\begin{frame}{Organisatorisches}
	\begin{itemize}
		\item \texttt{paul.brinkmeier@fsmi.uni-karlsruhe.de}
		\begin{itemize}
			\item Feedback
			\item Fragen zur Orga
		\end{itemize}
		\item \url{https://github.com/pbrinkmeier/pp-tut}
		\begin{itemize}
			\item Folien
			\item Codebeispiele
		\end{itemize}
		\item Bitte Laptop o.Ä. mitbringen
	\end{itemize}
\end{frame}

\begin{frame}{Übungsbetrieb}
	\begin{itemize}
		\item ProPa hat keinen Übungsschein
		\item $\leadsto$ ÜBs zur eigenen Übung!
		\item Abgabe per Praktomat
		\item \url{https://praktomat.cs.kit.edu/pp_2019_WS/tasks}
		\item Nicht-Code-Abgaben:
		\begin{itemize}
			\item Briefkasten im Infobau-UG
			\item Oder per Praktomat
			\item Zur Not per Mail
		\end{itemize}
	\end{itemize}
\end{frame}

\begin{frame}{Klausur}
	\begin{itemize}
		\item 24.03.2020, 11:00 bis 13:00
		\item Papier-Materialien dürfen mitgebracht werden!
		\item $\leadsto$ Skript, Mitschriebe, \enquote{Formelsammlung}
	\end{itemize}
\end{frame}

\section{Haskell}

\begin{frame}{GHCi}
	\code{code/ghci.output}

	\begin{itemize}
		\item Populärster, von der VL verwendeter Haskell-Compiler: GHC
		\item Interaktive Haskell-Shell: \texttt{ghci}
		\item Installation:
		\begin{itemize}
			\item Windows: Installer von Haskell-Website
			\item Linux: Je nach Distro \texttt{haskell-platform} oder \texttt{ghc} installieren
			\item macOS: \texttt{ghcup}
		\end{itemize}
	\end{itemize}
\end{frame}

\begin{frame}{Module}
	\code{demos/Maths.hs}

	\begin{itemize}
		\item Ein Haskell-Programm ist eine Folge von Funktionsdefinitionen
		\item Funktionen müssen keine Argumente haben
	\end{itemize}
\end{frame}

\begin{frame}{REPL}
	\code{code/ghci-maths.output}

	\begin{itemize}
		\item \texttt{ghci} ist ein sog. \enquote{Read-Eval-Print-Loop}
		\item \texttt{:l} --- Modul aus Datei laden
		\item \texttt{:t} --- Typ eines Audrucks abfragen
	\end{itemize}
\end{frame}

\begin{frame}{Funktionen}
	\code{demos/Maths.hs}

	\begin{itemize}
		\item Unterschied zu C-ähnlichen: Keine Klammern/Kommata, \texttt{=}
		\item Funktionsaufruf: Selbe Syntax
	\end{itemize}
\end{frame}

\begin{frame}{Basistypen}
	\begin{tabular}{ l | c | c }
		Wert & Typ in Java & Typ in Haskell \\
		\hline
		\texttt{"Hello, World!"} & \texttt{String} & \pause \texttt{String} \\
		\texttt{'x'} & \texttt{char} & \pause \texttt{Char} \\
		5 & \texttt{int} & \pause \texttt{Int} \\
		9999999999999999999999999 & \texttt{BigInteger} & \pause \texttt{Integer} \\
		3.1415927 & \texttt{float} & \pause \texttt{Float} \\
		3.141592653589793 & \texttt{double} & \pause \texttt{Double} \\
		\texttt{[Tt]rue}, \texttt{[Ff]alse} & \texttt{boolean} & \pause \texttt{Bool} \\
		$\frac{1}{3}$ & \false & \pause \texttt{Fractional a => a} \\
	\end{tabular}
\end{frame}

\begin{frame}{Basistypen}
	\code{demos/Maths.hs}

	\begin{itemize}
		\item Schreibt ein Modul \texttt{FirstSteps} mit folgenden Funktionen:
		\begin{itemize}
			\item \texttt{double x} --- verdoppelt \texttt{x}
			\item \texttt{dSum x y} --- verdoppelt \texttt{x} und \texttt{y} und summiert die Ergebnisse
			\item \texttt{area r} --- Fläche eines Kreises mit Radius \texttt{r}
			\item \texttt{sum3 a b c} --- Summiert \texttt{a}, \texttt{b} und \texttt{c}
			\item \texttt{sum4 a b c d} --- Summiert \texttt{a}, \texttt{b}, \texttt{c} und \texttt{d}
		\end{itemize}
	\end{itemize}
\end{frame}

\begin{frame}{Listen}
	\begin{itemize}
		\item \texttt{sum3}, \texttt{sum4}, \texttt{sumX} zu schreiben ist irgendwie doof
		\pause
		\item Lösung des Problems: Listen
		\item \texttt{[ a ]} ist der Typ einer Liste, deren Elemente von Typ \texttt{a} sind
		\item $\leadsto$ Listen sind homogen, nur eine Art von Element
	\end{itemize}

	\code{demos/Lists.hs}

	\begin{itemize}
		\item Ein Ausdruck des Typs \texttt{[ a ]} hat genau einen von zwei Werten:
		\begin{itemize}
			\item \texttt{[]} --- die leere Liste
			\item \texttt{(h : t)} --- Erstes El. + Rest, mit \texttt{h :: a} und \texttt{t :: [ a ]}
		\end{itemize}
	\end{itemize}
\end{frame}

\begin{frame}{Funktionstypen}
	\begin{itemize}
		\item \textbf{Funktionen sind Werte}
		\item $\leadsto$ Funktionen haben einen Typ
		\item Allgemeine Form: \texttt{x -> y}
		\item Beispiel: \texttt{length :: [ a ] -> Int}
		\begin{itemize}
			\item Java: \texttt{Function<List<A>, Integer> length;}
			\item C: \texttt{int (*strlen)(char *str)};
		\end{itemize}
	\end{itemize}
\end{frame}

\begin{frame}{Funktionstypen, mehrere Argumente}
	\code{demos/Maths.hs}

	\begin{itemize}
		\item Funktionen sind vom Typ \texttt{x -> y}
		\item Welchen Typ hat denn dann \texttt{add}? \\
		\pause
		$\leadsto$ \texttt{Num a => a -> a -> a}
		\item \texttt{->} ist rechtsassoz. \\
		$\leadsto$ \texttt{a -> a -> a} $\Leftrightarrow$ \texttt{a -> (a -> a)}
	\end{itemize}
\end{frame}

\begin{frame}{Unterversorgung}
	\begin{itemize}
		\item Haskell-Funktionen sind \enquote{ge-Curry-d}
		\item D.h.: Jede Funktion hat exakt ein Argument
		\item Funktionen mit (logisch gesehen) mehreren Argumenten geben solange Funktionen zurück, bis sie ausreichend \enquote{versorgt} sind
	\end{itemize}
\end{frame}

\begin{frame}{Typklassen}
	\code{code/ghci-typeclass.output}

	\begin{itemize}
		\item Typklassen übernehmen ähnliche Aufgaben wie Generics in Java
		\item Bspw. enthält \texttt{Ord a} alle Typen \texttt{a}, auf denen eine Ordnung definiert ist
		\begin{itemize}
			\item Java: \pause \texttt{Comparable<A>}
		\end{itemize}
	\end{itemize}
\end{frame}

\begin{frame}{Fallunterscheidung: if-then-else}
	\code{demos/MaxIf.hs}

	\begin{itemize}
		\item Einfachste Form der Fallunterscheidung
		\item \texttt{if <Bedingung> then <WertA> else <WertB>}
		\pause
		\item Das ist nichts anderes als der ternäre Operator in den C-ähnlichen
		\begin{itemize}
			\item \texttt{<Bedingung> ? <WertA> : <WertB>}
		\end{itemize}
	\end{itemize}
\end{frame}

\begin{frame}{Fallunterscheidung: Guard-Notation}
	\code{demos/MaxGuard.hs}

	\begin{itemize}
		\item \enquote{Guard}-Notation
		\item Wird einfach von oben nach unten abgearbeitet
		\item Oft kürzer als \texttt{if a then x else if b then y else z}
		\pause
		\item \texttt{otherwise == True}
	\end{itemize}
\end{frame}

\begin{frame}{Fallunterscheidung: Pattern Matching}
	\code{demos/Bool.hs}

	\begin{itemize}
		\item Statt Variablen einfach Werte in den Funktionskopf setzen
		\item Mehrere Funktionsdefinitionen möglich
		\item Funktioniert nicht immer (bspw. bei \texttt{max})
		\item Hier nützlich: \texttt{\textunderscore} ignoriert Argument
	\end{itemize}
\end{frame}

\begin{frame}{Eingebaute Funktionen}
	\begin{itemize}
		\item Für \texttt{Num a}:
		\begin{itemize}
			\item \texttt{(+)}, \texttt{(-)}, \texttt{(*)}, \texttt{(\textasciicircum)}
		\end{itemize}
		\item Für \texttt{Integral a}:
		\begin{itemize}
			\item \texttt{div}, \texttt{mod}
		\end{itemize}
		\item Für \texttt{Floating a}:
		\begin{itemize}
			\item \texttt{(/)}
		\end{itemize}
		\item Für \texttt{Bool}:
		\begin{itemize}
			\item \texttt{(\&\&)}, \texttt{(||)}
		\end{itemize}
		\item Für \texttt{Eq a}:
		\begin{itemize}
			\item \texttt{(==)}, \texttt{(/=)}
		\end{itemize}
		\item Für \texttt{Ord a}:
		\begin{itemize}
			\item \texttt{(<)}, \texttt{(<=)}, \texttt{(>)}, \texttt{(>=)}, \texttt{min}, \texttt{max}
		\end{itemize}
	\end{itemize}
\end{frame}

\begin{frame}{Eingebaute Funktionen}
	\begin{itemize}
		\item Für \texttt{[ a ]}:
		\begin{itemize}
			\item \texttt{(++) :: [ a ] -> [ a ] -> [ a ]}
			\item \texttt{(!!) :: [ a ] -> Int -> a}
			\item \texttt{head, last :: [ a ] -> a}
			\item \texttt{null :: [ a ] -> Bool}
			\item \texttt{take, drop :: Int -> [ a ] -> [ a ]}
			\item \texttt{length :: [ a ] -> Int}
			\item \texttt{reverse :: [ a ] -> [ a ]}
			\item \texttt{elem :: a -> [ a ] -> Bool}
		\end{itemize}
	\end{itemize}
\end{frame}

\begin{frame}{Eingebaute Funktionen: Funktionen höherer Ordnung}
	\begin{itemize}
		\item Für \texttt{[ a ]}:
		\begin{itemize}
			\item \texttt{map :: (a -> b) -> [ a ] -> [ b ]}
			\item \texttt{filter :: (a -> Bool) -> [ a ] -> [ a ]}
			\pause
			\item \texttt{all :: (a -> Bool) -> [ a ] -> Bool}
			\item \texttt{any :: (a -> Bool) -> [ a ] -> Bool}
			\pause
			\item \texttt{foldl :: (b -> a -> b) -> b -> [ a ] -> b}
		\end{itemize}
		\pause
		\item Für Funktionen:
		\begin{itemize}
			\item \texttt{(.) :: (a -> b) -> (b -> c) -> (a -> c)}
			\item \texttt{(\$) :: (a -> b) -> a -> b}
			\item \texttt{flip :: (a -> b -> c) -> (b -> a -> c)}
		\end{itemize}
	\end{itemize}
\end{frame}

\begin{frame}{Aufgaben}
	Schreibt ein Modul \texttt{Tut01} mit:

	\begin{itemize}
		\item \texttt{fac n} --- Berechnet Fakultät von \texttt{n}
		\item \texttt{fib n} --- Berechnet \texttt{n}-te Fibonacci-Zahl
		\item \texttt{fibs n} --- Liste der ersten \texttt{n} Fibonacci-Zahlen
		\item \texttt{fibsTo n} --- Liste der Fibonacci-Zahlen bis \texttt{n}
		\item \texttt{productL l} --- Berechnet das Produkt aller Einträge von \texttt{l}
		\item \texttt{odds} --- (Unendliche) Liste aller ungeraden natürlichen Zahlen
		\item \texttt{evens} --- (Unendliche) Liste aller geraden natürlichen Zahlen
		\item \texttt{squares l} --- Liste der Quadrate aller Einträge von \texttt{l}
	\end{itemize}
\end{frame}

\begin{frame}{Hangman}
	\begin{itemize}
		\item \url{//pbrinkmeier.de/Hangman.hs}
		\item \texttt{showHangman} --- Zeigt aktuellen Spielstand als \texttt{String}
		\item \texttt{initHangman} --- Anfangszustand, leere Liste
		\item \texttt{updateHangman} --- Bildet Usereingabe (als \texttt{String}) und alten Zustand auf neuen Zustand ab
	\end{itemize}
\end{frame}

\end{document}
