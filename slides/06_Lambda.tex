\documentclass{beamer}
\usetheme{metropolis}

\usepackage[ngerman]{babel}
\usepackage[autostyle=true,german=quotes]{csquotes}
\usepackage[linewidth=1pt]{mdframed}
\usepackage{hyperref}
\usepackage{makecell}
\usepackage{pifont}
\usepackage{tikz}
\usetikzlibrary{positioning, calc, arrows, fit, decorations.pathreplacing, shapes, shapes.multipart, snakes}
\usepackage{verbatim}
\usepackage{tabularx}
\usepackage{textcomp}
\usepackage{centernot}
\usepackage{amsmath}
\usepackage{xcolor}
%\usepackage{pdfpages}

\batchmode

\hypersetup{
	colorlinks,
	urlcolor=blue,
	linkcolor=black % for ToC
}
\newenvironment{qaa}[1]{
	#1

	\begin{mdframed}
		\small
}{
	\end{mdframed}
}

\newcommand{\true}{\ding{51}}
\newcommand{\false}{\ding{55}}
\newcommand{\code}[1]{
	\begin{mdframed}
		\verbatiminput{#1}
	\end{mdframed}
}

\title{Tutorium 06: $\lambda$-Kalkül}
% \subtitle{}
\author{Paul Brinkmeier}
\institute{Tutorium Programmierparadigmen am KIT}
\date{25. November 2019}

\begin{document}

\begin{frame}
	\titlepage
\end{frame}

\section{Heutiges Programm}
\begin{frame}{Programm}
	\begin{itemize}
		\item Church-Zahlen
		\item Übungsblatt 5
		\item Altklausuraufgaben zum $\lambda$-Kalkül
	\end{itemize}
\end{frame}

\section{Wiederholung}

\begin{frame}{$\lambda$-Terme}
	Ein Term im $\lambda$-Kalkül hat eine der drei folgenden Formen:

	\vspace{0.5cm}

	\begin{tabularx}{\textwidth}{ X | X | X }
		\textbf{Notation} & \textbf{Besteht aus}                      & \textbf{Bezeichnung} \\
		\hline
		$x$               & $x$ : Variablenname                       & Variable             \\
		\hline
		$\lambda{}p.b$    &
			\begin{tabular}[t]{@{}c@{}}$p$ : Variablenname\\$b$ : $\lambda$-Term\end{tabular}
									      & Abstraktion          \\
		\hline
		$f$ $a$           & $f$, $a$ : $\lambda$-Terme                & Funktionsanwendung   \\
	\end{tabularx}

	\vspace{0.5cm}

	\begin{itemize}
		\item \enquote{$\lambda$-Term}: rekursive Datenstruktur
	\end{itemize}
\end{frame}

\newcommand{\aeq}{\stackrel{\alpha}{=}}
\newcommand{\naeq}{\stackrel{\alpha}{\neq}}
\newcommand{\eeq}{\stackrel{\eta}{=}}

\newcolumntype{s}{>{\hsize=.8\hsize}X}

\begin{frame}[label=lambdaoverview]{Begriffe im $\lambda$-Kalkül}
	\fontsize{9pt}{13}\selectfont

	\begin{tabularx}{\textwidth}{ s | X | X }
		\textbf{Begriff} & \textbf{Formel} & \textbf{Bedeutung} \\
		\hline
		$\alpha$-Äquivalenz & $t_1 \aeq t_2$ & $t_1$, $t_2$ sind gleicher Struktur \\
		\hline
		$\eta$-Äquivalenz & $\lambda{}x.f$ $x \eeq f$ & \enquote{Unterversorgung} \\
		\hline
		Freie Variablen & $fv(\lambda{}p.b) = { b }$ & Menge der nicht durch $\lambda$s gebundenen Variablen \\
		\hline
		Substitution & $(\lambda{}p.b)\left[b\rightarrow{}c\right]=\lambda{}p.c$ & Ersetzung nicht-freier Variablen \\
		\hline
		Redex & $(\lambda{}p.b)$ $t$ & \enquote{Reducible expression} \\
		\hline
		$\beta$-Reduktion & $(\lambda{}p.b)$ $t \Rightarrow b\left[p\rightarrow{}t\right]$ & \enquote{Funktionsanwendung} \\
	\end{tabularx}
\end{frame}

\section{Church-Zahlen im $\lambda$-Kalkül}

\begin{frame}{Peano-Axiome}
	\begin{eqnarray*}
		c_0 =& ?\\
		c_1 =& s (c_0)\\
		c_2 =& s (s (c_0))\\
		c_3 =& s (s (s (c_0)))\\
		c_8 =& s (s (s (s (s (s (s (s (c_0))))))))
	\end{eqnarray*}

	\begin{enumerate}
		\item Die 0 ist Teil der natürlichen Zahlen
		\item Wenn $n$ Teil der natürlichen Zahlen ist,\\
	 	      ist auch $s(n) = n + 1$ Teil der natürlichen Zahlen
	\end{enumerate}
\end{frame}

\begin{frame}{Church-Zahlen}
	\begin{itemize}
		\item \enquote{Zahlen} im $\lambda$-Kalkül werden durch Funktionen in Normalform dargestellt
		\item $n$ $f$ $x =$ $f$ $n$-mal angewendet auf $x$
		\item Bspw. $(3$ $g$ $y) = g$ $(g $ $(g$ $y)) = g^3$ $y$\\
		      Mit $3 = \lambda{}f.\lambda{}x.f$ $(f $ $(f$ $x))$
		\item Schreibt eine $\lambda$-Funktion $succ$, die eine Church-Zahl nimmt und zu deren Nachfolger auswertet
		\pause
		\item Übertragt die Funktion in euren Haskell-Code vom letzten Mal und wertet $succ$ $c_0$ durch wiederholtes Anwenden von \texttt{normalBeta} aus
		\item Vergleicht euer Ergebnis mit dem von Wavelength
		\begin{itemize}
			\item \url{//pp.ipd.kit.edu/lehre/misc/lambda-ide/Wavelength.html}
		\end{itemize}
	\end{itemize}
\end{frame}

\section{Übungsblatt 5}

\newcommand{\E}{\;}

\newcommand{\liin}[2]{#1\E{}#2}
\newcommand{\liiin}[3]{#1\E{}#2\E{}#3}
\newcommand{\livn}[4]{#1\E{}#2\E{}#3\E{}#4}
\newcommand{\lvn}[5]{#1\E{}#2\E{}#3\E{}#4\E{}#5}

\newcommand{\lii}[2]{(#1\E{}#2)}
\newcommand{\liii}[3]{(#1\E{}#2\E{}#3)}

\newcommand{\liir}[2]{\textcolor{red}{\underline{(}}#1\E{}#2\textcolor{red}{\underline{)}}}
\newcommand{\liiir}[3]{\textcolor{red}{\underline{(}}#1\E{}#2\E{}#3\textcolor{red}{\underline{)}}}

\newcommand{\abs}[2]{\lambda{}#1.#2}

\begin{frame}{1.1 --- Klammerung im $\lambda$-Kalkül}
	\footnotesize
	\begin{eqnarray}
		\liiin{c_0}{c_1}{\liii{c_2}{c_3}{c_4}}{c_5}                   =& \pause \liin{\liir{\liir{c_0}{c_1}}{\lii{\liir{c_2}{c_3}}{c_4}}}{c_5}             \\
		\liin{\liii{c_0}{c_1}{c_2}}{\liii{c_3}{c_4}{c_5}}             =& \pause \liin{\lii{\liir{c_0}{c_1}}{c_2}}{\lii{\liir{c_3}{c_4}}{c_5}}              \\
		\livn{c_0}{c_1}{\liii{c_2}{c_3}{c_4}}{\lii{c_5}{c_6}}         =& \pause \liin{\liir{\liir{c_0}{c_1}}{\lii{\liir{c_2}{c_3}}{c_4}}}{\lii{c_5}{c_6}}  \\
		\lvn{c_0}{c_1}{\liii{c_2}{c_3}{c_4}}{c_5}{c_6}                =& \pause \liin{\liir{\liir{\liir{c_0}{c_1}}{\lii{\liir{c_2}{c_3}}{c_4}}}{c_5}}{c_6} \\
		\livn{c_0}{\lii{c_1}{\liii{c_2}{c_3}{c_4}}}{c_5}{c_6}         =& \pause \liin{\liir{\liir{c_0}{\lii{c_1}{\lii{\liir{c_2}{c_3}}{c_4}}}}{c_5}}{c_6}  \\
		\liin{(\abs{y}{\liiin{c_0}{c_1}{c_2}})}{\liii{c_3}{c_4}{c_5}} =& \pause \liin{(\abs{y}{\liin{\liir{c_0}{c_1}}{c_2}})}{\lii{\liir{c_3}{c_4}}{c_5}}  \\
		\liin{(\abs{y}{\lii{c_0}{(\abs{z}{\lii{c_1}{c_2}})}}}{\liii{c_3}{c_4}{c_5}} =& \pause \liin{(\abs{y}{\lii{c_0}{(\abs{z}{\lii{c_1}{c_2}})}}}{\lii{\liir{c_3}{c_4}}{c_5}}
	\end{eqnarray}

	\normalsize
	\begin{itemize}
		\item Funktionsaufrufe sind linksassoziativ, wie in Haskell
		\item Bzw. in Haskell sind FA linksassoz., wie im $\lambda$-Kalkül
	\end{itemize}
\end{frame}

\setcounter{equation}{0}

\begin{frame}{1.2 --- Klammerung im $\lambda$-Kalkül}
	\begin{eqnarray}
		\liin{(\abs{y}{y})}{c_0} \stackrel{?}{=} \abs{y}{\liin{y}{c_0}} \\
		\abs{y}{\lii{y}{c_0}} \stackrel{?}{=} \abs{y}{\liin{y}{c_0}}
	\end{eqnarray}

	\begin{itemize}
		\item Term 1 $\approx$ \texttt{App (Abs "y" (Var "y")) (Var "c0")}
		\item Term 2 $\approx$ \texttt{Abs "y" (App (Var "y") (Var "c0"))}
		\pause
		\item $\leadsto$ zweite Gleichung stimmt
	\end{itemize}
\end{frame}

\newcommand{\subst}[3]{(#1)\left[#2\,\to\,#3\right]}

\setcounter{equation}{0}

\begin{frame}{1.3 --- Klammerung im $\lambda$-Kalkül}
	\begin{eqnarray}
		\subst{\liin{(x)}{c_0}}{x}{\abs{y}{y}} =& \liin{(\abs{y}{y})}{c_0} \\
		\subst{\liin{x}{c_0}}{x}{(\abs{y}{y})} =& \liin{(\abs{y}{y})}{c_0} \\
		\subst{\liin{x}{c_0}}{x}{\abs{y}{y}} =& \liin{(\abs{y}{y})}{c_0}
	\end{eqnarray}

	\begin{itemize}
		\item Alle drei Substitutionen führen zum selben Ergebnis
		\item \enquote{Für beliebiges $t$ repräsentieren $t$ und $(t)$ den gleichen $\lambda$-Term} stimmt
	\end{itemize}
\end{frame}

\setcounter{equation}{0}

\begin{frame}{1.4 --- Klammerung im $\lambda$-Kalkül}
	Angenommen, $x = \liin{c_0}{c_1}$.\\
	Welche der folgenden Aussagen gelten?

	\begin{eqnarray}
		\liiin{c_0}{c_1}{c_2}                 =& \liin{x}{c_2} \\
		\liiin{c_2}{c_0}{c_1}                 =& \liin{c_2}{x} \\
		\livn{c_2}{\lii{c_3}{c_4}}{c_0}{c_1}  =& \liiin{c_2}{\lii{c_3}{c_4}}{x} \\
		\liin{c_2}{\liii{c_0}{c_1}{c_3}}{c_4} =& \liiin{c_2}{\lii{x}{c_3}}{c_4}
	\end{eqnarray}

	\pause

	\begin{itemize}
		\item 1 und 4 gelten
		\item $\liiin{c_2}{c_0}{c_1} = \liin{\liir{c_2}{c_0}}{c_1} \neq \liin{c_2}{\liir{c_0}{c_1}} = \liin{c_2}{x}$
		\item $\livn{c_2}{\lii{c_3}{c_4}}{c_0}{c_1} \stackrel{*}{\neq} \liiin{c_2}{\lii{c_3}{c_4}}{x}$
	\end{itemize}
\end{frame}

\begin{frame}{2 --- Church-Zahlen in Haskell}
	\code{demos/ChurchNumbers.hs}
\end{frame}

\section{Klausuraufgaben zum $\lambda$-Kalkül}

\subsection{19SS A4 --- SKI-Kalkül}

\begin{frame}{19SS A4 --- SKI-Kalkül (13P.)}
	\begin{eqnarray*}
		S =& \abs{x}{\abs{y}{\abs{z}{\liiin{x}{z}{\lii{y}{z}}}}} \\
		K =& \abs{x}{\abs{y}{x}} \\
		I =& \abs{x}{x}
	\end{eqnarray*}

	\begin{itemize}
		\item SKI-Kalkül kann alles, was $\lambda$-Kalkül auch kann, mit 3 \enquote{Kombinatoren}
		\item Definiere $U = \abs{x}{\liiin{x}{S}{K}}$
		\item Aufgabe: Beweise, dass man $S$, $K$ und $I$ durch $U$ darstellen kann
		\pause
		\begin{itemize}
			\item $\liiin{U}{U}{x} \stackrel{?}{\implies} x = \liin{I}{x}$
			\item $\liin{U}{\lii{U}{\lii{U}{U}}} = \liin{U}{\lii{U}{I}} \stackrel{?}{\implies} K$
			\item $\liin{U}{\lii{U}{\lii{U}{\lii{U}{U}}}} = \liin{U}{K} \stackrel{?}{\implies} S$
		\end{itemize}
	\end{itemize}
\end{frame}

\subsection{18WS A5 --- Listen im $\lambda$-Kalkül}

\begin{frame}{18WS A5 --- Listen im $\lambda$-Kalkül}
	\begin{eqnarray*}
		nil  =& \abs{n}{\abs{c}{n}} \\
		cons =& \abs{x}{\abs{xs}{\abs{n}{\abs{c}{\liiin{c}{x}{xs}}}}}
	\end{eqnarray*}

	\begin{itemize}
		\item Schreibe $head$ und $tail$, sodass:
		\begin{itemize}
			\item $\liin{head}{\liii{cons}{A}{B}} \stackrel{*}{\implies} A$
			\item $\liin{tail}{\liii{cons}{A}{B}} \stackrel{*}{\implies} B$
		\end{itemize}
		\pause
		\item Schreibe $replicate$, sodass:
		\begin{itemize}
			\item $\liiin{replicate}{c_n}{A} = \underbrace{\liiin{cons}{A}{(\lvn{cons}{A}{...}{\liii{cons}{A}{nil}}{...})}}_\text{$n$ mal}$
			\item Erinnerung: $\liiin{c_n}{f}{x} = \underbrace{\liin{f}{(\livn{f}{...}{\lii{f}{x}}{...})}}_\text{$n$ mal}$
		\end{itemize}
		\pause
		\item Werte aus: $\liiin{replicate}{c_3}{A} \stackrel{*}{\implies} ?$
	\end{itemize}
\end{frame}

\subsection{18SS A4 --- Currying im $\lambda$-Kalkül}

\begin{frame}{18SS A4 --- Currying im $\lambda$-Kalkül}
\end{frame}

\end{document}
