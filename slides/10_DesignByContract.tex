\documentclass{beamer}
\usetheme{metropolis}

\usepackage[ngerman]{babel}
\usepackage[autostyle=true,german=quotes]{csquotes}
\usepackage[linewidth=1pt]{mdframed}
\usepackage{hyperref}
\usepackage{makecell}
\usepackage{pifont}
\usepackage{tikz}
\usetikzlibrary{positioning, calc, arrows, fit, decorations.pathreplacing, shapes, shapes.multipart, snakes}
\usepackage{verbatim}
\usepackage{textcomp}
\usepackage{centernot}
\usepackage{tabularx}
\usepackage{ulem}
%\usepackage{pdfpages}

\batchmode

\hypersetup{
	colorlinks,
	urlcolor=blue,
	linkcolor=black % for ToC
}
\newenvironment{qaa}[1]{
	#1

	\begin{mdframed}
		\small
}{
	\end{mdframed}
}

\newcommand{\true}{\ding{51}}
\newcommand{\false}{\ding{55}}
\newcommand{\code}[1]{
	\begin{mdframed}
		\verbatiminput{#1}
	\end{mdframed}
}


\title{Tutorium 10: Design by Contract}
% \subtitle{}
\author{Paul Brinkmeier}
\institute{Tutorium Programmierparadigmen am KIT}
\date{26. Januar 2021}

\begin{document}

\begin{frame}
	\titlepage
\end{frame}

\begin{frame}{Cheatsheet: Design by Contract}
	ProPa-Stoff zu Design by Contract:

	\begin{itemize}
		\item Grundlagen: Pre-/Postconditions, Caller, Callee
		\begin{itemize}
                  \item A.K.A.: \emph{Vor-/Nachbedingungen}, \emph{Aufrufer}, \emph{Aufgerufener}
		\end{itemize}
		\item JML (Java Modeling Language):
		\begin{itemize}
			\item \texttt{@ requires}
			\item \texttt{@ ensures} (mit \texttt{\string\old} und \texttt{\string\result})
			\item \texttt{@ invariant}
			\item \texttt{/*@ pure @*/}, \texttt{/*@ nullable @*/}, \texttt{/*@ spec\_public @*/}
			\item Quantoren: \texttt{\string\forall}, \texttt{\string\exists}
                \end{itemize}
                item Liskovsches Substitutionsprinzip
	\end{itemize}
\end{frame}

\section{JML-Klausuraufgabe}

\begin{frame}{JML-Klausuraufgabe}
    Klausur 19SS, Aufgabe 6d (3P.)

    {
    \footnotesize
    \code{code/19ss-a6d.java}

    (d) Der Vertrag der Methode \texttt{combine} wird \emph{vom Aufgerufenen} verletzt.
    Begründen Sie dies und geben Sie an, wie die verletzte Nachbedingung angepasst werden könnte.
    }
\end{frame}

\begin{frame}{JML-Klausuraufgabe}
    Klausur 19SS, Aufgabe 6e (2P.)

    {
    \footnotesize
    \code{code/19ss-a6e.java}

    (d) Wird der Vertrag hier \emph{vom Aufrufer} erfüllt?
    Begründen Sie kurz.
    }
\end{frame}

\section{JML}

\begin{frame}{\texttt{@ requires}}
	\code{code/jml/requires.java}

	\begin{itemize}
		\item \texttt{@ requires} definiert eine Vorbedingung für eine Methode.
		\item Vorbedingungen müssen vom Aufrufer erfüllt werden.
	\end{itemize}
\end{frame}

\begin{frame}{\texttt{@ ensures}}
	\code{code/jml/ensures.java}

	\begin{itemize}
		\item \texttt{@ ensures} definiert eine Nachbedingung für eine Methode.
		\item Nachbedingungen müssen vom Aufgerufenen erfüllt werden.
        \item Mit \texttt{\string\old} und \texttt{\string\result} werden Beziehungen zwischen Ursprungszustand, Rückgabewert und neuem Zustand eingeführt.
	\end{itemize}
\end{frame}

\begin{frame}{\texttt{@ invariant}}
	\code{code/jml/invariant.java}

	\begin{itemize}
		\item \texttt{@ invariant} definiert Invarianten für eine Klasse.
		\item Diese können bspw. wiederverwendet werden, um Vorbedingungen für Methoden zu erfüllen.
	\end{itemize}
\end{frame}

\begin{frame}{\texttt{/*@ pure @*/}}
	\code{code/jml/pure.java}

	\begin{itemize}
		\item Verträge sind implizit \texttt{public}.\\
		$\leadsto$ \texttt{private}-Attribute nicht verwendbar
		\item Um Getter-Funktionen in Verträgen nutzen zu können, müssen diese frei von Seiteneffekten und mit \texttt{/*@ pure *@/} markiert sein.
	\end{itemize}
\end{frame}

\begin{frame}{\texttt{/*@ spec\_public @*/}}
	\code{code/jml/specpublic.java}

	\begin{itemize}
		\item Alternative: \texttt{private}-Attribute als \texttt{/*@ spec\_public @*/} markieren.
		\item Immer noch \texttt{private}, können vom Checker aber trotzdem gesehen werden.
	\end{itemize}
\end{frame}

\begin{frame}{Quantoren, logische Operatoren}
	\code{code/jml/quantors.java}

	\begin{itemize}
		\item Für das Arbeiten mit Aussagen in Verträgen gibt es ein paar Helferchen:
		\begin{itemize}
			\item \texttt{\string\forall <decl>; <cond>; <expr>}
			\item \texttt{\string\exists <decl>; <cond>; <expr>}
			\item \texttt{<cond> ==> <expr>}
		\end{itemize}
	\end{itemize}
\end{frame}

\begin{frame}{Übungsaufgabe 1 --- \texttt{Set}}
	\begin{itemize}
		\item \texttt{demos/java/set/Set.java}
		\item Behebt alle Compiler- und Laufzeitfehler in der Klasse \texttt{Set}.
		\pause
		\item Achtet darauf, dass die gegebenen JML-Verträge erfüllt sind.
		\pause
		\item Fügt je mind. eine (sinnvolle) Vor- oder Nachbedingung zu folgenden Methoden hinzu:
		\begin{itemize}
			\item \texttt{size()}
			\item \texttt{isEmpty()}
			\item \texttt{add()}
			\item \texttt{contains()}
			\item \texttt{getElements()}
		\end{itemize}
	\end{itemize}

        \vfill

        \begin{itemize}
          \item OpenJML bekommt ihr auf \href{https://openjml.org}{openjml.org}
          \item Zum automatischen Prüfen der Verträge:
        \end{itemize}
        \footnotesize
        \texttt{java -jar .../openjml.jar -exec .../z3-4.7.1 -esc <file>.java}
\end{frame}

\begin{frame}{Übungsaufgabe 2 --- Gefilterte Summe (18SS)}
	\code{code/filteredsum.pseudo}

	\begin{itemize}
		\item Wie lässt sich ein \enquote{Speedup} von 0.005 gegenüber einer Lösung ohne Threads erklären, wenn $|\textrm{\texttt{xs}}| \approx 10^7$ (3P.)?
		\item Ersetzen Sie [die entsprechenden Zeilen] durch eine Implementierung mit möglichst hohem Speedup (7,5P.).
		\item Geben Sie eine Implementierung mittels parallelen Streams an (2,5P.).
		\item Zusätzlich: DbC-Fragen für 5P.
	\end{itemize}
\end{frame}

\begin{frame}{Übungsaufgabe 3}
	\begin{itemize}
		\item \texttt{demos/java/bruteforce/Bruteforce.java}
		\item \texttt{new Bruteforce(w).countZeroStartSequential(k)} berechnet die Anzahl der \texttt{w}-stelligen kleinbuchstabigen Worte, deren SHA-256-Hash mit \texttt{k} 0-Bytes beginnt.
		\item Ist dieses Problem parallelisierbar?
		\pause
		\item $\leadsto$ Ja; daten-/taskparallel?
		\pause
		\item $\leadsto$ Datenparallel; Eingabemenge ist auf identische Ausführungsträger (Threads) aufteilbar.
		\pause
		\item Implementiert:\\
		      \texttt{new Bruteforce(w).countZeroStartParallel(k, n)}, wobei \texttt{n} die Zahl der verwendeten Threads bezeichnet.
	        \item Vergleicht \texttt{Sequential} mit \texttt{Parallel}\\
		      für $\textrm{\texttt{w}} \in \{ 5, 6 \}$, $\textrm{\texttt{k}} = 2$, $\textrm{\texttt{n}} \in \{ 1, 2, 4 \}$.
	\end{itemize}
\end{frame}


\section{Ende}

\begin{frame}{Ende}
	\begin{itemize}
		\item Im Campus-System kann man sich bis zum 17.03. für die ProPa-Klausur anmelden
		\item \href{https://campus.studium.kit.edu/renewal/payment.php}{Rückmelden} bis zum 15.02.
	\end{itemize}
\end{frame}

\end{document}
