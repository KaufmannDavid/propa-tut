\documentclass{beamer}
\usetheme{metropolis}

\usepackage[ngerman]{babel}
\usepackage[autostyle=true,german=quotes]{csquotes}
\usepackage[linewidth=1pt]{mdframed}
\usepackage{hyperref}
\usepackage{makecell}
\usepackage{pifont}
\usepackage{tikz}
\usetikzlibrary{positioning, calc, arrows, fit, decorations.pathreplacing, shapes, shapes.multipart, snakes}
\usepackage{verbatim}
\usepackage{textcomp}
\usepackage{centernot}
\usepackage{tabularx}
\usepackage{ulem}
%\usepackage{pdfpages}

\batchmode

\hypersetup{
	colorlinks,
	urlcolor=blue,
	linkcolor=black % for ToC
}
\newenvironment{qaa}[1]{
	#1

	\begin{mdframed}
		\small
}{
	\end{mdframed}
}

\newcommand{\true}{\ding{51}}
\newcommand{\false}{\ding{55}}
\newcommand{\code}[1]{
	\begin{mdframed}
		\verbatiminput{#1}
	\end{mdframed}
}


\title{Tutorium 12: Actor Model \& Design by Contract}
% \subtitle{}
\author{David Kaufmann}
\institute{Tutorium Programmierparadigmen am KIT}
\date{01. Februar 2022}

\begin{document}

\begin{frame}
    \titlepage
\end{frame}

\section{Actor Model}
\begin{frame}{Actor Model}
    \begin{itemize}
        \item Actors sind computation units mit State, Behaviour, Mailbox
        \item kommunizieren über Nachrichten
        \item verarbeiten immer nur eine Nachricht
        \item verarbeiten Nachrichten in der Reihenfolge in der sie empfangen wurden
    \end{itemize}
\end{frame}

\begin{frame}{HelloWorldActor}
    \footnotesize
    \code{./code/actors/HelloWorldActor.java}
\end{frame}

\begin{frame}{Further Methods}
    \begin{itemize}
        \item \texttt{preStart(), postStop(), preRestart(), postRestart()}
        \item \texttt{getSelf()}: Referenz auf sich selbst
        \item \texttt{getContext()}: Context um weiter Actoren zu erzeugen
        \item \texttt{getSender()}: Sender der aktuell verarbeiteten Nachricht
    \end{itemize}
\end{frame}

\begin{frame}{Actor Creation}
    Muss auf einen \textbf{Context} aufgerufen werden, entweder \texttt{ActorSystem} oder \texttt{getContext()} von innerhalb eines Actors.
    
    Actor beaufsichtigt alle Aktoren die er erstellt hat
    \footnotesize
    \code{./code/actors/CreateActor.java} 
\end{frame}

\begin{frame}{Messages}
    Muss auf den Empfänger aufgerufen werden
    \begin{itemize}
        \item \texttt{tell(Object message, ActorRef sender)}: asyncron, nicht blockierend
        \item \texttt{Future<?> Patterns.ask(ActorRef target, Object msg, Timeout timeout)}: kann awaited werden, sollte man aber vermeiden
    \end{itemize}
\end{frame}

\begin{frame}{Running Actors}
    \begin{itemize}
        \item \texttt{ActorSystem.create(String name)}: Erzeugt ein ActorSystem
        \item \texttt{void stop(ActorRef actorToStop)}: Muss auf eine ActorRefFactory aufgerufen werden (ActorSystem, Context)
        \item \texttt{PoisonPill.getInstance()}: Kann als Nachricht an einen Actor gesendet werden
        \item \texttt{ActorSystem.terminater()}: Terminiert ActorSystem
    \end{itemize}
\end{frame}

\section{Klausuraufgabe SS21}

\section{JML-Klausuraufgabe}

\begin{frame}{Cheatsheet: Design by Contract}
	ProPa-Stoff zu Design by Contract:

	\begin{itemize}
		\item Grundlagen: Pre-/Postconditions, Caller, Callee
		\begin{itemize}
                  \item A.K.A.: \emph{Vor-/Nachbedingungen}, \emph{Aufrufer}, \emph{Aufgerufener}
		\end{itemize}
		\item JML (Java Modeling Language):
		\begin{itemize}
			\item \texttt{@ requires}
			\item \texttt{@ ensures} (mit \texttt{\string\old} und \texttt{\string\result})
			\item \texttt{@ invariant}
			\item \texttt{/*@ pure @*/}, \texttt{/*@ nullable @*/}, \texttt{/*@ spec\_public @*/}
			\item Quantoren: \texttt{\string\forall}, \texttt{\string\exists}
			\item ...
		\end{itemize}
		\item Liskovsches Substitutionsprinzip
	\end{itemize}
\end{frame}


\begin{frame}{JML-Klausuraufgabe}
    Klausur 19SS, Aufgabe 6d (3P.)

    {
    \footnotesize
    \code{code/19ss-a6d.java}

    (d) Der Vertrag der Methode \texttt{combine} wird \emph{vom Aufgerufenen} verletzt.
    Begründen Sie dies und geben Sie an, wie die verletzte Nachbedingung angepasst werden könnte.
    }
\end{frame}

\begin{frame}{JML-Klausuraufgabe}
    Klausur 19SS, Aufgabe 6e (2P.)

    {
    \footnotesize
    \code{code/19ss-a6e.java}

    (d) Wird der Vertrag hier \emph{vom Aufrufer} erfüllt?
    Begründen Sie kurz.
    }
\end{frame}

\section{JML}

\begin{frame}{\texttt{@ requires}}
	\code{code/jml/requires.java}

	\begin{itemize}
		\item \texttt{@ requires} definiert eine Vorbedingung für eine Methode.
		\item Vorbedingungen müssen vom Aufrufer erfüllt werden.
	\end{itemize}
\end{frame}

\begin{frame}{\texttt{@ ensures}}
	\code{code/jml/ensures.java}

	\begin{itemize}
		\item \texttt{@ ensures} definiert eine Nachbedingung für eine Methode.
		\item Nachbedingungen müssen vom Aufgerufenen erfüllt werden.
        \item Mit \texttt{\string\old} und \texttt{\string\result} werden Beziehungen zwischen Ursprungszustand, Rückgabewert und neuem Zustand eingeführt.
	\end{itemize}
\end{frame}

\begin{frame}{\texttt{@ invariant}}
	\code{code/jml/invariant.java}

	\begin{itemize}
		\item \texttt{@ invariant} definiert Invarianten für eine Klasse.
		\item Diese können bspw. wiederverwendet werden, um Vorbedingungen für Methoden zu erfüllen.
	\end{itemize}
\end{frame}

\begin{frame}{\texttt{/*@ pure @*/}}
	\code{code/jml/pure.java}

	\begin{itemize}
		\item Verträge sind implizit \texttt{public}.\\
		$\leadsto$ \texttt{private}-Attribute nicht verwendbar
		\item Um Getter-Funktionen in Verträgen nutzen zu können, müssen diese frei von Seiteneffekten und mit \texttt{/*@ pure @*/} markiert sein.
	\end{itemize}
\end{frame}

\begin{frame}{\texttt{/*@ spec\_public @*/}}
	\code{code/jml/specpublic.java}

	\begin{itemize}
		\item Alternative: \texttt{private}-Attribute als \texttt{/*@ spec\_public @*/} markieren.
		\item Immer noch \texttt{private}, können vom Checker aber trotzdem gesehen werden.
	\end{itemize}
\end{frame}

\begin{frame}{Quantoren, logische Operatoren}
	\code{code/jml/quantors.java}

	\begin{itemize}
		\item Für das Arbeiten mit Aussagen in Verträgen gibt es ein paar Helferchen:
		\begin{itemize}
			\item \texttt{\string\forall <decl>; <cond>; <expr>}
			\item \texttt{\string\exists <decl>; <cond>; <expr>}
			\item \texttt{<cond> ==> <expr>}
		\end{itemize}
	\end{itemize}
\end{frame}

\begin{frame}{Übungsaufgabe 1 --- \texttt{Set}}
	\begin{itemize}
		\item \texttt{demos/java/jml/Set.java}
		\item Behebt alle Compiler- und Laufzeitfehler in der Klasse \texttt{Set}.
		\pause
		\item Achtet darauf, dass die gegebenen JML-Verträge erfüllt sind.
		\pause
		\item Fügt je mind. eine (sinnvolle) Vor- oder Nachbedingung zu folgenden Methoden hinzu:
		\begin{itemize}
			\item \texttt{size()}
			\item \texttt{isEmpty()}
			\item \texttt{add()}
			\item \texttt{contains()}
		\end{itemize}
	\end{itemize}
\end{frame}

\section{Ende}

\end{document}
